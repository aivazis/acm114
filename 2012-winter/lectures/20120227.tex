% -*- LaTeX -*-
% -*- coding: utf-8 -*-
%
% michael a.g. aïvázis
% california institute of technology
% (c) 1998-2012 all rights reserved
%

\lecture{The shared memory implementation}{20120227}

% --------------------------------------
% required changes to the sequential driver
\begin{frame}[fragile]
%
  \frametitle{Required changes to the sequential solution}
%
  \begin{itemize}
%
  \item what is needed
    \begin{itemize}
    \item an object to hold the problem information shared among the threads
    \item the per-thread administrative data structure that holds the thread id and the pointer
      to the shared information
      \begin{itemize}
      \item this is the argument to \function{pthread\_create}
      \end{itemize}
    \item a mutex to protect the update of the global convergence criterion
    \item a \function{pthread\_create} compatible worker routine
    \item a change at the top-level driver to enable the user to choose the number of threads
    \end{itemize}
%
  \item and a strategy for managing the thread life cycle
    \begin{itemize}
%
    \item synchronization is trivial if
      \begin{itemize}
      \item we spawn our threads to perform the updates of a single iteration
      \item harvest them
      \item check the convergence criterion 
      \item stop, or respawn them if another iteration is necessary
      \end{itemize}
%
    \item can the convergence test be done in parallel?
      \begin{itemize}
        \item so we don't have to pay the create/harvest overhead?
        \item if so, how do we guarantee correctness and consistency?
      \end{itemize}
%
    \end{itemize}
% 
  \end{itemize}
%
\end{frame}

% --------------------------------------
% the jacobi solver 
\begin{frame}[fragile]
% 
  \frametitle{Threaded \class{Jacobi}: thread data}
%
  \begin{lstlisting}[language=c++,name=Jacobi:threaded]
struct Task {
    // shared information 
    size_t workers;
    Grid & current;
    Grid & next;
    double maxDeviation;
    // mutex to control access to the convergence criterion
    pthread_mutex_t lock; 

    // constructor
    Task(size_t workers, Grid & current, Grid & next) :
        workers(workers), current(current), next(next), maxDeviation(0.0) {
        pthread_mutex_init(&lock, 0);
    }
    // destructor
    ~Task() {
        pthread_mutex_destroy(&lock);
    }
};

struct Context {
    // thread info
    size_t id;
    pthread_t descriptor;
    Task * task;
};

  \end{lstlisting}
%
\end{frame}

% --------------------------------------
% the jacobi solver 
\begin{frame}[fragile]
% 
  \frametitle{Threaded \class{Jacobi}: driving the update}
%
  \begin{lstlisting}[language=c++,name=Jacobi:threaded]
void Jacobi::solve(Problem & problem) {
    // initialize the problem
    problem.initialize();
    // do the actual solve
    _solve(problem);
    // compute and store the error
    std::cout << "  computing absolute error" << std::endl;
    //  compute the relative error
    Grid & error = problem.error();
    const Grid & exact = problem.exact();
    const Grid & solution = problem.solution();

    for (size_t j=0; j < exact.size(); j++) {
        for (size_t i=0; i < exact.size(); i++) {
            if (exact(i,j) == 0.0) {
                error(i,j) = std::abs(solution(i,j));
            } else {
                error(i,j) = std::abs(solution(i,j) - exact(i,j))/exact(i,j);
            }
        }
    }
    std::cout << " --- done." << std::endl;
    return;
}
  \end{lstlisting}
%
\end{frame}

% --------------------------------------
% the jacobi solver 
\begin{frame}[fragile]
% 
  \frametitle{Threaded \class{Jacobi}: the master thread}
%
  \begin{lstlisting}[language=c++,name=Jacobi:threaded]
void Jacobi::_solve(Problem & problem) {
    Grid & current = problem.solution();

    // create and initialize temporary storage
    Grid next(current.size());
    problem.initialize(next);

    // shared thread info
    Task task(_workers, current, next);
    // per-thread information
    Context context[_workers];

    // let's get going
    std::cout << "jacobi: tolerance=" << _tolerance << std::endl;

    // put an upper bound on the number of iterations
    const size_t max_iterations = (size_t) 1.0e4;
  \end{lstlisting}
%
\end{frame}

% --------------------------------------
% the jacobi solver 
\begin{frame}[fragile]
% 
  \frametitle{Threaded \class{Jacobi}: the master thread, part 2}
%
  \begin{lstlisting}[language=c++,name=Jacobi:threaded,basicstyle=\tt\bfseries\tiny]
    for (size_t iterations = 0; iterations<max_iterations; iterations++) {
        if (iterations % 100 == 0) {
            std::cout << "     " << iterations << std::endl;
        }
        // reset the maximium deviation
        task.maxDeviation = 0.0;
        // spawn the threads
        for (size_t tid=0; tid < _workers; tid++) {
            context[tid].id = tid;
            context[tid].task = &task;

            int status = pthread_create(&context[tid].descriptor, 0, _update, &context[tid]);
            if (status) {
                throw ("error in pthread_create");
            }
        }
        // harvest the threads
        for (size_t tid = 0; tid < _workers; tid++) {
            pthread_join(context[tid].descriptor, 0);
        }

        // swap the blocks between the two grids
        Grid::swapBlocks(current, next);
        // check covergence
        if (task.maxDeviation < _tolerance) {
            std::cout << " ### convergence in " << iterations << " iterations!" << std::endl;
            break;
        }
    }
    std::cout << " --- done." << std::endl;

    return;
}
  \end{lstlisting}
%
\end{frame}

% --------------------------------------
% the jacobi solver 
\begin{frame}[fragile]
% 
  \frametitle{Threaded \class{Jacobi}: update in the worker threads}
%
  \begin{lstlisting}[language=c++,basicstyle=\tt\bfseries\tiny,name=Jacobi:threaded]
void * Jacobi::_update(void * arg) {
    Context * context = static_cast<Context *>(arg);

    size_t id = context->id;
    Task * task = context->task;

    size_t workers = task->workers;
    Grid & current = task->current;
    Grid & next = task->next;
    pthread_mutex_t lock = task->lock;

    double max_dev = 0.0;
    // do an iteration step
    // leave the boundary alone
    // iterate over the interior of the grid
    for (size_t j=id+1; j < current.size()-1; j+=workers) {
        for (size_t i=1; i < current.size()-1; i++) {
            next(i,j) = 0.25*(current(i+1,j)+current(i-1,j)+current(i,j+1)+current(i,j-1));
            // compute the deviation from the last generation
            double dev = std::abs(next(i,j) - current(i,j));
            // and update the maximum deviation
            if (dev > max_dev) {
                max_dev = dev;
            }
        }
    }

    // grab the lock and update the global maximum deviation
    pthread_mutex_lock(&lock);
    if (task->maxDeviation < max_dev) {
        task->maxDeviation = max_dev;
    }
    pthread_mutex_unlock(&lock);

    return 0;
}
  \end{lstlisting}
%
\end{frame}

% --------------------------------------
% improving the threaded implementation
\begin{frame}[fragile]
%
  \frametitle{Assessing the threaded implementation}
%
  \begin{itemize}
%
  \item the implemented synchronization scheme is very simple
    \begin{itemize}
    \item each grid update step spawns some number of workers to update a subset of the cells
    \item the workers are harvested after the grid is updated
    \item the main thread checks for convergence
    \item if another iteration is required, a new set of workers is spawned
    \end{itemize}
%
  \item the simplicity of this strategy comes at a cost
    \begin{itemize}
    \item {\em scalability} suffers when the overhead of creating and harvesting threads is
      comparable to amount of work done by each thread
    \item for low thread counts, it is still an overall win, since the time to solution
      decreases and the machine utilization is better
    \item but as the number of threads increases, the program becomes {\em slower}
      \begin{itemize}
      \item timing a $100\times100$ grid to convergence on a recent MacPro
%
        \begin{equation*}
          \begin{array}{r|ccccc}
% header
            {\rm threads } & 1 & 2 & 4 & 8 & 16 \\
            \hline 
            {\rm time} (s) & 4.367 & 2.517 & 1.918 & 1.937 & 3.537
% 
          \end{array}
        \end{equation*}
      \item and 10,000 iterations of a $1000\times1000$ grid
        \begin{equation*}
          \begin{array}{r|ccccc}
% header
            {\rm threads } & 1 & 2 & 4 & 8 & 16 \\
            \hline 
            {\rm time} (s) & 413.306 & 211.050 & 109.509 & 98.279 & 74.087
% 
          \end{array}
        \end{equation*}
      \end{itemize}

    \end{itemize}
%
\end{itemize}
%
\end{frame}

% --------------------------------------
% Improving the update loop
\begin{frame}[fragile]
%
  \frametitle{Improving the update loop}
%
  \begin{itemize}
%
  \item the plan is to keep the workers alive and updating the grid while either we converge or
    \identifier{max\_iterations} is reached
  \item the main thread
    \begin{itemize}
    \item loops to spawn all the threads
    \item and immediately enters a loop to harvest them
    \end{itemize}
%
  \item the workers use a condition variable to synchronize among themselves
    \begin{itemize}
    \item they iterate, updating the grid
    \item grab a mutex, deposit their local maximum deviation from the last iterations, update
      a counter that records how many workers have completed their update, and release the lock
    \item enter another critical section with the termination logic
      \begin{itemize}
      \item everybody uses a condition variable to wait for the slowest worker
      \item the slowest worker checks the convergence criterion and updates the termination
        flag, swaps the grid blocks and signals everybody else
      \item if the termination flag is set, or if the maximum number of iterations has been
        reached, all threads exit
      \end{itemize}
    \end{itemize}
%
  \end{itemize}
%
\end{frame}

% --------------------------------------
% the jacobi solver 
\begin{frame}[fragile]
% 
  \frametitle{Threaded \class{Jacobi}: the main thread}
%
  \begin{lstlisting}[language=c++,name=Jacobi:updated-solve,basicstyle=\tt\bfseries\tiny]
void Jacobi::_solve(Problem & problem) {
    Grid & current = problem.solution();

    // create and initialize temporary storage
    Grid next(current.size());
    problem.initialize(next);

    // shared thread info
    Task task(_workers, _tolerance, current, next);
    // per-thread information
    Context context[_workers];
    // spawn the threads
    std::cout << "jacobi: spawning " << _workers << " workers" << std::endl;
    for (size_t tid=0; tid < _workers; tid++) {
        context[tid].id = tid;
        context[tid].task = &task;
        
        int status = pthread_create(&context[tid].descriptor, 0, _update, &context[tid]);
        if (status) {
            throw ("error in pthread_create");
        }
    }
    // harvest the threads
    for (size_t tid = 0; tid < _workers; tid++) {
            pthread_join(context[tid].descriptor, 0);
        }
    // done
    std::cout << "jacobi: done." << std::endl;
    return;
}
  \end{lstlisting}
%
\end{frame}

% --------------------------------------
% the jacobi solver 
\begin{frame}[fragile]
% 
  \frametitle{Threaded \class{Jacobi}: updated thread data}
%
  \begin{lstlisting}[language=c++,name=Jacobi:updated-threaded,basicstyle=\tt\bfseries\tiny]
struct Task {
    // shared information 
    size_t workers; // the number of threads
    double tolerance; // the covergence tolerance
    Grid & current;
    Grid & next;

    bool done; // is there more work?
    double maxDeviation; // the value
    size_t contributions; // the number of threads that have deposited contributions
    pthread_mutex_t gridUpdate_lock; //the mutex
    pthread_cond_t gridUpdate_check;

    Task(size_t workers, double tolerance, Grid & current, Grid & next) :
        workers(workers), tolerance(tolerance), current(current), next(next),
        done(false), maxDeviation(0.0), contributions(0),
        gridUpdate_lock(), gridUpdate_check() {
        // initialize the grid update lock
        pthread_mutex_init(&gridUpdate_lock, 0);
        pthread_cond_init(&gridUpdate_check, 0);
    }

    ~Task() {
        pthread_mutex_destroy(&gridUpdate_lock);
        pthread_cond_destroy(&gridUpdate_check);
    }
};

  \end{lstlisting}
%
\end{frame}

% --------------------------------------
% the jacobi solver 
\begin{frame}[fragile]
% 
  \frametitle{Threaded \class{Jacobi}: workers, part 1}
%
  \begin{lstlisting}[language=c++,name=Jacobi:updated-solve,basicstyle=\tt\bfseries\tiny]
// the threaded update
void * Jacobi::_update(void * arg) {
    Context * context = static_cast<Context *>(arg);

    size_t id = context->id;
    Task * task = context->task;

    const size_t workers = task->workers;
    Grid & current = task->current;
    Grid & next = task->next;

    size_t maxIterations = (size_t) 1e4;
    // iterate, updating the grid until done
    for (size_t iteration = 0; iteration < maxIterations; iteration++) {
        // thread 0: print an update
        if (id == 0 && iteration % 100 == 0) {
            std::cout << "    " << iteration << std::endl;
        }

        double max_dev = 0.0;
        // do an iteration step
        // leave the boundary alone
        // iterate over the interior of the grid
        for (size_t j=id+1; j < current.size()-1; j+=workers) {
            for (size_t i=1; i < current.size()-1; i++) {
                next(i,j) = 0.25*(current(i+1,j)+current(i-1,j)+current(i,j+1)+current(i,j-1));
                // compute the deviation from the last generation
                double dev = std::abs(next(i,j) - current(i,j));
                // and update the maximum deviation
                if (dev > max_dev) {
                    max_dev = dev;
                }
            }
        }
        // done with the grid update
  \end{lstlisting}
%
\end{frame}

% --------------------------------------
% the jacobi solver 
\begin{frame}[fragile]
% 
  \frametitle{Threaded \class{Jacobi}: workers, part 2}
%
  \begin{lstlisting}[language=c++,name=Jacobi:updated-solve,basicstyle=\tt\bfseries\tiny]
        // grab the grid update lock
        pthread_mutex_lock(&task->gridUpdate_lock);
        // update the global maximum deviation 
        if (task->maxDeviation < max_dev) {
            task->maxDeviation = max_dev;
        }
        // leave a mark
        task->contributions++;
        // bookkeeping at the end of the  update
        if (task->contributions == workers) {
            // if i am the slowest worker
            // swap the blocks between the two grids
            Grid::swapBlocks(current, next);
            // check covergence
            if (task->maxDeviation < task->tolerance) {
                std::cout
                    << " +++ thread " << id << ": convergence in " << iteration << " iterations"
                    <<std::endl;
                task->done = true;
            }
            // reset our accounting and signal everybody
            task->contributions = 0;
            task->maxDeviation = 0;
            pthread_cond_broadcast(&task->gridUpdate_check);
        } else {
            // all but the slowest wait here
            pthread_cond_wait(&task->gridUpdate_check, &task->gridUpdate_lock);
        } 
        // release
        pthread_mutex_unlock(&task->gridUpdate_lock);
        // check whether we are done
        if (task->done) {
            break;
        }
    }
    return 0;
}
  \end{lstlisting}
%
\end{frame}

% --------------------------------------
% improving the threaded implementation
\begin{frame}[fragile]
%
  \frametitle{Assessing the improved implementation}
%
  \begin{itemize}
%
  \item the improved threading scheme is not much more complex
    \begin{itemize}
    \item we keep track of how many threads have computed their grid update
    \item the slowest worker check the convergence criterion and performs all the necessary
      bookkeeping 
    \item while everybody else waits
    \item use \function{pthread\_cond\_brodacast} to wake the other workers
    \end{itemize}
%
  \item here is the performance comparison for 10,000 iterations on a $1000 \times 1000$ grid
    on the same 8-core MacPro
    \begin{equation*}
      \begin{array}{r|ccccc}
% header
        {\rm threads } & 1 & 2 & 4 & 8 & 16 \\
        \hline 
        {\rm previous} (s) & 413.306 & 211.050 & 109.509 & 98.279 & 74.087 \\
        {\rm updated} (s)  & 408.636 & 208.832 & 107.015 & 59.043 & 61.481 \\
% 
      \end{array}
    \end{equation*}
%
  \end{itemize}
%
\end{frame}

% --------------------------------------
% parallelization with MPI
\begin{frame}[fragile]
%
  \frametitle{Parallelization with \mpi}
%
  \begin{itemize}
%
  \item the \mpi\ implementation will require careful data management
    \begin{itemize}
    \item we must partition the mesh among processes
    \item each process work on its own subgrid
      \begin{itemize}
      \item it will allocate its own memory, for both actual data and the guard zones
      \item it must locate its patch in physical space
      \end{itemize}
    \item communication is required every iteration
      \begin{itemize}
      \item so that neighbors can synchronize their boundaries
      \item think of the synchronization as a kind of boundary condition!
      \end{itemize}
    \item parallel convergence testing involves a collective operation
    \end{itemize}
%
  \end{itemize}
%
  \begin{figure}
    \includegraphics[scale=0.5]{figures/structured-partitioning.pdf}
  \end{figure} 
% 
\end{frame}

% --------------------------------------
% template
\begin{frame}[fragile]
%
  \frametitle{A little bit of help}
%
  \begin{itemize}
%
  \item \mpi\ supports this common use case through a Cartesian {\em virtual topology}
    \begin{itemize}
    \item a special communicator with a map from a $d$-dimensional virtual process grid to the
      normal linear process ranks
    \item and local operations that enable you to discover the ranks of your virtual neighbors
    \item there is even a special form of send/receive so that you don't have to worry about
      contention and race conditions during the boundary synchronization
    \end{itemize}
%
  \item to create a Cartesian communicator
    \begin{C}
int MPI_Cart_create(MPI_Comm oldcomm,
        int ndims, int* layout, int* periods, int reorder, MPI_Comm* newcomm);
    \end{C}
%
  \item to find out the coordinates of a process in the virtual grid given its rank
    \begin{C}
int MPI_Cart_coords(MPI_Comm cartesian,
        int rank, int ndims, int* coords); 
    \end{C}
%
    \item you can also find out the ranks of your neighbors
    \begin{C}
int MPI_Cart_shift(MPI_Comm cartesian,
        int dimension, int shift, int* origin, int* neighbor); 
    \end{C}
%
  \end{itemize}
%
\end{frame}

% --------------------------------------
% the mpi driver
\begin{frame}[fragile]
%
  \frametitle{The \mpi\ driver, part 1}
%
  \begin{lstlisting}[language=c++,name=mpi:driver,firstnumber=26,basicstyle=\tt\bfseries\tiny]
int main(int argc, char* argv[]) {
    int status;
    // initialize mpi
    status = MPI_Init(&argc, &argv);
    if (status) {
        throw("error in MPI_Init");
    }
    // get my rank in the world communicator
    int worldRank, worldSize;
    MPI_Comm_rank(MPI_COMM_WORLD, &worldRank);
    MPI_Comm_size(MPI_COMM_WORLD, &worldSize);
    size_t processors = static_cast<size_t>(std::sqrt(worldSize));

    // default values for our user configurable settings
    size_t n = 9; // points per processor
    size_t threads = 1;
    double tolerance = 1.0e-3;

    // read the command line
    int command;
    while ((command = getopt(argc, argv, "n:e:t:")) != -1) {
        switch (command) {
        // get the convergence tolerance
        case 'e':
            tolerance = atof(optarg);
            break;
        // get the grid size
        case 'n':
            n = (size_t) atof(optarg);
            break;
        // get the number of threads
        case 't':
            threads = (size_t) atoi(optarg);
            break;
        }
    }
  \end{lstlisting}
% 
\end{frame}

% --------------------------------------
% the mpi driver
\begin{frame}[fragile]
%
  \frametitle{The \mpi\ driver, part 2}
%
  \begin{lstlisting}[language=c++,name=mpi:driver,basicstyle=\tt\bfseries\tiny]
    // print out the chosen options
    if (worldRank == 0) {
        for (int arg = 0; arg < argc; ++arg) {
            std::cout << argv[arg] << " ";
        }
        std::cout
            << std::endl
            << "    grid size: " << n << std::endl
            << "      workers: " << threads << std::endl
            << "    tolerance: " << tolerance << std::endl;
    }

    // instantiate a problem
    Example problem("cliche", 1.0, processors, n);

    // instantiate a solver
    Jacobi solver(tolerance, threads);
    // solve
    solver.solve(problem);
    // save the results
    Visualizer vis;
    vis.csv(problem);

    // initialize mpi
    status = MPI_Finalize();
    if (status) {
        throw("error in MPI_Finalize");
    }

    // all done
    return 0;
}
  \end{lstlisting}
% 
\end{frame}

% --------------------------------------
% the mpi driver
\begin{frame}[fragile]
%
  \frametitle{The \class{Jacobi} declaration}
%
  \begin{lstlisting}[language=c++,name=mpi:jacobi-decl,basicstyle=\tt\bfseries\tiny]
class acm114::laplace::Jacobi : public acm114::laplace::Solver {
    // interface
public:
    virtual void solve(Problem &);

    // meta methods
public:
    inline Jacobi(double tolerance, size_t workers);
    virtual ~Jacobi();

    // data members
private:
    double _tolerance;
    size_t _workers;

    // disable these
private:
    Jacobi(const Jacobi &);
    const Jacobi & operator= (const Jacobi &);
};
  \end{lstlisting}
% 
\end{frame}

% --------------------------------------
% the problem base class
\begin{frame}[fragile]
%
  \frametitle{The \class{Problem} declaration}
%
  \begin{lstlisting}[language=c++,name=mpi:problem-decl,basicstyle=\tt\bfseries\tiny]
class acm114::laplace::Problem {
    //typedefs
public:
    typedef std::string string_t;
    // interface
public:
    string_t name() const;
    inline MPI_Comm communicator() const;
    inline int rank() const;
    // access to my grid
    inline Grid & solution();
    inline const Grid & solution() const;
    // interface used by the solver
    virtual void initialize();
    virtual void applyBoundaryConditions() = 0;
    // meta methods
public:
    Problem(string_t name, double interval, int processors, size_t points);
    virtual ~Problem();
    // data members
protected:
    string_t _name;
    double _delta, _x0, _y0;
    int _rank, _size, _processors;
    int _place[2];
    MPI_Comm _cartesian;
    Grid _solution;
    // disable these
private:
    Problem(const Problem &);
    const Problem & operator= (const Problem &);
};
  \end{lstlisting}
% 
\end{frame}

% --------------------------------------
% problem definitions
\begin{frame}[fragile]
%
  \frametitle{The \class{Problem} constructor}
%
  \begin{lstlisting}[language=c++,name=mpi:driver,basicstyle=\tt\bfseries\tiny]
Problem::Problem(
    string_t name, double interval, int processors, size_t points) :
    _name(name),
    _delta(interval/((points-2)*processors+1)),
    _x0(0.0), _y0(0.0),
    _rank(0), _size(0), _processors(processors), _place(),
    _cartesian(),
    _solution(points) {

    // build the intended layout
    int layout[] = { processors, processors };
    // find my rank in the world communicator
    int worldRank;
    MPI_Comm_rank(MPI_COMM_WORLD, &worldRank);
    // build a Cartesian communicator
    int periods[] = { 0, 0 };
    MPI_Cart_create(MPI_COMM_WORLD, 2, &layout[0], periods, 1, &_cartesian);
    // check whether i can paritcipate
    if (_cartesian != MPI_COMM_NULL) {
        // get my rank in the cartesian communicator
        MPI_Comm_rank(_cartesian, &_rank);
        MPI_Comm_size(_cartesian, &_size);
        // get my logical position on the process grid
        MPI_Cart_coords(_cartesian, _rank, 2, &_place[0]);
        // now compute my offset in physical space
        _x0 = 0.0 + (points-2)*_place[0]*_delta;
        _y0 = 0.0 + (points-2)*_place[1]*_delta;
    } else {
        // i was left out because the total number of processors is not a square
        std::cout
            << "world rank " << worldRank << ": not a member of the cartesian communicator "
            << std::endl;
    }
}
  \end{lstlisting}
% 
\end{frame}


% --------------------------------------
% the Example declaration
\begin{frame}[fragile]
%
  \frametitle{The \class{Example} declaration}
%
  \begin{lstlisting}[language=c++,name=mpi:example-decl]
class acm114::laplace::Example : public acm114::laplace::Problem {
    // interface
public:
    virtual void applyBoundaryConditions();

    // meta methods
public:
    inline Example(
        string_t name, double interval, int processors, size_t points);
    virtual ~Example();

    // disable these
private:
    Example(const Example &);
    const Example & operator= (const Example &);
};
  \end{lstlisting}
% 
\end{frame}

% --------------------------------------
% the solver
\begin{frame}[fragile]
%
  \frametitle{The implementation of \method{Jacobi::solve}, part 1}
%
  \begin{lstlisting}[language=c++,name=mpi:solver,fistnumber=14,basicstyle=\tt\bfseries\tiny]
void Jacobi::solve(Problem & problem) {
    // initialize the problem
    problem.initialize();

    // get a reference to the solution grid
    Grid & current = problem.solution();
    // build temporary storage for the next iterant
    Grid next(current.size());

    // put an upper limit on the number of iterations
    size_t maxIterations = (size_t) 1e4;
    for (size_t iteration = 0; iteration < maxIterations; iteration++) {
        // print out a progress repot
        if ((problem.rank() == 0) && (iteration % 100 == 0)) {
            std::cout 
                << "jacobi: iteration " << iteration
                << std::endl;
        }
        // enforce  the boundary conditions
        problem.applyBoundaryConditions();
        // reset the local maximum change
        double localMax = 0.0;
        // update the interior of the grid
        for (size_t j=1; j < next.size()-1; j++) {
            for (size_t i=1; i < next.size()-1; i++) {
                // the cell update
                next(i,j) = .25*(current(i+1,j)+current(i-1,j)+current(i,j+1)+current(i,j-1));
                // compute the change from the current cell value
                double dev = std::abs(next(i,j) - current(i,j));
                // and update the local maximum
                if (dev > localMax) {
                    localMax = dev;
                }
            }
        } // done with the grid update
  \end{lstlisting}
% 
\end{frame}

% --------------------------------------
% the solver driver
\begin{frame}[fragile]
%
  \frametitle{The implementation of \method{Jacobi::solve}, part 2}
%
  \begin{lstlisting}[language=c++,name=mpi:solver,basicstyle=\tt\bfseries\tiny]
        // swap the blocks of the two grids, leaving the solution in current
        Grid::swapBlocks(current, next);
        // compute global maximum deviation
        double globalMax;
        MPI_Allreduce(&localMax, &globalMax, 1, MPI_DOUBLE, MPI_MAX, problem.communicator());
        // convergence check
        if (globalMax < _tolerance) {
            if (problem.rank() == 0) {
                std::cout 
                    << "jacobi: convergence in " << iteration << " iterations"
                    << std::endl;
            }
            break;
        }
        // otherwise
    }
    // when we get here, either we have converged or ran out of iterations
    // update the fringe of the current grid
    problem.applyBoundaryConditions();
    // all  done
    return;
}
  \end{lstlisting}
% 
\end{frame}

% --------------------------------------
% applying boundary conditions
\begin{frame}[fragile]
%
  \frametitle{Boundary conditions and data exchanges, part 1}
%
  \begin{lstlisting}[language=c++,name=mpi:example-impl]
void Example::applyBoundaryConditions() {
    // a reference to my grid
    Grid & g = _solution;
    // my rank;
    int rank = _rank;
    // the ranks of my four neighbors
    int top, right, bottom, left;
    // get them
    MPI_Cart_shift(_cartesian, 1, 1, &rank, &top);
    MPI_Cart_shift(_cartesian, 0, 1, &rank, &right);
    MPI_Cart_shift(_cartesian, 1, -1, &rank, &bottom);
    MPI_Cart_shift(_cartesian, 0, -1, &rank, &left);

    // allocate send and receive buffers
    double * sendbuf = new double[g.size()];
    double * recvbuf = new double[g.size()];
  \end{lstlisting}
% 
\end{frame}

% --------------------------------------
% applying boundary conditions
\begin{frame}[fragile]
%
  \frametitle{Boundary conditions and data exchanges, part 2}
%
  \begin{lstlisting}[language=c++,name=mpi:example-impl]
    // shift to the right
    // fill my sendbuf with my RIGHT DATA BORDER
    for (size_t cell=0; cell < g.size(); cell++) {
        sendbuf[cell] = g(g.size()-2, cell); 
    }
    // do the shift 
    MPI_Sendrecv(
                 sendbuf, g.size(), MPI_DOUBLE, right, 17,
                 recvbuf, g.size(), MPI_DOUBLE, left, 17,
                 _cartesian, MPI_STATUS_IGNORE
                 );
    if (left == MPI_PROC_NULL) {
        // if i am on the boundary, paint the dirichlet conditions
        for (size_t cell=0; cell < g.size(); cell++) {
            g(0, cell) = 0;
        }
    } else {
        // fill my LEFT FRINGE with the received data
        for (size_t cell=0; cell < g.size(); cell++) {
            g(0, cell) = recvbuf[cell];
        }
    }
  \end{lstlisting}
% 
\end{frame}

% --------------------------------------
% applying boundary conditions
\begin{frame}[fragile]
%
  \frametitle{Boundary conditions and data exchanges, part 4}
%
  \begin{lstlisting}[language=c++,name=mpi:example-impl]
    // shift to the left
    // fill my sendbuf with my LEFT DATA BORDER
    for (size_t cell=0; cell < g.size(); cell++) {
        sendbuf[cell] = g(1, cell);
    }
    // do the shift 
    MPI_Sendrecv(
                 sendbuf, g.size(), MPI_DOUBLE, left, 17,
                 recvbuf, g.size(), MPI_DOUBLE, right, 17,
                 _cartesian, MPI_STATUS_IGNORE
                 );
    if (right == MPI_PROC_NULL) {
        // if i am on the boundary, paint the dirichlet conditions
        for (size_t cell=0; cell < g.size(); cell++) {
            g(g.size()-1, cell) = 0;
        }
    } else {
        // fill my RIGHT FRINGE with the received data
        for (size_t cell=0; cell < g.size(); cell++) {
            g(g.size()-1, cell) = recvbuf[cell];
        }
    }
    
  \end{lstlisting}
% 
\end{frame}

% --------------------------------------
% applying boundary conditions
\begin{frame}[fragile]
%
  \frametitle{Boundary conditions and data exchanges, part 5}
%
  \begin{lstlisting}[language=c++,name=mpi:example-impl]
    // shift up
    // fill my sendbuf with my TOP DATA BORDER
    for (size_t cell=0; cell < g.size(); cell++) {
        sendbuf[cell] = g(cell, g.size()-2);
    }
    // do the shift
    MPI_Sendrecv(
                 sendbuf, g.size(), MPI_DOUBLE, top, 17,
                 recvbuf, g.size(), MPI_DOUBLE, bottom, 17,
                 _cartesian, MPI_STATUS_IGNORE
                 );
    if  (bottom == MPI_PROC_NULL) {
        // if i am on the boundary, paint the dirichlet conditions
        for (size_t cell=0; cell < g.size(); cell++) {
            g(cell, 0) = std::sin((_x0 + cell*_delta)*pi);
        }
    } else {
        // fill my BOTTOM FRINGE with the received data
        for (size_t cell=0; cell < g.size(); cell++) {
            g(cell, 0) = recvbuf[cell];
        }
    }
  \end{lstlisting}
% 
\end{frame}

% --------------------------------------
% applying boundary conditions
\begin{frame}[fragile]
%
  \frametitle{Boundary conditions and data exchanges, part 6}
%
  \begin{lstlisting}[language=c++,name=mpi:example-impl]
    // shift down
    // fill my sendbuf with my BOTTOM DATA BORDER
    for (size_t cell=0; cell < g.size(); cell++) {
        sendbuf[cell] = g(cell, 1);
    }
    // do the shift
    MPI_Sendrecv(
                 sendbuf, g.size(), MPI_DOUBLE, bottom, 17,
                 recvbuf, g.size(), MPI_DOUBLE, top, 17,
                 _cartesian, MPI_STATUS_IGNORE
                 );
    if  (top == MPI_PROC_NULL) {
        // if i am on the boundary, paint the dirichlet conditions
        for (size_t cell=0; cell < g.size(); cell++) {
            g(cell, g.size()-1) = 
                std::sin((_x0 + cell*_delta)*pi) * std::exp(-pi);
        }
    } else {
        // fill my TOP FRINGE with the received data
        for (size_t cell=0; cell < g.size(); cell++) {
            g(cell, g.size()-1) = recvbuf[cell];
        }
    }
    
    return;
}
  \end{lstlisting}
% 
\end{frame}

% end of file 
