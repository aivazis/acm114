% -*- LaTeX -*-
% -*- coding: utf-8 -*-
%
% michael a.g. aïvázis
% california institute of technology
% (c) 1998-2012 all rights reserved
%

\lecture{Wrapping up the sequential implementation}{20120224}

% --------------------------------------
% assessment
\begin{frame}[fragile]
%
  \frametitle{Shortcomings}
%
  \begin{itemize}
%
  \item numerics:
    \begin{itemize}
    \item it converges very slowly; other update {\em schemes} improve on this
    \item our approximation is very low order, so it takes very large grids to produce a few
      digits of accuracy
    \item the convergence criterion has some unwanted properties; it triggers
      \begin{itemize}
      \item prematurely: large swaths of constant values may never get updated
      \item it would trigger even if we were updating the wrong grid!
      \end{itemize}
    \end{itemize}
%
  \item design:
    \begin{itemize}
    \item separate the problem specification from its solution
    \item there are other objects lurking, waiting to be uncovered
    \item someone should make the graphic visualizer
    \item restarts anybody?
    \item how would you try out different convergence criteria? update schemes? memory layouts?
    \end{itemize}
%
    \item usability:
      \begin{itemize}
      \item supporting interchangeable parts requires damage to the top level driver
        \begin{itemize}
        \item to enable the user to make the selection
        \item to expose new command line arguments that configure the new parts
        \end{itemize}
      \end{itemize}
%
  \end{itemize}
%
\end{frame}

% --------------------------------------
% assessment
\begin{frame}[fragile]
%
  \frametitle{Assessing our fundamentals}
%
  \begin{itemize}
%
  \item \class{Grid} is a good starting point for abstracting structured grids
    \begin{itemize}
    \item assumes ownership of the memory associated with a structured grid
    \item encapsulates the indexing function
    \item extend it to
      \begin{itemize}
      \item support different memory layout strategies
      \item support non-square grids (?)
      \item support non-uniform grids (?)
      \item higher dimensions
      \item if you need any of these, consider using one of the many excellent class libraries
        written by experts
      \end{itemize}
    \end{itemize}
%
  \item \class{Visualizer}, under another name, can form the basis for a more general
    persistence library
    \begin{itemize}
    \item to support HDF5, NetCDF, bitmaps, voxels, etc.
    \end{itemize}
%
  \end{itemize}
%
\end{frame}

% --------------------------------------
% the Problem class
\begin{frame}[fragile]
% 
  \frametitle{The \class{Problem} class: the interface}
%
  \begin{lstlisting}[language=c++,name=Problem]
// the solution representation
class acm114::laplace::Problem {
    //typedefs
public:
    typedef std::string string_t;
    // interface
public:
    inline string_t name() const;
    inline const Grid & exact() const;
    inline const Grid & deviation() const;
    inline Grid & solution();
    inline const Grid & solution() const;
    inline Grid & error();
    inline const Grid & error() const;
    // abstract
    virtual void initialize() = 0;
    virtual void initialize(Grid &) const = 0;
    // meta methods
public:
    inline Problem(string_t name, double width, size_t points);
    virtual ~Problem();
    // data members
  \end{lstlisting}
%
\end{frame}

% --------------------------------------
% the Problem class
\begin{frame}[fragile]
% 
  \frametitle{The \class{Problem} class: the data}
%
  \begin{lstlisting}[language=c++,name=Problem]
protected:
    string_t _name;
    double _delta;
    Grid _solution;
    Grid _exact;
    Grid _error;
    Grid _deviation;
    // disable these
private:
    Problem(const Problem &);
    const Problem & operator= (const Problem &);
};
  \end{lstlisting}
%
\end{frame}

% --------------------------------------
% our specific example
\begin{frame}[fragile]
% 
  \frametitle{The \class{Example} class}
%
  \begin{lstlisting}[language=c++]
class acm114::laplace::Example : public acm114::laplace::Problem {
    // interface
public:
    virtual void initialize();
    virtual void initialize(Grid &) const;

    // meta methods
public:
    inline Example(string_t name, double width, size_t points);
    virtual ~Example();

    // disable these
private:
    Example(const Example &);
    const Example & operator= (const Example &);
};

  \end{lstlisting}
%
\end{frame}

% --------------------------------------
% the solver basis
\begin{frame}[fragile]
% 
  \frametitle{The \class{Solver} class}
%
  \begin{lstlisting}[language=c++]
class acm114::laplace::Solver {
    // interface
public:
    virtual void solve(Problem &) = 0;

    // meta methods
public:
    inline Solver();
    virtual ~Solver();

    // data members
private:

    // disable these
private:
    Solver(const Solver &);
    const Solver & operator= (const Solver &);
};

  \end{lstlisting}
%
\end{frame}

% --------------------------------------
% the jacobi solver 
\begin{frame}[fragile]
% 
  \frametitle{The \class{Jacobi} class}
%
  \begin{lstlisting}[language=c++]
class acm114::laplace::Jacobi : public acm114::laplace::Solver {
    // interface
public:
    virtual void solve(Problem &);

    // meta methods
public:
    inline Jacobi(double tolerance, size_t workers);
    virtual ~Jacobi();

    // implementation details
protected:
    virtual void _solve(Problem &);
    static void * _update(void *);

    // data members
private:
    double _tolerance;
    size_t _workers;

    // disable these
private:
    Jacobi(const Jacobi &);
    const Jacobi & operator= (const Jacobi &);
};

  \end{lstlisting}
%
\end{frame}

% --------------------------------------
% parallelization using threads
\begin{frame}[fragile]
%
  \frametitle{Parallelization using threads}
%
  \begin{itemize}
%
  \item the shared memory implementation requires
    \begin{itemize}
    \item a scheme so that threads can update cells without the need for locks
    \item while maximizing locality of data access
    \item even the computation of the convergence criterion can be parallelized
    \end{itemize}
%
  \item parallelization strategy
    \begin{itemize}
    \item we will focus on parallelizing the iterative grid update
      \begin{itemize}
      \item grid initialization, visualization, computing the exact answer and the error field
        do not depend on the {\em number of iterations}
      \end{itemize}
    \item the finest grain of work is clearly an individual cell update based on the value of
      its four nearest neighbors
    \item for this two dimensional example, we can build coarser grain tasks using
      \begin{itemize}
      \item horizontal or vertical strips
      \item non-overlapping blocks
      \item the strategy gets more complicated if you want to perform the update in place
      \end{itemize}
    \item the communication patterns are trivial for the double buffering layout; only the
      final update of the convergence criterion requires any locking
    \item each coarse grain task can be assigned to a thread
    \end{itemize}
%
  \end{itemize}
% 
\end{frame}

% --------------------------------------
% required changes to the sequential driver
\begin{frame}[fragile]
%
  \frametitle{Required changes to the sequential solution}
%
  \begin{itemize}
%
  \item what is needed
    \begin{itemize}
    \item an object to hold the problem information shared among the threads
    \item the per-thread administrative data structure that holds the thread id and the pointer
      to the shared information
      \begin{itemize}
      \item this is the argument to \function{pthread\_create}
      \end{itemize}
    \item a mutex to protect the update of the global convergence criterion
    \item a \function{pthread\_create} compatible worker routine
    \item a change at the top-level driver to enable the user to choose the number of threads
    \end{itemize}
%
  \item and a strategy for managing the thread life cycle
    \begin{itemize}
%
    \item synchronization is trivial if
      \begin{itemize}
      \item we spawn our threads to perform the updates of a single iteration
      \item harvest them
      \item check the convergence criterion 
      \item stop, or respawn them if another iteration is necessary
      \end{itemize}
%
    \item can the convergence test be done in parallel?
      \begin{itemize}
        \item so we don't have to pay the create/harvest overhead?
        \item if so, how do we guarantee correctness and consistency?
      \end{itemize}
%
    \end{itemize}
% 
  \end{itemize}
%
\end{frame}

% --------------------------------------
% the jacobi solver 
\begin{frame}[fragile]
% 
  \frametitle{Threaded \class{Jacobi}: thread data}
%
  \begin{lstlisting}[language=c++,name=Jacobi:threaded]
struct Task {
    // shared information 
    size_t workers;
    Grid & current;
    Grid & next;
    double maxDeviation;
    // mutex to control access to the convergence criterion
    pthread_mutex_t lock; 

    // constructor
    Task(size_t workers, Grid & current, Grid & next) :
        workers(workers), current(current), next(next), maxDeviation(0.0) {
        pthread_mutex_init(&lock, 0);
    }
    // destructor
    ~Task() {
        pthread_mutex_destroy(&lock);
    }
};

struct Context {
    // thread info
    size_t id;
    pthread_t descriptor;
    Task * task;
};

  \end{lstlisting}
%
\end{frame}

% --------------------------------------
% the jacobi solver 
\begin{frame}[fragile]
% 
  \frametitle{Threaded \class{Jacobi}: driving the update}
%
  \begin{lstlisting}[language=c++,name=Jacobi:threaded]
void Jacobi::solve(Problem & problem) {
    // initialize the problem
    problem.initialize();
    // do the actual solve
    _solve(problem);
    // compute and store the error
    std::cout << "  computing absolute error" << std::endl;
    //  compute the relative error
    Grid & error = problem.error();
    const Grid & exact = problem.exact();
    const Grid & solution = problem.solution();

    for (size_t j=0; j < exact.size(); j++) {
        for (size_t i=0; i < exact.size(); i++) {
            if (exact(i,j) == 0.0) {
                error(i,j) = std::abs(solution(i,j));
            } else {
                error(i,j) = std::abs(solution(i,j) - exact(i,j))/exact(i,j);
            }
        }
    }
    std::cout << " --- done." << std::endl;
    return;
}
  \end{lstlisting}
%
\end{frame}

% end of file 
