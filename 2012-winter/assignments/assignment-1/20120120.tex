% -*- LaTeX -*-
% -*- coding: utf-8 -*-
%
% michael a.g. aïvázis
% california institute of technology
% (c) 1998-2012 all rights reserved
%

\documentclass[12pt,answers]{exam}

% packages, setup, macros, etc.
% -*- LaTeX -*-
% -*- coding: utf-8 -*-
%
% ~~~~~~~~~~~~~~~~~~~~~~~~~~~~~~~~~~~~~~~~~~~~~~~~~~~~~~~~~~~~~~~~~~~~~~~~~~~~~~
%
%                             michael a.g. aïvázis
%                      california institute of technology
%                      (c) 1998-2010  all rights reserved
%
% ~~~~~~~~~~~~~~~~~~~~~~~~~~~~~~~~~~~~~~~~~~~~~~~~~~~~~~~~~~~~~~~~~~~~~~~~~~~~~~
%


% language
\usepackage[english]{babel}

% color
\usepackage{xcolor}

% fonts
\usepackage{amsfonts}
\usepackage{times}
%\usepackage[iso-8859-7]{inputenc}

% figures
\usepackage{graphicx}

% listings and their configurations
\usepackage[slide,algoruled,linesnumbered,noend]{algorithm2e}
\SetKwComment{tnm}{\#}{}
\SetKw{KwAnd}{and}
\SetKw{KwSend}{send}
\SetKw{KwRecv}{recv}
\SetKw{KwFrom}{from}

\usepackage{textcomp}
\usepackage{listings}
\definecolor{keywordcolor}{rgb}{.0,.0,1.0}
\definecolor{commentcolor}{gray}{.3}
\definecolor{stringcolor}{gray}{.0}
\definecolor{linenumbercolor}{gray}{.3}
\definecolor{listingbgcolor}{gray}{.97}

\definecolor{acm114@sand}{HTML}{dddbc5}
\definecolor{acm114@olive}{HTML}{474a41}
\definecolor{acm114@lava}{HTML}{413b38}

\lstnewenvironment{python}[2][]{
  \lstset{
    language=python,
    morekeywords={self,yield,False,True,None},
    %
    columns=flexible,
    upquote=true,
    %
    aboveskip=\bigskipamount,
    belowskip=\bigskipamount,
    %
    numbers=left,
    numberstyle=\color{linenumbercolor}\tiny,
    stepnumber=1,
    numbersep=5pt,
    numberblanklines=true,
    %
    basicstyle=\tt\scriptsize,
    keywordstyle=\color{keywordcolor},
    commentstyle=\color{commentcolor}\slshape,
    stringstyle=\color{stringcolor}\slshape,
    showstringspaces=false,
    %
    frame=tb,
    captionpos=t,
    backgroundcolor=\color{white},
    xleftmargin=.25in,
    xrightmargin=.1in,
    %
    escapeinside={\#@}{@},
    %
    #1
  }}{#2}

\lstnewenvironment{C}[1][]{
  \lstset{
    language=c,
    columns=flexible,
    upquote=true,
    %
    aboveskip=\bigskipamount,
    belowskip=\bigskipamount,
    %
    numbers=left,
    numberstyle=\color{linenumbercolor}\tiny,
    stepnumber=1,
    numbersep=5pt,
    numberblanklines=true,
    %
    basicstyle=\tt\scriptsize,
    keywordstyle=\color{blue},
    commentstyle=\color{commentcolor}\slshape,
    showstringspaces=false,
    %
    frame=tb,
    captionpos=t,
    backgroundcolor=\color{listingbgcolor},
    xleftmargin=.25in,
    xrightmargin=.1in,
    %
    escapeinside={//@}{@},
    %
    #1
  }}{}

% references
\usepackage[numbers]{natbib}
\bibliographystyle{unsrtnat}
\renewcommand\bibsection{\section{\refname}}
\def\newblock{\small}

% misc
\usepackage{dcolumn}
\newcolumntype{d}[1]{D{.}{.}{#1}}

\usepackage{url}
\usepackage{hyperref}


% shortcuts
\def\algref#1{{Alg.~\ref{alg:#1}}}
\def\alglineref#1{{line~\ref{line:#1}}}
\def\eqref#1{{Eq.~\ref{eq:#1}}}
\def\figref#1{{Fig.~\ref{fig:#1}}}
\def\secref#1{{Sec.~\ref{sec:#1}}}
\def\tabref#1{{Table~\ref{tab:#1}}}
\def\lstref#1{{Listing~\ref{lst:#1}}}
\def\lstlineref#1{{line~\ref{line:#1}}}

% macros
\def\bydef{\mathrel{\mathop:}=}
\def\CC{\mbox{\tt C}}
\def\GNU{\mbox{\tt GNU}}
\def\GSL{\mbox{\tt GSL}}
\def\RANLUX{\mbox{\tt RANLUX}}

\def\cpp{\mbox{\tt C++}}
%\def\cpp{\mbox{\tt C\raise.4ex\hbox{++}}}
\def\fortran{{\tt FORTRAN}}
\def\f90{{\tt FORTRAN90}}

\def\pyre{{\tt pyre}}

\def\order#1{\mbox{$\mathcal{O}(#1)$}}
\def\class#1{\mbox{\tt #1}}
\def\component#1{\mbox{\tt #1}}
\def\function#1{\mbox{\tt #1}}
\def\method#1{\mbox{\tt #1}}
\def\identifier#1{\mbox{\tt #1}}
\def\keyword#1{\mbox{\tt #1}}
\def\srcfile#1{\mbox{\tt #1}}

\def\insertionsort{\mbox{\sc Insertion-Sort}}
\def\mergesort{\mbox{\sc Merge-Sort}}
\def\merge{\mbox{\sc Merge}}

\def\TODO#1{{%
\subsubsection*{Still to do}%
\scriptsize\tt%
\begin{list}{\leftpointright}{} #1 \end{list}}}

% set up the PDF options
\hypersetup{
    pdftitle={ACM/CS 114: Winter 2010},
    pdfauthor={Michael A.G. A\"iv\'azis},
    pdfsubject={Lecture notes},
    pdfkeywords=,           % list of keywords
%
    bookmarks=true,         % show bookmarks bar?
    unicode=false,          % non-Latin characters in Acrobat's bookmarks
    pdftoolbar=true,        % show Acrobat's toolbar?
    pdfmenubar=true,        % show Acrobat's menu?
    pdffitwindow=true,      % page fit to window when opened
    pdfnewwindow=true,      % links in new window
    colorlinks=true,        % false: boxed links; true: colored links
    linkcolor=acm114@sand,  % color of internal links
    citecolor=acm114@sand,  % color of links to bibliography
    filecolor=acm114@sand,  % color of file links
    urlcolor=acm114@sand    % color of external links
}
% end of file 


\begin{document}
\pagestyle{headandfoot}
\runningfootrule
\firstpageheader{ACM/CS 114}{Assignment 1}{Due: 20 Jan 2012}
\runningheader{}{}{}
\firstpagefooter{}{}{}
\runningfooter{ACM/CS 114}{Assignment 1}{\thepage}

\begin{questions}

% --------------------------------------
% merge sort
\question
\mergesort\ is an example of a {\em divide-and conquer} algorithm. The algorithm
sorts an input sequence $S$ of numbers using the following steps:
%
\begin{itemize}
\item {\em divide}: split $S$ into two parts of roughly equal length,
\item {\em conquer}: sort the subsequences recursively,
\item {\em combine}: merge the two sorted subsequences to produce the sorted output.
\end{itemize}
%
In pseudocode:
%
\begin{center}
  \begin{minipage}{.5\linewidth}
    \begin{algorithm}[H]
      \label{alg:merge-sort}
%
      \DontPrintSemicolon
      %\NoCaptionOfAlgo
      \SetAlCapHSkip{0ex}
%
      \caption{\mergesort($S$, $p$, $r$)}
      \vspace{.5em}
%
      \If{$p < r$}{
        $q \leftarrow \lfloor (p+r)/2 \rfloor$ \;
        \mergesort($S$, $p$, $q$) \;
        \mergesort($S$, $q+1$, $r$) \;
        \merge($S$, $p$, $q$, $r$) \;
      }
%
      \vspace{.5em}
%
    \end{algorithm}
  \end{minipage}
\end{center}
%
\begin{parts}

\part Explain the role of $p$ and $r$ in the algorithm specification. What values should they
have upon initial invocation of the algorithm?

\part Write \merge. It was claimed in class that \merge\ can be implemented to run in
$\Theta(r-p+1)$ time. How does your implementation compare?

\part Implement \mergesort\ in a language of your choice. 
\begin{subparts}
  \subpart Write a driver that invokes it with $S = (5, 2, 4, 6, 1, 3)$.
  \subpart Build a container with $10^6$ random numbers. Sort it using your implementation.
\end{subparts}

\end{parts}

% --------------------------------------
% reduction 
\def\Li{\mbox{\rm Li}_{2}}
\def\len{\mbox{\rm length}}
\def\dilog{\mbox{\tt dilog}}
\def\sdlog{\mbox{\tt sdilog}}

\question Consider the function $\Li$ defined for $|z| \leq 1$ by
\[
\Li(z) \bydef
     \sum_{n=1}^{\infty} \frac{z^{n}}{n^{2}}
     =
     z + \frac{z^{2}}{2^{2}} + \frac{z^{3}}{3^{2}} + \frac{z^{4}}{4^{2}} + \cdots
\]

\begin{parts}

\part In the language of your choice, implement a procedure \dilog($z$, $n$) that computes the
sum of the first $n$ terms of the series for the given floating point number $z$.

\part Implement a procedure \sdlog($Z$, $n$) that accepts a sequence $Z$ of floating point
numbers and computes the sum
\[
\sdlog(Z, n) \bydef \sum_{z \in Z} \dilog(z, n)
\]

\part Build a cost model for \sdlog\ as a function of $n$ and $m \bydef \len(Z)$. Assume that
additions and subtractions cost $c_{+}$ each, multiplications and divisions cost $c_{\times}$
each, and that raising a number to the $n$th power costs $c_{\star}$.

\end{parts}
        
\end{questions}

\end{document}

% end of file 
