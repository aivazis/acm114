% -*- LaTeX -*-
% -*- coding: utf-8 -*-
%
% michael a.g. aïvázis
% california institute of technology
% (c) 1998-2012 all rights reserved
%

\documentclass[12pt,answers]{exam}

% packages, setup, macros, etc.
% -*- LaTeX -*-
% -*- coding: utf-8 -*-
%
% ~~~~~~~~~~~~~~~~~~~~~~~~~~~~~~~~~~~~~~~~~~~~~~~~~~~~~~~~~~~~~~~~~~~~~~~~~~~~~~
%
%                             michael a.g. aïvázis
%                      california institute of technology
%                      (c) 1998-2010  all rights reserved
%
% ~~~~~~~~~~~~~~~~~~~~~~~~~~~~~~~~~~~~~~~~~~~~~~~~~~~~~~~~~~~~~~~~~~~~~~~~~~~~~~
%


% language
\usepackage[english]{babel}

% color
\usepackage{xcolor}

% fonts
\usepackage{amsfonts}
\usepackage{times}
%\usepackage[iso-8859-7]{inputenc}

% figures
\usepackage{graphicx}

% listings and their configurations
\usepackage[slide,algoruled,linesnumbered,noend]{algorithm2e}
\SetKwComment{tnm}{\#}{}
\SetKw{KwAnd}{and}
\SetKw{KwSend}{send}
\SetKw{KwRecv}{recv}
\SetKw{KwFrom}{from}

\usepackage{textcomp}
\usepackage{listings}
\definecolor{keywordcolor}{rgb}{.0,.0,1.0}
\definecolor{commentcolor}{gray}{.3}
\definecolor{stringcolor}{gray}{.0}
\definecolor{linenumbercolor}{gray}{.3}
\definecolor{listingbgcolor}{gray}{.97}

\definecolor{acm114@sand}{HTML}{dddbc5}
\definecolor{acm114@olive}{HTML}{474a41}
\definecolor{acm114@lava}{HTML}{413b38}

\lstnewenvironment{python}[2][]{
  \lstset{
    language=python,
    morekeywords={self,yield,False,True,None},
    %
    columns=flexible,
    upquote=true,
    %
    aboveskip=\bigskipamount,
    belowskip=\bigskipamount,
    %
    numbers=left,
    numberstyle=\color{linenumbercolor}\tiny,
    stepnumber=1,
    numbersep=5pt,
    numberblanklines=true,
    %
    basicstyle=\tt\scriptsize,
    keywordstyle=\color{keywordcolor},
    commentstyle=\color{commentcolor}\slshape,
    stringstyle=\color{stringcolor}\slshape,
    showstringspaces=false,
    %
    frame=tb,
    captionpos=t,
    backgroundcolor=\color{white},
    xleftmargin=.25in,
    xrightmargin=.1in,
    %
    escapeinside={\#@}{@},
    %
    #1
  }}{#2}

\lstnewenvironment{C}[1][]{
  \lstset{
    language=c,
    columns=flexible,
    upquote=true,
    %
    aboveskip=\bigskipamount,
    belowskip=\bigskipamount,
    %
    numbers=left,
    numberstyle=\color{linenumbercolor}\tiny,
    stepnumber=1,
    numbersep=5pt,
    numberblanklines=true,
    %
    basicstyle=\tt\scriptsize,
    keywordstyle=\color{blue},
    commentstyle=\color{commentcolor}\slshape,
    showstringspaces=false,
    %
    frame=tb,
    captionpos=t,
    backgroundcolor=\color{listingbgcolor},
    xleftmargin=.25in,
    xrightmargin=.1in,
    %
    escapeinside={//@}{@},
    %
    #1
  }}{}

% references
\usepackage[numbers]{natbib}
\bibliographystyle{unsrtnat}
\renewcommand\bibsection{\section{\refname}}
\def\newblock{\small}

% misc
\usepackage{dcolumn}
\newcolumntype{d}[1]{D{.}{.}{#1}}

\usepackage{url}
\usepackage{hyperref}


% shortcuts
\def\algref#1{{Alg.~\ref{alg:#1}}}
\def\alglineref#1{{line~\ref{line:#1}}}
\def\eqref#1{{Eq.~\ref{eq:#1}}}
\def\figref#1{{Fig.~\ref{fig:#1}}}
\def\secref#1{{Sec.~\ref{sec:#1}}}
\def\tabref#1{{Table~\ref{tab:#1}}}
\def\lstref#1{{Listing~\ref{lst:#1}}}
\def\lstlineref#1{{line~\ref{line:#1}}}

% macros
\def\bydef{\mathrel{\mathop:}=}
\def\CC{\mbox{\tt C}}
\def\GNU{\mbox{\tt GNU}}
\def\GSL{\mbox{\tt GSL}}
\def\RANLUX{\mbox{\tt RANLUX}}

\def\cpp{\mbox{\tt C++}}
%\def\cpp{\mbox{\tt C\raise.4ex\hbox{++}}}
\def\fortran{{\tt FORTRAN}}
\def\f90{{\tt FORTRAN90}}

\def\pyre{{\tt pyre}}

\def\order#1{\mbox{$\mathcal{O}(#1)$}}
\def\class#1{\mbox{\tt #1}}
\def\component#1{\mbox{\tt #1}}
\def\function#1{\mbox{\tt #1}}
\def\method#1{\mbox{\tt #1}}
\def\identifier#1{\mbox{\tt #1}}
\def\keyword#1{\mbox{\tt #1}}
\def\srcfile#1{\mbox{\tt #1}}

\def\insertionsort{\mbox{\sc Insertion-Sort}}
\def\mergesort{\mbox{\sc Merge-Sort}}
\def\merge{\mbox{\sc Merge}}

\def\TODO#1{{%
\subsubsection*{Still to do}%
\scriptsize\tt%
\begin{list}{\leftpointright}{} #1 \end{list}}}

% set up the PDF options
\hypersetup{
    pdftitle={ACM/CS 114: Winter 2010},
    pdfauthor={Michael A.G. A\"iv\'azis},
    pdfsubject={Lecture notes},
    pdfkeywords=,           % list of keywords
%
    bookmarks=true,         % show bookmarks bar?
    unicode=false,          % non-Latin characters in Acrobat's bookmarks
    pdftoolbar=true,        % show Acrobat's toolbar?
    pdfmenubar=true,        % show Acrobat's menu?
    pdffitwindow=true,      % page fit to window when opened
    pdfnewwindow=true,      % links in new window
    colorlinks=true,        % false: boxed links; true: colored links
    linkcolor=acm114@sand,  % color of internal links
    citecolor=acm114@sand,  % color of links to bibliography
    filecolor=acm114@sand,  % color of file links
    urlcolor=acm114@sand    % color of external links
}
% end of file 


\begin{document}
\pagestyle{headandfoot}
\runningfootrule
\firstpageheader{ACM/CS 114}{Assignment 2}{Due: 1 February 2012}
\runningheader{}{}{}
\firstpagefooter{}{}{}
\runningfooter{ACM/CS 114}{Assignment 2}{\thepage}



% --------------------------------------
% reduction using threads
\def\Li{\mbox{\rm Li}_{2}}
\def\len{\mbox{\rm length}}
\def\dilog{\mbox{\tt dilog}}

A better definition for the dilogarithm $\Li$ is given by
%
\begin{equation}
\Li(z) \bydef
- \int_{0}^{z} dz' \; \frac{\log(1-z')}{z'} \label{eq:li-def}
\end{equation}
%
This is well defined for arbitrary complex values $z$, with a branch cut along the real axis
for $z > 1$. It can be shown that
%
\begin{eqnarray}
  \Li(1)  & = & \frac{\pi^{2}}{6} \label{eq:li1} \\
  \Li(-1) & = & - \frac{\pi^{2}}{12} \label{eq:li-1}
\end{eqnarray}
%
We will try to establish that \eqref{li1} and \eqref{li-1} are true by numerically evaluating
the integral in \eqref{li-def} both sequentially and {\em in parallel} using threads and \mpi.
%

\begin{questions}

\question Let's implement a function \function{dilog} that accepts a real number $z$ and a
potentially large integer $N$, and returns an approximation to \eqref{li-def}. For example, a
\cc\ or \cpp\ prototype would look like
%
\begin{C}
    double dilog(double z, long N);
\end{C}
%
\begin{parts}
%
  \part First, take care of the obvious cases $N < 1$, $z = 0$ and, for the sake of simplicity,
  $z > 1$ where the dilogarithm picks up an imaginary part. You may throw exceptions, exit
  after printing an error message, or use any other means of signaling the user that the input
  is invalid.
%
  \part Implement the numerical approximation to the integral in \eqref{li-def} by splitting
  the interval $(0,z)$ in $N$ subintervals of equal length $dz$. Each subinterval contributes
  an amount equal to its width multiplied by the value of the integrand at the center of the
  subinterval. Return the sum of these contributions as the approximation to the dilogarithm. 
%
  \part Why do we only evaluate the function at the center of each subinterval?
%
  \part Write a driver for \function{dilog} and convince yourself that it gives reasonable
  approximations to the values in \eqref{li1} and \eqref{li-1}. Fill out the following table: 
  \begin{table}[h]
    \small
    \centering
    \[
    \begin{array}{l|d{8}d{8}|d{8}d{8}}
      % header
      \multicolumn{1}{c|}{N} &
      \multicolumn{1}{c}{\Li(-1)} &
      \multicolumn{1}{c|}{\Delta} &
      \multicolumn{1}{c}{\Li(1)} &
      \multicolumn{1}{c}{\Delta}
      \\ [.75ex]
      % body
      \hline 
%
      10^{1} & & & & \\ [.75ex]
%
      10^{3} & & & & \\ [.75ex]
%
      10^{6} & & & & \\ [.75ex]
%
      10^{9} & & & & \\ [.75ex]
%
      10^{12} & & & & 
%
    \end{array}
    \]
  \end{table}
%

where $\Delta$ is the relative error in each case.
%
\end{parts}
%

\question Explain how you would compute an approximation of the integral in \eqref{li-def} {\em
  in parallel}. In particular:
\begin{parts}
  \part What are the finest grain parallel tasks?
  \part What are the communication requirements among these tasks?
  \part What is the correct coarsening strategy?
  \part How you would map the coarse tasks onto physical processing units?
\end{parts}

\question Implement \function{dilog\_threads} that uses pthreads for enhanced performance.
\begin{parts}
  \part What is the correct partitioning, coarsening and mapping strategies when using threads?
  \part Implement the routine that becomes the body of each thread.
  \part Implement \function{dilog\_threads} to perform the error checking, spawn $p$ threads to
  perform the calculation, collect the results of each thread and return the approximation to
  the integral.
  \part Convince yourself that \function{dilog\_threads} returns correct values regardless of
  the number of threads it spawned.
\end{parts}

\question Implement a parallel version \function{dilog\_mpi} using \mpi.
\begin{parts}
  \part What is the correct partitioning, coarsening and mapping strategies when using
  distributed memory parallelism?
  \part Implement \function{dilog\_mpi}, making sure that every processor holds the final
  answer for the value of the integral. 
  \part Convince yourself that \function{dilog\_mpi} returns correct values regardless of the
  number of processors used.
\end{parts}

\end{questions}

\end{document}

% end of file 
