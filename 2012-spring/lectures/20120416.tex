% -*- LaTeX -*-
% -*- coding: utf-8 -*-
%
% michael a.g. aïvázis
% california institute of technology
% (c) 1998-2012 all rights reserved
%

\lecture{Introduction to python}{20120416}

% --------------------------------------
% namespaces
\begin{frame}[fragile]
%
  \frametitle{Namespaces}
%
  \begin{itemize}
%
  \item names are placed in \emph{namespaces} in order to avoid collisions
    \begin{itemize}
    \item no specific type or construct: anything that supports the \keyword{.} operator
    \item examples: classes, modules, packages
    \end{itemize}
%
  \item modules are objects created when requesting access to the names from a different file
    \begin{itemize}
    \item python sources, which are byte-compiled on first import
    \item shared libraries, which are dynamically loaded on first request
    \item folders on the filesystem that contain the marker \literal{\_\_init\_\_.py}
    \item statically linked when the interpreter was compiled
    \end{itemize}
%
  \item the interpreter has a \emph{search path} for modules, which is controlled
    \begin{itemize}
    \item at interpreter compile time
    \item by the current working directory of the process
    \item by reading user settings at interpreter start up
      \begin{itemize}
      \item the \literal{PYTHONPATH} environment variable on unix, the registry on windows
      \end{itemize}
    \end{itemize}
%
  \end{itemize}
%
\end{frame}

% --------------------------------------
% namespace access
\begin{frame}[fragile]
%
  \frametitle{Namespace access}
%
  \begin{itemize}
%
  \item names within a namespace are accessed with the \keyword{import} implicit assignment
    statement
    \begin{ipython}{}
      import <namespace>
      from <namespace> import <name>
      from <namespace> import *
      from <namespace> import <name> as <alias>
    \end{ipython}
%
  \item namespaces may be nested
    \begin{ipython}{}
      from sys.path import abspath
    \end{ipython}
    so \emph{name qualifications} allow fine tuning of the list of imported symbols
%
  \item folders, and their sub-folders and files, become a hierarchy of nested namespaces
    automatically
    \begin{itemize}
    \item files with the \literal{.py} extension
    \item folders with the \literal{\_\_init\_\_.py} special file
    \end{itemize}
%
  \end{itemize}
%
\end{frame}

% --------------------------------------
% namespaces as objects
\begin{frame}[fragile]
%
  \frametitle{Namespaces as objects}
%
  \begin{itemize}
%
  \item modules and packages are objects:
    \begin{ipython}{}
      def load(material):
          'load the named {material} model'
          # build the comman string
          cmd = 'from materials import {} as model'.format(material)
          # get the interpreter to do its thing
          exec(cmd)
          # if all goes well, return the loaded module
          return model

      # load the material model
      model = load(material='perfectGas')
      # ask for an equation of state
      eos = model.newMaterial()
    \end{ipython}
    dynamic programming!
%
  \item this example is brittle; one can do much better...
%
  \end{itemize}
%
\end{frame}

% --------------------------------------
% classes
\begin{frame}[fragile]
%
  \frametitle{Classes}
%
  \begin{itemize}
%
  \item classes are object factories
    \begin{itemize}
    \item they introduce new types with state and behavior
    \item using the name of a class in a call expression invokes the constructor
    \item each instance has access to all the class attributes
    \item assignments in the class declaration create class attributes
    \item assignments to \keyword{self} create per-instance attributes
    \end{itemize}
%
    \begin{ipython}{}
      class Shape:
          'the basis of all shapes'

          # public data
          name = 'generic shape'

          # interface
          def kind(self): return self.name

          # meta methods
          def __init__(self, **kwds):
              super().__init_(**kwds)
              self.rep = None
              return
    \end{ipython}
%
  \end{itemize}
%
\end{frame}

% --------------------------------------
% using classes
\begin{frame}[fragile]
%
  \frametitle{Class records and class instances}
%
  \begin{itemize}
%
  \item the class declaration is an implicit assignment to a \emph{class record}
  \item class records are a built-in type
    \begin{ipython}{}
      print(Shape)
      print(Shape.name)
      print(Shape.kind)
    \end{ipython}
%
  \item to make an \emph{instance}, use the name of the class in a call expression
    \begin{ipython}{}
      shape = Shape()
      print(shape.name)
      print(shape.kind())
    \end{ipython}
%
  \item of course, neither \class{Shape} nor its instances are very interesting
%
  \end{itemize}
%
\end{frame}

% --------------------------------------
% methods
\begin{frame}[fragile]
%
  \frametitle{Methods}
%
  \begin{itemize}
%
  \item the class declaration creates a class record and assigns it to whatever name you used
    for the class
%
  \item invoking the class name as a function builds new instances of that class
%
  \item \emph{methods} are functions defined within the class declaration; they provide
    behavior for the instances
    \begin{itemize}
    \item they must take at least one parameter to receive the instance through which they were
      invoked
    \item this special parameter is named \identifier{self}, by \emph{convention}
    \end{itemize}
%
  \end{itemize}
%
\end{frame}

% --------------------------------------
% inheritance
\begin{frame}[fragile]
%
  \frametitle{Inheritance}
%
  \begin{itemize}
%
  \item specialization through inheritance 
    \begin{itemize}
    \item super-classes must be listed in the class declaration
    \item derived classes inherit all the attributes of their ancestors
    \item instances inherit attributes from all accessible classes
    \end{itemize}
%
    \begin{ipython}{}
      class Disk(Shape):
          'the shape bounded by a circle'

          # public data
          name = 'disk'
          radius = 1
          center = (0,0)

          # meta methods
          def __init__(self, radius=radius, center, **kwds):
              super().__init_(**kwds)
              self.radius = radius
              self.center = center
              return
    \end{ipython}
%
  \item all classes inherit from \identifier{object}
%
  \end{itemize}
%
\end{frame}

% --------------------------------------
% class glossary
\begin{frame}[fragile]
%
  \frametitle{Class glossary}
%
  \begin{itemize}
%
  \item \emph{class}
    \begin{itemize}
    \item a blueprint for the construction of new types of objects
    \end{itemize}
%
  \item \emph{instance}
    \begin{itemize}
    \item an object created using a class constructor
    \end{itemize}
%
  \item \emph{member}
    \begin{itemize}
    \item an attribute of a class instance
    \end{itemize}
%
  \item \emph{method}
    \begin{itemize}
    \item an attribute of a class instance that is bound to a function object
    \end{itemize}
%
  \item \emph{self}
    \begin{itemize}
    \item the conventional name give to the method parameter that receives the referenced
      instance
    \end{itemize}
%
  \end{itemize}
%
\end{frame}

% --------------------------------------
% special methods
\begin{frame}[fragile]
%
  \frametitle{Overloading operators in classes}
%
  \begin{itemize}
%
  \item \emph{Don't!}
  \item most python operations involving instances can be intercepted and customized
  \item through methods that have special names
%
  \end{itemize}
%
  \vspace{.5em}
  \begin{minipage}{.40\linewidth}
    \begin{table}\footnotesize
      \begin{tabular}{ll}
        \emph{method} & \emph{purpose} \\
        \method{\_\_init\_\_} & construction: \literal{x = X()} \\
        \method{\_\_del\_\_} & destruction \\
        \method{\_\_str\_\_} & string coercion: \literal{str(x)} \\
        \method{\_\_repr\_\_} & representation: \literal{repr(x)} \\
        \method{\_\_len\_\_} & size, truth tests: \literal{len(x)} \\
        \method{\_\_cmp\_\_} & comparisons: \literal{cmp(x)}, \literal{x < other} \\
        \method{\_\_call\_\_} & function class: \literal{x()}
      \end{tabular}
    \end{table}
  \end{minipage}
%
  \hspace{.1\linewidth}
%
  \begin{minipage}{.40\linewidth}
    \begin{table}\footnotesize
      \begin{tabular}{ll}
        \emph{method} & \emph{purpose} \\
        \method{\_\_getattr\_\_} & member access: \literal{x.name} \\
        \method{\_\_getitem\_\_} & indexing: \literal{x[5]} \\
        \method{\_\_setitem\_\_} & indexing: \literal{x[5] = 0} \\
        \method{\_\_add\_\_} & addition: \literal{x + other} \\
        \method{\_\_radd\_\_} & addition: \literal{other + x} \\
        \method{\_\_and\_\_} & logic: \literal{x and other} \\
        \method{\_\_or\_\_} & logic: \literal{x or other}
      \end{tabular}
    \end{table}
  \end{minipage}
%
\end{frame}

% --------------------------------------
% namespace rules
\begin{frame}[fragile]
%
  \frametitle{Namespace rules}
%
  \begin{itemize}
%
  \item a more complete story
    \begin{itemize}
    \item unqualified names are looked up in a chain of lexical namespaces
    \item qualified names conduct a search in the indicated namespace
    \item scopes initialize object namespaces: packages/modules, classes, instances
    \end{itemize}
%
  \item unqualified names
    \begin{itemize}
    \item are global on read
    \item are local on write, unless explicitly marked \keyword{global}
    \end{itemize}
%
  \item qualified names, e.g. \literal{instance.name}, are looked up in the indicated namespace
    \begin{itemize}
    \item module and package
    \item instance, then class record, then ancestors as specified in the \identifier{\_\_mro\_\_}
    \end{itemize}
%
  \item namespace dictionaries
    \begin{itemize}
    \item \identifier{\_\_dict\_\_}
    \item name qualification is a dictionary lookup
    \end{itemize}
%
  \end{itemize}
%
\end{frame}

% end of file 
