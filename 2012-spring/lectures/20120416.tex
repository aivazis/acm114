% -*- LaTeX -*-
% -*- coding: utf-8 -*-
%
% michael a.g. aïvázis
% california institute of technology
% (c) 1998-2012 all rights reserved
%

\lecture{Introduction to python}{20120416}

% --------------------------------------
% namespaces
\begin{frame}[fragile]
%
  \frametitle{Namespaces}
%
  \begin{itemize}
%
  \item names are placed in \emph{namespaces} in order to avoid collisions
    \begin{itemize}
    \item no specific type or construct: anything that supports the \keyword{.} operator
    \item examples: classes, modules, packages
    \end{itemize}
%
  \item modules are objects created when requesting access to the names from a different file
    \begin{itemize}
    \item python sources, which are byte-compiled on first import
    \item shared libraries, which are dynamically loaded on first request
    \item folders on the filesystem that contain the marker \literal{\_\_init\_\_.py}
    \item statically linked when the interpreter was compiled
    \end{itemize}
%
  \item the interpreter has a \emph{search path} for modules, which is controlled
    \begin{itemize}
    \item at interpreter compile time
    \item by the current working directory of the process
    \item by reading user settings at interpreter start up
      \begin{itemize}
      \item the \literal{PYTHONPATH} environment variable on unix, the registry on windows
      \end{itemize}
    \end{itemize}
%
  \end{itemize}
%
\end{frame}

% --------------------------------------
% namespace access
\begin{frame}[fragile]
%
  \frametitle{Namespace access}
%
  \begin{itemize}
%
  \item names within a namespace are accessed with the \keyword{import} implicit assignment
    statement
    \begin{ipython}{}
      import <namespace>
      from <namespace> import <name>
      from <namespace> import *
      from <namespace> import <name> as <alias>
    \end{ipython}
%
  \item namespaces may be nested
    \begin{ipython}{}
      from sys.path import abspath
    \end{ipython}
    so \emph{name qualifications} allow fine tuning of the list of imported symbols
%
  \item folders, and their sub-folders and files, become a hierarchy of nested namespaces
    automatically
    \begin{itemize}
    \item files with the \literal{.py} extension
    \item folders with the \literal{\_\_init\_\_.py} special file
    \end{itemize}
%
  \end{itemize}
%
\end{frame}

% --------------------------------------
% namespaces as objects
\begin{frame}[fragile]
%
  \frametitle{Namespaces as objects}
%
  \begin{itemize}
%
  \item modules and packages are objects:
    \begin{ipython}{}
      def load(material):
          'load the named {material} model'
          # build the comman string
          cmd = 'from materials import {} as model'.format(material)
          # get the interpreter to do its thing
          exec(cmd)
          # if all goes well, return the loaded module
          return model

      # load the material model
      model = load(material='perfectGas')
      # ask for an equation of state
      eos = model.newMaterial()
    \end{ipython}
    dynamic programming!
%
  \item this example is brittle; one can do much better...
%
  \end{itemize}
%
\end{frame}

% --------------------------------------
% classes
\begin{frame}[fragile]
%
  \frametitle{Classes}
%
  \begin{itemize}
%
  \item classes are object factories
    \begin{itemize}
    \item they introduce new types with state and behavior
    \item using the name of a class in a call expression calls the constructor
    \item each instance has access to all the class attributes
    \item assignments in class declaration create class attributes
    \item assignments to \keyword{self} create per-instance attributes
    \end{itemize}
%
    \begin{ipython}{}
      class Shape:
          'the basis of all shapes'

          # public data
          name = 'generic shape'

          # interface
          def kind(self): return self.name

          # meta methods
          def __init__(self):
              self.rep = None
              return
    \end{ipython}
%
  \end{itemize}
%
\end{frame}

% --------------------------------------
% inheritance
\begin{frame}[fragile]
%
  \frametitle{Inheritance}
%
  \begin{itemize}
%
  \item 
    \begin{itemize}
    \item
    \end{itemize}
%
  \end{itemize}
%
\end{frame}

% --------------------------------------
% methods
\begin{frame}[fragile]
%
  \frametitle{Methods}
%
  \begin{itemize}
%
  \item 
    \begin{itemize}
    \item
    \end{itemize}
%
  \end{itemize}
%
\end{frame}

% --------------------------------------
% class glossary
\begin{frame}[fragile]
%
  \frametitle{Class glossary}
%
  \begin{itemize}
%
  \item 
    \begin{itemize}
    \item
    \end{itemize}
%
  \end{itemize}
%
\end{frame}

% --------------------------------------
% special methods
\begin{frame}[fragile]
%
  \frametitle{Special methods}
%
  \begin{itemize}
%
  \item 
    \begin{itemize}
    \item
    \end{itemize}
%
  \end{itemize}
%
\end{frame}

% --------------------------------------
% namespace rules
\begin{frame}[fragile]
%
  \frametitle{Namespace rules}
%
  \begin{itemize}
%
  \item 
    \begin{itemize}
    \item
    \end{itemize}
%
  \end{itemize}
%
\end{frame}

% --------------------------------------
% classes as objects
\begin{frame}[fragile]
%
  \frametitle{Classes as objects}
%
  \begin{itemize}
%
  \item 
    \begin{itemize}
    \item
    \end{itemize}
%
  \end{itemize}
%
\end{frame}

% --------------------------------------
% methods as objects
\begin{frame}[fragile]
%
  \frametitle{Methods as objects}
%
  \begin{itemize}
%
  \item 
    \begin{itemize}
    \item
    \end{itemize}
%
  \end{itemize}
%
\end{frame}

% --------------------------------------
% exceptions
\begin{frame}[fragile]
%
  \frametitle{Exceptions}
%
  \begin{itemize}
%
  \item 
    \begin{itemize}
    \item
    \end{itemize}
%
  \end{itemize}
%
\end{frame}

% --------------------------------------
% raising exceptions
\begin{frame}[fragile]
%
  \frametitle{Raising exceptions}
%
  \begin{itemize}
%
  \item 
    \begin{itemize}
    \item
    \end{itemize}
%
  \end{itemize}
%
\end{frame}

% --------------------------------------
% catching exceptions
\begin{frame}[fragile]
%
  \frametitle{Catching exceptions}
%
  \begin{itemize}
%
  \item 
    \begin{itemize}
    \item
    \end{itemize}
%
  \end{itemize}
%
\end{frame}

% --------------------------------------
% object oriented programming
\begin{frame}[fragile]
%
  \frametitle{Object oriented programming}
%
  \begin{itemize}
%
  \item 
    \begin{itemize}
    \item
    \end{itemize}
%
  \end{itemize}
%
\end{frame}

% --------------------------------------
% other built-in objects
\begin{frame}[fragile]
%
  \frametitle{Other built-in objects}
%
  \begin{itemize}
%
  \item 
    \begin{itemize}
    \item
    \end{itemize}
%
  \end{itemize}
%
\end{frame}

% --------------------------------------
% type hierarchy
\begin{frame}[fragile]
%
  \frametitle{Type hierarchy}
%
  \begin{itemize}
%
  \item 
    \begin{itemize}
    \item
    \end{itemize}
%
  \end{itemize}
%
\end{frame}

% --------------------------------------
% printing
\begin{frame}[fragile]
%
  \frametitle{Printing}
%
  \begin{itemize}
%
  \item 
    \begin{itemize}
    \item
    \end{itemize}
%
  \end{itemize}
%
\end{frame}

% end of file 
