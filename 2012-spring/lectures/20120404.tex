% -*- LaTeX -*-
% -*- coding: utf-8 -*-
%
% michael a.g. aïvázis
% california institute of technology
% (c) 1998-2012 all rights reserved
%

\lecture{Introduction to python}{20120404}

% --------------------------------------
% programming paradigms
\begin{frame}[fragile]
%
  \frametitle{Languages and programming paradigms}
%
  \begin{itemize}
%
  \item a very active area of research
    \begin{itemize}
    \item dozens of languages and runtime environments of the last 50 years
    \end{itemize}
%
  \item the survivors:
    \begin{itemize}
      \item procedural programming, and its offspring structured programming
      \item functional programming
      \item object oriented programming
    \end{itemize}
%
  \item current areas of research:
    \begin{itemize}
    \item component oriented programming
    \item aspect programming
    \end{itemize}
%
  \item languages are important:
    \begin{itemize}
    \item they reflect an approach to computing
    \item they shape what is easily expressible
    \end{itemize}
%
  \item we'll take a quick tour of python
    \begin{itemize}
    \item resources: \url{www.python.org}
    \item overview of the language
    \item interactive sessions with the interpreter
    \item building extensions in \cc/\cpp
    \end{itemize}
%
  \end{itemize}
%
\end{frame}

% --------------------------------------
% preview: a python script
\begin{frame}[fragile]
%
  \frametitle{A python script}
%
  \begin{itemize}
  \item python reads like pseudocode
  \item here is the code for the $\pi$ estimator using Monte Carlo integration over the quarter
    disk
  \end{itemize}
%
\python{
  firstline=9,lastline=25,
  label={lst:python:pi},
  caption={\srcfile{pi.py}: Estimating $\pi$ in python},
}{listings/pi.py}

%
\end{frame}

% --------------------------------------
% overview
\begin{frame}[fragile]
%
  \frametitle{Overview}
%
  \begin{itemize}
%
  \item built-in objects and their operators
    \begin{itemize}
    \item numbers, strings, containers
    \item files
    \end{itemize}
%
  \item statements
    \begin{itemize}
      \item evaluating expressions, explicit and implicit assignments, logic, iteration
    \end{itemize}
%
  \item functions
    \begin{itemize}
    \item scope rules, argument passing, callable objects
    \end{itemize}
%
  \item modules and packages
    \begin{itemize}
    \item name qualification, importing symbols
    \end{itemize}
%
  \item user defined objects
    \begin{itemize}
    \item declarations and definitions, inheritance, overloading operators
    \end{itemize}
%
  \item exceptions
    \begin{itemize}
    \item raising and catching, exception hierarchies
    \end{itemize}
%
  \end{itemize}
%
\end{frame}

% --------------------------------------
% built-in objects
\begin{frame}[fragile]
%
  \frametitle{Built-in objects}
%
  \begin{itemize}
  \item the more commonly used types
  \end{itemize}
%
  \begin{table}
    \begin{tabular}{l|l}
      Type           & Sample \\[.25em] \hline \\
      booleans       & \literal{True}, \literal{False} \\[.25em]
      numbers        & \literal{1234}, \literal{3.14159}, \literal{3+4j} \\[.25em]
      strings        & \literal{'help'}, \literal{"hello"}, \literal{"it's mine"},
                       \literal{"""multi-line strings"""} \\[.25em]
      tuples         & \literal{(1, 'this', "other")} \\[.25em]
      lists          & \literal{['this', ['and', 0], 2]} \\[.25em]
      sets           & \literal{\{1,2,3\}} \\[.25em]
      dictionaries   & \literal{\{'first': 'Jim', 'last': 'Brown'\}}
    \end{tabular}
  \end{table}
%
  \begin{itemize}
  \item there are others; details to follow, as necessary
  \end{itemize}
%
\end{frame}

% --------------------------------------
% operators and precedence
\begin{frame}[fragile]
%
  \frametitle{Operators and precedence}
%
  \begin{itemize}
  \item from lower to higher precendece
  \end{itemize}
%
  \begin{table}\footnotesize
    \begin{tabular}{l|l}
      Operator & Description \\ \hline
      \literal{lambda} & used to build anonymous functions \\
      \literal{if} -- \literal{else} & conditional expression (similar to \literal{?:} from
                                       \cc) \\
      \literal{or} & boolean or \\
      \literal{and} & boolean and \\
      \literal{not} & boolean not \\
      \literal{in}, \literal{not in}, \literal{is}, \literal{is not} &
      \multirow{2}{*}{membership tests, identity tests, comparisons} \\
      \literal{<}, \literal{<=}, \literal{>}, \literal{>=}, \literal{!=}, \literal{==} & \\
      \literal{|} & bitwise or \\
      \literal{\^{}} & bitwise xor \\
      \literal{\&} & bitwise and \\
      \literal{<<}, \literal{>>} & left and right bit shifts \\
      \literal{+}, \literal{-} & binary addition, binary subtraction \\
      \literal{*}, \literal{/}, \literal{//}, \literal{\%} &
      multiplication, division, integer division, modulo \\
      \literal{+}, \literal{-}, \literal{\~{}} & positive, negative, bitwise not \\
      \literal{**} & exponentiation \\
      \literal{[]}, \literal{[:]}, \literal{()}, \literal{.} &
      indexing, slicing, function call, attribute reference 
    \end{tabular}
  \end{table}
%
\end{frame}



% end of file 
