% -*- LaTeX -*-
% -*- coding: utf-8 -*-
%
% michael a.g. aïvázis
% california institute of technology
% (c) 1998-2012 all rights reserved
%

\lecture{Introduction to python}{20120404}

% --------------------------------------
% programming paradigms
\begin{frame}[fragile]
%
  \frametitle{Languages and programming paradigms}
%
  \begin{itemize}
%
  \item a very active area of research
    \begin{itemize}
    \item dozens of languages and runtime environments of the last 50 years
    \end{itemize}
%
  \item the survivors:
    \begin{itemize}
      \item procedural programming, and its offspring structured programming
      \item functional programming
      \item object oriented programming
    \end{itemize}
%
  \item current areas of research:
    \begin{itemize}
    \item component oriented programming
    \item aspect programming
    \end{itemize}
%
  \item languages are important:
    \begin{itemize}
    \item they reflect an approach to computing
    \item they shape what is easily expressible
    \end{itemize}
%
  \item we'll take a quick tour of python
    \begin{itemize}
    \item resources: \url{www.python.org}
    \item overview of the language
    \item interactive sessions with the interpreter
    \item building extensions in \cc/\cpp
    \end{itemize}
%
  \end{itemize}
%
\end{frame}

% --------------------------------------
% preview: a python script
\begin{frame}[fragile]
%
  \frametitle{A python script}
%
  \begin{itemize}
  \item python reads like pseudocode
  \item here is the code for the $\pi$ estimator using Monte Carlo integration over the quarter
    disk
  \end{itemize}
%
\python{
  firstline=9,lastline=25,
  label={lst:python:pi},
  caption={\srcfile{pi.py}: Estimating $\pi$ in python},
}{listings/pi.py}

%
\end{frame}

% --------------------------------------
% overview
\begin{frame}[fragile]
%
  \frametitle{Overview}
%
  \begin{itemize}
%
  \item built-in objects and their operators
    \begin{itemize}
    \item numbers, strings, containers
    \item files
    \end{itemize}
%
  \item statements
    \begin{itemize}
      \item evaluating expressions, explicit and implicit assignments, logic, iteration
    \end{itemize}
%
  \item functions
    \begin{itemize}
    \item scope rules, argument passing, callable objects
    \end{itemize}
%
  \item modules and packages
    \begin{itemize}
    \item name qualification, importing symbols
    \end{itemize}
%
  \item user defined objects
    \begin{itemize}
    \item declarations and definitions, inheritance, overloading operators
    \end{itemize}
%
  \item exceptions
    \begin{itemize}
    \item raising and catching, exception hierarchies
    \end{itemize}
%
  \end{itemize}
%
\end{frame}

% --------------------------------------
% syntax
\begin{frame}[fragile]
%
  \frametitle{Syntax}
%
  \begin{itemize}
%
  \item comments: from a \literal{\#} to the end of the line
%
  \item indentation denotes scope
    \begin{itemize}
    \item avoid using tab characters; set your editor to insert a fixed number of spaces when
      the tab key is pressed
    \end{itemize}
%
  \item statements end at the end of the line, or at \literal{;}
    \begin{itemize}
    \item open delimiters imply continuation
      \item explicit continuation with \literal{$\backslash$}, but considered obsolete
    \end{itemize}
%
  \item identifiers
    \begin{itemize}
    \item start with an underscore or letter, followed by underscores, letters or digits
    \item unicode is supported in identifier names; details at
      {\scriptsize \url{http://docs.python.org/py3k/reference/lexical_analysis.html#literals}}
    \item identifiers are case sensitive
    \end{itemize}
%
  \item certain classes of identifiers have special meaning
    \begin{itemize}
    \item the pattern \literal{\_\_*\_\_} is reserved by python for its own use
    \item identifiers of the form \literal{\_\_*} in class definition are mangled and become
      private
    \item identifiers of the form \literal{\_*} are not bulk imported from modules; more on
      this later
    \end{itemize}
%
  \end{itemize}
%
\end{frame}

% --------------------------------------
% reserved words
\begin{frame}[fragile]
%
  \frametitle{Reserved words}
%
  \begin{itemize}
%
  \item the following words are reserved
    \begin{table}
      \begin{tabular}{lllll}
%
 \keyword{False} & \keyword{None} & \keyword{True} & \operator{and} & \keyword{as} \\
 \keyword{assert} & \keyword{break} & \keyword{class} & \keyword{continue} & \keyword{def} \\ 
 \keyword{del} & \keyword{elif} & \keyword{else} & \keyword{except} & \keyword{finally} \\ 
 \keyword{for} & \keyword{from} & \keyword{global} & \keyword{if} & \keyword{import} \\
 \operator{in} & \operator{is} & \keyword{lambda} & \keyword{nonlocal} & \operator{not} \\
 \operator{or} & \keyword{pass} & \keyword{raise} & \keyword{return} & \keyword{try} \\ 
 \keyword{while} & \keyword{with} & \keyword{yield} & &
%
      \end{tabular}
    \end{table}
%
  \end{itemize}
%
\end{frame}

% --------------------------------------
% built-in objects
\begin{frame}[fragile]
%
  \frametitle{Built-in objects}
%
  \begin{itemize}
  \item the more commonly used types
  \end{itemize}
%
  \begin{table}\footnotesize
    \begin{tabular}{l|l}
      \emph{Type}    & \emph{Sample} \\[.25em] \hline \\
      booleans       & \keyword{True}, \keyword{False} \\[.25em]
      numbers        & \literal{1234}, \literal{3.14159}, \literal{3+4j} \\[.25em]
      strings        & \literal{'help'}, \literal{"hello"}, \literal{"it's mine"},
                       \literal{"""multi-line strings"""} \\[.25em]
      tuples         & \literal{(1, 'this', "other")} \\[.25em]
      lists          & \literal{['this', ['and', 0], 2]} \\[.25em]
      sets           & \literal{\{1,2,3\}} \\[.25em]
      dictionaries   & \literal{\{'first': 'Jim', 'last': 'Brown'\}}
    \end{tabular}
  \end{table}
%
  \begin{itemize}
  \item there are others; details to follow, as necessary
  \end{itemize}
%
\end{frame}

% --------------------------------------
% operators and precedence
\begin{frame}[fragile]
%
  \frametitle{Operators and precedence}
%
  \begin{itemize}
  \item from lower to higher precendece
  \end{itemize}
%
  \begin{table}\footnotesize
    \begin{tabular}{l|l}
      \emph{Operator} & \emph{Description} \\ \hline
      \operator{lambda} & used to build anonymous functions \\
      \keyword{if} -- \keyword{else} & conditional expression (similar to \literal{?:} from
                                       \cc) \\
      \operator{or} & boolean or \\
      \operator{and} & boolean and \\
      \operator{not} & boolean not \\
      \operator{in}, \operator{not in}, \operator{is}, \operator{is not} &
      \multirow{2}{*}{membership tests, identity tests, comparisons} \\
      \operator{<}, \operator{<=}, \operator{>}, \operator{>=}, \operator{!=}, \operator{==} & \\
      \operator{|} & bitwise or \\
      \operator{\^{}} & bitwise xor \\
      \operator{\&} & bitwise and \\
      \operator{<<}, \operator{>>} & left and right bit shifts \\
      \operator{+}, \operator{-} & binary addition, binary subtraction \\
      \operator{*}, \operator{/}, \operator{//}, \operator{\%} &
      multiplication, division, integer division, modulo \\
      \operator{+}, \operator{-}, \operator{\~{}} & positive, negative, bitwise not \\
      \operator{**} & exponentiation \\
      \literal{[]}, \literal{[:]}, \literal{()}, \literal{.} &
      indexing, slicing, function call, attribute reference 
    \end{tabular}
  \end{table}
%
\end{frame}

% --------------------------------------
% number
\begin{frame}[fragile]
%
  \frametitle{Numbers}
%
  \begin{itemize}
%
  \item numeric literals
    \begin{table}\footnotesize
      \begin{tabular}{l|l}
        \emph{Literal} & \emph{Description} \\ \hline
        \literal{1234} & arbitrary precision integers \\
        \literal{3.1415}, \literal{6.023e23} & floats \\
        \literal{j}, \literal{1+j} & complex numbers \\
        \literal{0b1001} & binary integers \\
        \literal{0o777} & octal integers \\
        \literal{0xdeadbeef} & hexadecimal integers
      \end{tabular}
    \end{table}
%
  \item expressions:
    \begin{itemize}
    \item the usual arithmetic operators
    \item bitwise operators similar to \cc
    \item adjust precedence and association by using parenthesis; be aware of the ``tuple
      conflict''
    \item in expressions with mixed types, python converts towards the wider types
    \end{itemize}
%
  \end{itemize}
%
\end{frame}

% --------------------------------------
% strings
\begin{frame}[fragile]
%
  \frametitle{Strings}
%
  \begin{itemize}
%
  \item string literals
    \begin{table}\footnotesize
      \begin{tabular}{ll}
        \emph{Literal} & \emph{Description} \\ \hline
        \literal{'Hello'} & single quotes \\
        \literal{"Hello"}, & double quotes \\
        \literal{"It's embedded"} & mix and match \\
        \literal{'''Hello'''}, \literal{"""Hello"""} & triple quotes -- extend over multiple
        lines \\
      \end{tabular}
    \end{table}
%
  \item strings are immutable ordered sequences of unicode characters
    \begin{itemize}
    \item no \literal{char} type: \literal{'a'} is a string with one character
    \item native support for all code pages
  \end{itemize}
%
  \item common operations: given the strings $s_{i}$:
    \begin{table}\footnotesize
      \begin{tabular}{ll}
        \emph{Expression} & \emph{Description} \\ \hline
        \literal{s1+s2}, \literal{s*4} & concatenation, repetition \\
        \literal{s[3]}, \literal{s[3:4]} & indexing, slicing \\
        \literal{len(s)} & length\\
        \literal{'H' in s} & membership\\
        \literal{for x in s} & iteration
      \end{tabular}
    \end{table}
%
  \end{itemize}
%
\end{frame}

% --------------------------------------
% string coercions
\begin{frame}[fragile]
%
  \frametitle{String coercions}
%
  \begin{itemize}
%
  \item all objects are representable as strings
  \item coercion can be triggered explicitly
    \begin{itemize}
    \item using the \keyword{str} constructor
    \item using the \keyword{repr} built-in function
    \end{itemize}
  \item strings have a powerful formatting method:
    \begin{ipython}{}
      'temperature={:+13.4f}, pressure={:13.4f}'.format(T, P)
    \end{ipython}
  \item more details after we go over the rules for function calls
%
  \end{itemize}
%
\end{frame}

% --------------------------------------
% tuples
\begin{frame}[fragile]
%
  \frametitle{Tuples}
%
  \begin{itemize}
%
  \item tuples are immutable ordered inhomogeneous sequences of objects
    \begin{table}\footnotesize
      \begin{tabular}{ll}
        \emph{Literal} & \emph{Description} \\ \hline
        \literal{()} & the empty tuple \\
        \literal{(1,)}, & a tuple with one item \\
        \literal{(1,2,3,4)} & a longer tuple \\
        \literal{(1,'Hello', 'world')} & tuple elements don't have to be the same type\\
        \literal{(1,2,('Hello', 'world'),4)} & tuples nest arbitrarily deeply
      \end{tabular}
    \end{table}
%
  \item common operations:
    \begin{table}\footnotesize
      \begin{tabular}{ll}
        \emph{Expression} & \emph{Description} \\ \hline
        \literal{t1+t2}, \literal{t*4} & concatenation, repetition \\
        \literal{t[3]}, \literal{t[3:4]} & indexing, slicing \\
        \literal{len(t)} & length\\
        \literal{x in t} & membership\\
        \literal{for x in t} & iteration
      \end{tabular}
    \end{table}
%
  \end{itemize}
%
\end{frame}

% --------------------------------------
% lists
\begin{frame}[fragile]
%
  \frametitle{Lists}
%
  \begin{itemize}
%
  \item lists are mutable ordered inhomogeneous sequences of objects
    \begin{table}\footnotesize
      \begin{tabular}{ll}
        \emph{Literal} & \emph{Description} \\ \hline
        \literal{[]} & the empty list \\
        \literal{[1])}, & a list with one item \\
        \literal{[1,2,3,4]} & a longer list \\
        \literal{[1,'Hello', 'world']} & list elements don't have to be the same type\\
        \literal{[1,2,['Hello', 'world'],4]} & lists nest arbitrarily deeply
      \end{tabular}
    \end{table}
%
  \item common operations:
    \begin{table}\footnotesize
      \begin{tabular}{ll}
        \emph{Expression} & \emph{Description} \\ \hline
        \literal{l1+l2}, \literal{l*4} & concatenation, repetition \\
        \literal{len(l)}, \literal{x in l}, \literal{for x in l} & length, membership, iteration \\
        \literal{l[3]}, \literal{l[3:4]} & indexing, slicing \\
        \literal{del l[3]}, \literal{l[3:7]=[]} & shrink \\
        \literal{l[3:7]=[1,2]} & slice assignment \\
        \literal{l.append(1)} & add to the end of the list
      \end{tabular}
    \end{table}
%
  \end{itemize}
%
\end{frame}

% --------------------------------------
% sets
\begin{frame}[fragile]
%
  \frametitle{Sets}
%
  \begin{itemize}
%
  \item sets are mutable unordered inhomogeneous containers of objects
    \begin{table}\footnotesize
      \begin{tabular}{ll}
        \emph{Literal} & \emph{Description} \\ \hline
        \literal{set()} & the empty set \\
        \literal{\{1\})}, & a set with one item \\
        \literal{\{1,2,3,4\}} & a longer set \\
        \literal{\{1,'Hello', 'world'\}} & set elements don't have to be the same type
      \end{tabular}
    \end{table}
%
  \item common operations:
    \begin{table}\footnotesize
      \begin{tabular}{ll}
        \emph{Expression} & \emph{Description} \\ \hline
        \literal{s1|s2} & union \\
        \literal{s1\&s2} & intersection \\
        \literal{s1-s2}, \literal{s1\^{}s2} & difference, symmetric difference \\
        \literal{len(s)}, \literal{x in s}, \literal{for x in s} & length, membership, iteration \\
        \literal{s.add(x)} & add \\
        \literal{s.discard(x)} & remove if present\\
        \literal{s.remove(x)} & remove; raise \literal{KeyError} if not present
      \end{tabular}
    \end{table}
%
  \end{itemize}
%
\end{frame}

% --------------------------------------
% dictionaries
\begin{frame}[fragile]
%
  \frametitle{Dictionaries}
%
  \begin{itemize}
%
  \item dictionaries are mutable inhomogeneous associative maps
    \begin{table}\footnotesize
      \begin{tabular}{ll}
        \emph{Literal} & \emph{Description} \\ \hline
        \literal{\{\}} & the empty dictionary \\
        \literal{\{'first':'Guido', 'last':'van Rossum'\}}, & a dictionary with two items \\
      \end{tabular}
    \end{table}
%
  \item dictionary keys must be \emph{hashable}; dictionary values may be of any type
%
  \item common operations
    \begin{table}\footnotesize
      \begin{tabular}{ll}
        \emph{Expression} & \emph{Description} \\ \hline
        \literal{d['first']='Guido'} & make an association \\
        \multirow{2}{*}{\literal{d['first']}} & value retrieval \\
        & raise \literal{KeyError} if key not present \\
        \literal{d.get('first')}, \literal{d.get('first', default='')} & value retrieval \\
        \literal{len(d)} & the number of keys \\
        \literal{x in d} & key presence \\
        \literal{d.keys()}, \literal{d.values()}, \literal{d.items()} &
        views of the dictionary contents
      \end{tabular}
    \end{table}
%
  \end{itemize}
%
\end{frame}

% --------------------------------------
% other built-in objects
\begin{frame}[fragile]
%
  \frametitle{Other built-in objects}
%
  \begin{itemize}
%
  \item 
    \begin{itemize}
    \item
    \end{itemize}
%
  \end{itemize}
%
\end{frame}

% --------------------------------------
% type hierarchy
\begin{frame}[fragile]
%
  \frametitle{Type hierarchy}
%
  \begin{itemize}
%
  \item 
    \begin{itemize}
    \item
    \end{itemize}
%
  \end{itemize}
%
\end{frame}

% --------------------------------------
% comparisons
\begin{frame}[fragile]
%
  \frametitle{Comparisons}
%
  \begin{itemize}
%
  \item the following are considered false in logical expressions
    \begin{itemize}
    \item the boolean constant \keyword{False}
    \item the special object \keyword{None}
    \item the number \literal{0}
    \item any empty container
    \end{itemize}
%
  \item other values are true, including the boolean constant \keyword{True}
%
  \item operators:
    \begin{itemize}
    \item the object identity operator \operator{is}
    \item the container membership operator \operator{is}
    \item the usual relational operators -- borrowed from \cc
    \end{itemize}
%
  \item object comparisons
    \begin{itemize}
    \item strings are compared lexicographically
    \item nested data structured are checked recursively
    \item lists and tuples are compared depth first, left to right
    \item dictionaries are compared as sorted (key, value) tuples
    \item user defined types can supply custom comparison functions by overloading the special
      methods
    \end{itemize}
%
  \end{itemize}
%
\end{frame}

% --------------------------------------
% assignments
\begin{frame}[fragile]
%
  \frametitle{Assignments}
%
  \begin{itemize}
%
  \item explicitly, using the \operator{=} operator
    \begin{ipython}{}
      greeting = 'Hello world!'
    \end{ipython}
    which makes the symbol \literal{greeting} become a name for the literal string in the right
    hand side
%
  \item implicitly, when defining a function
    \begin{ipython}{}
      def greeting(name): pass
    \end{ipython}
    which makes \literal{greeting} become a name for the function object built out of the
    statements that follow the \literal{:}
%
  \item implicitly, when defining a class
    \begin{ipython}{}
      class greeting: pass
    \end{ipython}
    which makes \literal{greeting} become a name for the \emph{class object} built out of the
    statements that follow the \literal{:}
%
  \item implicitly, when importing symbols from a module
    \begin{ipython}{}
      import sys
      from math import pi
      from math import pi as #@$\pi$@ 
    \end{ipython}
%
  \end{itemize}
%
\end{frame}

% --------------------------------------
% selections
\begin{frame}[fragile]
%
  \frametitle{Selections}
%
  \begin{itemize}
%
  \item using \keyword{if}
    \begin{ipython}{}
      if <expression>:
          <statements>
      elif <expression>:
          <statements>
      else:
          <statements>
    \end{ipython}
%
  \item no \keyword{switch} statement
    \begin{itemize}
    \item use an \keyword{if} cascade
    \item better yet, think about achieving the same result using containers; it's typically
      more efficient and robust
    \end{itemize}
%
  \end{itemize}
%
\end{frame}

% --------------------------------------
% iteration
\begin{frame}[fragile]
%
  \frametitle{Iteration}
%
  \begin{itemize}
%
  \item the \keyword{while} loop
    \begin{ipython}{}
      while <expression>:
          <statements>
      else:
          <statements>
    \end{ipython}
    executes the statements in its body until its expression evaluates to \keyword{False}, at
    which point it executes the optional \keyword{else} clause
%
  \item the \keyword{for} loop 
    \begin{ipython}{}
      for <name> in <expression>:
          <statements>
      else:
          <statements>
    \end{ipython}
    evaluates its expression once to get an iterator, binds \literal{name} to each object
    provided by the iterator and executes its body; iteration stops when the iterator is
    exhausted, at which point the \keyword{else} clause is executed; if execution encounters a
    \keyword{break} statement in the body of the loop, iteration is terminated, and the
    \keyword{else} clause is skipped; if execution encounters a \keyword{continue} statement,
    it skips the remainder of the loop body and proceeds with the next item from the iterator
%
  \end{itemize}
%
\end{frame}

% --------------------------------------
% functions
\begin{frame}[fragile]
%
  \frametitle{Functions}
%
  \begin{itemize}
%
  \item 
    \begin{itemize}
    \item
    \end{itemize}
%
  \end{itemize}
%
\end{frame}

% --------------------------------------
% scoping rules
\begin{frame}[fragile]
%
  \frametitle{Scope and visibility}
%
  \begin{itemize}
%
  \item 
    \begin{itemize}
    \item
    \end{itemize}
%
  \end{itemize}
%
\end{frame}

% --------------------------------------
% function arguments
\begin{frame}[fragile]
%
  \frametitle{Function arguments}
%
  \begin{itemize}
%
  \item 
    \begin{itemize}
    \item
    \end{itemize}
%
  \end{itemize}
%
\end{frame}

% --------------------------------------
% function objects
\begin{frame}[fragile]
%
  \frametitle{Functions as objects}
%
  \begin{itemize}
%
  \item 
    \begin{itemize}
    \item
    \end{itemize}
%
  \end{itemize}
%
\end{frame}

% --------------------------------------
% printing
\begin{frame}[fragile]
%
  \frametitle{Printing}
%
  \begin{itemize}
%
  \item 
    \begin{itemize}
    \item
    \end{itemize}
%
  \end{itemize}
%
\end{frame}

% --------------------------------------
% namespaces
\begin{frame}[fragile]
%
  \frametitle{Namespaces}
%
  \begin{itemize}
%
  \item 
    \begin{itemize}
    \item
    \end{itemize}
%
  \end{itemize}
%
\end{frame}

% --------------------------------------
% namespace access
\begin{frame}[fragile]
%
  \frametitle{Namespace access}
%
  \begin{itemize}
%
  \item 
    \begin{itemize}
    \item
    \end{itemize}
%
  \end{itemize}
%
\end{frame}

% --------------------------------------
% namespaces as objects
\begin{frame}[fragile]
%
  \frametitle{Namespaces as objects}
%
  \begin{itemize}
%
  \item 
    \begin{itemize}
    \item
    \end{itemize}
%
  \end{itemize}
%
\end{frame}

% --------------------------------------
% classes
\begin{frame}[fragile]
%
  \frametitle{Classes}
%
  \begin{itemize}
%
  \item 
    \begin{itemize}
    \item
    \end{itemize}
%
  \end{itemize}
%
\end{frame}

% --------------------------------------
% inheritance
\begin{frame}[fragile]
%
  \frametitle{Inheritance}
%
  \begin{itemize}
%
  \item 
    \begin{itemize}
    \item
    \end{itemize}
%
  \end{itemize}
%
\end{frame}

% --------------------------------------
% methods
\begin{frame}[fragile]
%
  \frametitle{Methods}
%
  \begin{itemize}
%
  \item 
    \begin{itemize}
    \item
    \end{itemize}
%
  \end{itemize}
%
\end{frame}

% --------------------------------------
% class glossary
\begin{frame}[fragile]
%
  \frametitle{Class glossary}
%
  \begin{itemize}
%
  \item 
    \begin{itemize}
    \item
    \end{itemize}
%
  \end{itemize}
%
\end{frame}

% --------------------------------------
% special methods
\begin{frame}[fragile]
%
  \frametitle{Special methods}
%
  \begin{itemize}
%
  \item 
    \begin{itemize}
    \item
    \end{itemize}
%
  \end{itemize}
%
\end{frame}

% --------------------------------------
% namespace rules
\begin{frame}[fragile]
%
  \frametitle{Blank}
%
  \begin{itemize}
%
  \item 
    \begin{itemize}
    \item
    \end{itemize}
%
  \end{itemize}
%
\end{frame}

% --------------------------------------
% classes as objects
\begin{frame}[fragile]
%
  \frametitle{Classes as objects}
%
  \begin{itemize}
%
  \item 
    \begin{itemize}
    \item
    \end{itemize}
%
  \end{itemize}
%
\end{frame}

% --------------------------------------
% methods as objects
\begin{frame}[fragile]
%
  \frametitle{Methods as objects}
%
  \begin{itemize}
%
  \item 
    \begin{itemize}
    \item
    \end{itemize}
%
  \end{itemize}
%
\end{frame}

% --------------------------------------
% exceptions
\begin{frame}[fragile]
%
  \frametitle{Exceptions}
%
  \begin{itemize}
%
  \item 
    \begin{itemize}
    \item
    \end{itemize}
%
  \end{itemize}
%
\end{frame}

% --------------------------------------
% raising exceptions
\begin{frame}[fragile]
%
  \frametitle{Raising exceptions}
%
  \begin{itemize}
%
  \item 
    \begin{itemize}
    \item
    \end{itemize}
%
  \end{itemize}
%
\end{frame}

% --------------------------------------
% catching exceptions
\begin{frame}[fragile]
%
  \frametitle{Catching exceptions}
%
  \begin{itemize}
%
  \item 
    \begin{itemize}
    \item
    \end{itemize}
%
  \end{itemize}
%
\end{frame}

% --------------------------------------
% object oriented programming
\begin{frame}[fragile]
%
  \frametitle{Object oriented programming}
%
  \begin{itemize}
%
  \item 
    \begin{itemize}
    \item
    \end{itemize}
%
  \end{itemize}
%
\end{frame}

% end of file 
