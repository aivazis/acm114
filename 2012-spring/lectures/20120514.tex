% -*- LaTeX -*-
% -*- coding: utf-8 -*-
%
% michael a.g. aïvázis
% california institute of technology
% (c) 1998-2012 all rights reserved
%

\lecture{Introduction to pyre}{20120514}

% --------------------------------------
% assessment
\begin{frame}[fragile]
%
  \frametitle{Design assessment}
%
  \begin{itemize}
%
  \item the OO solution represents several improvements over the initial implementation
    \begin{itemize}
    \item the problem has been decomposed into several parts that can evolve independently
    \item there is a natural correspondence with the mathematics of Monte Carlo integration 
    \item our driver assembles the pieces together in a natural fashion
    \end{itemize}
%
  \item our abstract base classes
    \begin{itemize}
      \item do not play a string r\^ole in dynamically typed languages, such as python
      \item in strongly typed languages, they become constraints on their subclasses
    \end{itemize}
%
  \item change involves modifying the script and running again
    \begin{itemize}
    \item how would you build the performance table, or compare multiple random number
      generators?
    \end{itemize}
%
  \end{itemize}
%
\end{frame}

% --------------------------------------
% the Integrator
\begin{frame}[fragile]
%
  \frametitle{Reassembling the pieces}
%
  if we are interested in turning our simple script into a package, we should build a class to
  take responsibility of managing all the necessary parts; consider the class
  \class{Integrator}
%
  \begin{ipython}{}
class Integrator:
    """
    The abstract base class for integrators
    """

    # interface
    def integrate(self):
        """
        Integrate my {integrand} over my {region}
        """
        raise NotImplementedError(
            "class {.__name__!r} should implement 'integrate'".format(type(self)))
  \end{ipython}
%
  and let descendants specify the details of the quadrature
%
\end{frame}

% --------------------------------------
% Monte Carlo
\begin{frame}[fragile]
%
  \frametitle{The Monte Carlo integrator}
%
  perhaps something like
%
  \begin{ipython}[basicstyle=\tt\tiny]{}
from Integrator import Integrator

class MonteCarlo(Integrator):
    """
    A Monte Carlo integrator
    """

    # public state
    samples = 10**5 # default value?
    box = ????
    mesh = ????
    region = ????
    integrand = ????

    # interface
    def integrate(self):
        """
        Integrate my {integrand} over the {region} by sampling it using random
        numbers from {mesh}
        """
        # compute the normalization
        normalization = self.box.measure() / self.samples
        # build the sample set
        points = self.mesh.points(n=self.samples, box=self.box)
        # narrow the set down to the interior points
        interior = self.region.interior(points=points)
        # build the value of the integral
        integral = normalization * sum(self.integrand.eval(interior))
        # and return it
        return integral
  \end{ipython}
%
  where the parts are specified at \emph{runtime}?
%
\end{frame}

% --------------------------------------
% inversion of control
\begin{frame}[fragile]
%
  \frametitle{Inversion of control}
%
  let's be a bit more ambitious: 
  \begin{itemize}
  \item can we postpone until \emph{runtime} the selection, instantiation and initialization of
    some subset of the classes in our applications?
  \item and hence give the end user total control over \emph{what} and \emph{how}?
  \end{itemize}

  the correct solution would to the application like a user interface
%
\end{frame}

% --------------------------------------
% traits
\begin{frame}[fragile]
%
  \frametitle{Small steps: properties}
%
  let's step back and contemplate a simpler problem
%
  \begin{ipython}{}
class Disk:

    # public state
    radius = 1     # default value
    center = (0,0) # default value

    # interface
    def interior(self, points):
        ...
  \end{ipython}
%
  what do we have to do to tie instances of \class{Disk} with information in some configuration
  file
%
  \begin{icfg}{}
    [ disk1 ]
    center = (-1,1)   ; leave {radius} alone

    [ disk2 ]
    radius = .5
    center = (1,1)
  \end{icfg}
%
  or, equivalently, from the command line
%
  \begin{ish}{}
gauss.py --disk1.center=(1,1) --disk2.radius=.5 --disk2.center=(-1,1)
  \end{ish}
%
\end{frame}

% --------------------------------------
% components:
\begin{frame}[fragile]
%
  \frametitle{Components}
%
  \begin{itemize}
%
  \item informally, \emph{classes} are software specifications that establish a relationship
    between \emph{state} and \emph{behavior}
    \begin{itemize}
    \item we have syntax that let's us specify these very close to each other
    \end{itemize}
%
  \item \emph{instances} are containers of state; there are special rules
    \begin{itemize}
    \item that grant access to this state
    \item allow you to call functions that get easy access to this state
    \end{itemize}
%
  \item \emph{components} are classes that specifically grant access to some of their state to
    the end user
    \begin{itemize}
    \item the public data are the \emph{properties} of the component
    \end{itemize}
%
  \item rule 1: components have properties
%
  \end{itemize}
%
\end{frame}

% --------------------------------------
% disk as components
\begin{frame}[fragile]
%
  \frametitle{A trivial component}
%
  \pyre\ is a package that provides support for writing components
%
  \begin{ipython}{}
import pyre

class Disk(pyre.component):

    # public state
    radius = pyre.properties.float()
    radius.default = 1
    radius.doc = 'the radius of the disk'

    center = pyre.properties.array()
    center.default = (0,0)
    center.doc = 'the location of the center of the circle'

    # interface
    ...
  \end{ipython}
%
  why bother specifying the type of component properties?
  \begin{itemize}
  \item command line, configuration files, dialog boxes, web pages: they all gather information
    from the user as strings
  \item we need \emph{metadata} so we can convert from strings to the intended object
  \end{itemize}
%
\end{frame}

% --------------------------------------
% the names of things
\begin{frame}[fragile]
%
  \frametitle{The names of things}
%
  in order to connect components to configurations, we need explicit associations
  \begin{itemize}
  \item component instances must be given unique names
  \item component classes must be given unique family names
  \item components belong to packages
  \end{itemize}
%
  \begin{ipython}{}
import pyre

class Disk(pyre.component, family="gauss.shapes.disk"):

    # public state
    radius = pyre.properties.float()
    radius.default = 1
    radius.doc = 'the radius of the disk'

    center = pyre.properties.array()
    center.default = (0,0)
    center.doc = 'the location of the center of the circle'
    ...
  \end{ipython}
%
  and here are a couple of component instances
%
  \begin{ipython}{}
left = Disk(name='disk1')
right = Disk(name='disk2')
  \end{ipython}
\end{frame}

% --------------------------------------
% configuration
\begin{frame}[fragile]
%
  \frametitle{Configuration}
%
  \begin{itemize}
  \item the package name is deduced from the component family name
    \begin{itemize}
    \item it is the part up to the first delimiter
    \end{itemize}
  \item \pyre\ automatically loads configuration files whose name matches the name of a package
  \item there's even a way to override the default values that the developer hardwired into
    the class declaration
  \end{itemize}
%
  \begin{icfg}{}
    [ gauss.shapes.disk ] ; the family name
    radius = 2
    center = (-1,-1)

    [ disk1 ] ; the name of an instance
    center = (-1,1)   ; leave {radius} alone

    [ disk2 ] ; the name of another instance
    radius = .5
    center = (1,1)
  \end{icfg}
%
\end{frame}

% end of file 
