% -*- LaTeX -*-
% -*- coding: utf-8 -*-
%
% michael a.g. aïvázis
% california institute of technology
% (c) 1998-2012 all rights reserved
%

\lecture{Introduction to python}{20120411}

% --------------------------------------
% comparisons
\begin{frame}[fragile]
%
  \frametitle{Comparisons}
%
  \begin{itemize}
%
  \item the following are considered false in logical expressions
    \begin{itemize}
    \item the boolean constant \keyword{False}
    \item the special object \keyword{None}
    \item the number \literal{0}
    \item any empty container
    \end{itemize}
%
  \item other values are true, including the boolean constant \keyword{True}
%
  \item operators:
    \begin{itemize}
    \item the object identity operator \operator{is}
    \item the container membership operator \operator{in}
    \item the usual relational operators -- borrowed from \cc
    \end{itemize}
%
  \item object comparisons
    \begin{itemize}
    \item strings are compared lexicographically
    \item nested data structured are checked recursively
    \item lists and tuples are compared depth first, left to right
    \item dictionaries are compared as sorted (key, value) tuples
    \item user defined types can supply custom comparison functions by overloading the special
      methods
    \end{itemize}
%
  \end{itemize}
%
\end{frame}

% --------------------------------------
% assignments
\begin{frame}[fragile]
%
  \frametitle{Assignments}
%
  \begin{itemize}
%
  \item explicitly, using the \operator{=} operator
    \begin{ipython}{}
      greeting = 'Hello world!'
    \end{ipython}
    which makes the symbol \literal{greeting} become a name for the literal string in the right
    hand side
%
  \item implicitly, when defining a function
    \begin{ipython}{}
      def greeting(name): pass
    \end{ipython}
    which makes \literal{greeting} become a name for the function object built out of the
    statements that follow the \literal{:}
%
  \item implicitly, when defining a class
    \begin{ipython}{}
      class greeting: pass
    \end{ipython}
    which makes \literal{greeting} become a name for the \emph{class object} built out of the
    statements that follow the \literal{:}
%
  \item implicitly, when importing symbols from a module
    \begin{ipython}{}
      import sys
      from math import pi
      from math import pi as #@$\pi$@ 
    \end{ipython}
%
  \end{itemize}
%
\end{frame}

% --------------------------------------
% selections
\begin{frame}[fragile]
%
  \frametitle{Selections}
%
  \begin{itemize}
%
  \item using \keyword{if}
    \begin{ipython}{}
      if <expression>:
          <statements>
      elif <expression>:
          <statements>
      else:
          <statements>
    \end{ipython}
%
  \item no \keyword{switch} statement
    \begin{itemize}
    \item use an \keyword{if} cascade
    \item better yet, think about achieving the same result using containers; it's typically
      more efficient and robust
    \end{itemize}
%
  \end{itemize}
%
\end{frame}

% --------------------------------------
% iteration
\begin{frame}[fragile]
%
  \frametitle{Iteration}
%
  \begin{itemize}
%
  \item the \keyword{while} loop
    \begin{ipython}{}
      while <expression>:
          <statements>
      else:
          <statements>
    \end{ipython}
    executes the statements in its body until its expression evaluates to \keyword{False}, at
    which point it executes the optional \keyword{else} clause
%
  \item the \keyword{for} loop 
    \begin{ipython}{}
      for <name> in <expression>:
          <statements>
      else:
          <statements>
    \end{ipython}
    evaluates its expression once to get an iterator, binds \literal{name} to each object
    provided by the iterator and executes its body; iteration stops when the iterator is
    exhausted, at which point the \keyword{else} clause is executed; if execution encounters a
    \keyword{break} statement in the body of the loop, iteration is terminated, and the
    \keyword{else} clause is skipped; if execution encounters a \keyword{continue} statement,
    it skips the remainder of the loop body and proceeds with the next item from the iterator
%
  \end{itemize}
%
\end{frame}

% end of file 
