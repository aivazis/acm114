% -*- LaTeX -*-
% -*- coding: utf-8 -*-
%
% michael a.g. aïvázis
% california institute of technology
% (c) 1998-2012 all rights reserved
%

\lecture{Introduction to \pyre}{20120525}

% --------------------------------------
% module definition
\begin{frame}[fragile]
%
  \frametitle{Module definition}
%
  the identifier \identifier{module\_definition} refers to a \keyword{struct}
  \begin{iC++}{}
// the module documentation string
const char * const __doc__ = "sample module documentation string";
// the module definition structure
PyModuleDef gsl::module_definition = {
    // header
    PyModuleDef_HEAD_INIT,
    // the name of the module
    "gsl",
    // the module documentation string
    __doc__,
    // size of the per-interpreter state of the module
    -1, // don't touch
    // the methods defined in this module
    gsl::module_methods
};
  \end{iC++}
%
\end{frame}

% --------------------------------------
% method table
\begin{frame}[fragile]
%
  \frametitle{Method table}
%
  the method table is an array of method meta-data
%
  \begin{iC++}{}
// the module method table
PyMethodDef module_methods[] = {
    // random numbers
    { rng::avail__name__, rng::avail, METH_VARARGS, rng::avail__doc__ },
    { rng::alloc__name__, rng::alloc, METH_VARARGS, rng::alloc__doc__ },
    { rng::get__name__, rng::get, METH_VARARGS, rng::get__doc__ },
    { rng::name__name__, rng::name, METH_VARARGS, rng::name__doc__ },
    { rng::range__name__, rng::range, METH_VARARGS, rng::range__doc__ },
    { rng::set__name__, rng::set, METH_VARARGS, rng::set__doc__ },
    { rng::uniform__name__, rng::uniform, METH_VARARGS, rng::uniform__doc__ },
};
  \end{iC++}
%
  we have placed all \identifier{RNG} related symbols in their own namespace, so we can shorten
  their names without conflicts
%
\end{frame}

% --------------------------------------
% available RNG
\begin{frame}[fragile]
%
  \frametitle{Getting the set of available generators}
%
  \begin{iC++}{}
// get the name of all the generators known to GSL
const char * const gsl::rng::avail__name__ = "rng_avail";
const char * const gsl::rng::avail__doc__ =
     "return the set of all known generators";

PyObject * 
gsl::rng::avail(PyObject *, PyObject * args) {
    // unpack the argument tuple
    int status = PyArg_ParseTuple(args, ":rng_avail");
    // raise an exception if something went wrong
    if (!status) return 0;

    // make a frozen set to hold the names
    PyObject *names = PyFrozenSet_New(0);

    // iterate over the registered names
    for (
         gsl::rng::map_t::const_iterator i = gsl::rng::generators.begin();
         i != gsl::rng::generators.end();
         i++ ) {
        // add the name to the set
        PySet_Add(names, PyUnicode_FromString(i->first.c_str()));
    }

    // return the set of names
    return names;
}
  \end{iC++}
%
\end{frame}

% --------------------------------------
% allocating a generator
\begin{frame}[fragile]
%
  \frametitle{Allocating generators}
%
  \begin{iC++}{}
// allocation
const char * const gsl::rng::alloc__name__ = "rng_alloc";
const char * const gsl::rng::alloc__doc__ = "allocate a rng";

PyObject * 
gsl::rng::alloc(PyObject *, PyObject * args) {
    // place holders for the python arguments
    char * name;
    // unpack the argument tuple
    int status = PyArg_ParseTuple(args, "s:rng_alloc", &name);
    // raise an exception, if something went wrong
    if (!status) return 0;

    // get the rng type
    const gsl_rng_type *algorithm = gsl::rng::generators[name];
    // if it's not in table
    if (!algorithm) {
        PyErr_SetString(PyExc_ValueError, "unknown random number generator");
        return 0;
    }

    // allocate one
    gsl_rng * r = gsl_rng_alloc(algorithm);
    // wrap it in a capsule and return it
    return PyCapsule_New(r, capsule_t, free);
}
  \end{iC++}
%
\end{frame}

% --------------------------------------
% destroying generators
\begin{frame}[fragile]
%
  \frametitle{Deallocating generators}
%
  capsules are python objects that hold pointers to low level entities; the name is used to
  check that you received the capsule you expected
%
  \begin{iC++}{}
// capsules
namespace gsl {
    // rng
    namespace rng {
        const char * const capsule_t = "gsl.rng"; 
        void free(PyObject *);
    }
}
  \end{iC++}
%
  the destructor is an example
%
  \begin{iC++}{}
// destructor
void gsl::rng::free(PyObject * capsule)
{
    // bail out if the capsule is not valid
    if (!PyCapsule_IsValid(capsule, gsl::rng::capsule_t)) return;
    // get the rng
    gsl_rng * r = static_cast<gsl_rng *>(
        PyCapsule_GetPointer(capsule, gsl::rng::capsule_t));
    // deallocate
    gsl_rng_free(r);
    // and return
    return;
}
  \end{iC++}
%
\end{frame}

% --------------------------------------
% generating a random number
\begin{frame}[fragile]
%
  \frametitle{Generating a random number}
%
  \begin{iC++}{}
// a random double in [0,1)
const char * const gsl::rng::uniform__name__ = "rng_uniform";
const char * const gsl::rng::uniform__doc__ = 
    "return the next random integer with the range of the generator";

PyObject * 
gsl::rng::uniform(PyObject *, PyObject * args) {
    // the arguments
    PyObject * capsule;
    // unpack the argument tuple
    int status = PyArg_ParseTuple(args, "O!:rng_uniform", &PyCapsule_Type, &capsule);
    // raise an exception if something went wrong
    if (!status) return 0;
    // bail out if the capsule is not valid
    if (!PyCapsule_IsValid(capsule, capsule_t)) {
        PyErr_SetString(PyExc_TypeError, "invalid rng capsule");
        return 0;
    }

    // get the rng
    gsl_rng * r = static_cast<gsl_rng *>(PyCapsule_GetPointer(capsule, capsule_t));

    // get a random number and return it
    return PyFloat_FromDouble(gsl_rng_uniform(r));
}
  \end{iC++}
%
\end{frame}

% --------------------------------------
% the top layer
\begin{frame}[fragile]
%
  \frametitle{Filling out the top layer}
%
  \begin{ipython}[basicstyle=\tt\tiny]{}
import pyre, itertools, gsl
from .PointCloud import PointCloud

class GSL(pyre.component, family="gauss.meshes.gsl", implements=PointCloud):
    """
    A point generator implemented using the large set of random number
    generators available as part of the gnu scientific library (GSL)
    """

    # public state
    seed = pyre.properties.float(default=0)
    algorithm = pyre.properties.str(default="ranlxs2")
    
    # interface
    @pyre.export
    def points(self, n, box):
        """
        Generate {n} random points in the interior of {box}
        """
        # unfold the bounding box
        intervals = tuple(box.intervals()) # realize, so we can reuse in the loop
        # loop over the sample size
        while n > 0:
            # make a point and yield it
            yield tuple(itertools.starmap(self.rng.uniform, intervals))
            # update the counter
            n -= 1
        # all done
        return
  \end{ipython}
%
\end{frame}

% --------------------------------------
% the top layer
\begin{frame}[fragile]
%
  \frametitle{Filling out the top layer -- continued}
%
  \begin{ipython}{}
    # meta methods
    def __init__(self, **kwds):
        # chain up
        super().__init__(**kwds)
        # build the RNG
        self.rng = gsl.rng_alloc(self.algorithm)
        # and seed it
        gsl.rng_set(self.rng, int(self.seed))
        # all done
        return

    # private data
    rng = None
  \end{ipython}
%
\end{frame}

% end of file 
