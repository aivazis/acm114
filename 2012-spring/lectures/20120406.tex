% -*- LaTeX -*-
% -*- coding: utf-8 -*-
%
% michael a.g. aïvázis
% california institute of technology
% (c) 1998-2012 all rights reserved
%

\lecture{Introduction to python}{20120406}

% --------------------------------------
% programming paradigms
\begin{frame}[fragile]
%
  \frametitle{Languages and programming paradigms}
%
  \begin{itemize}
%
  \item a very active area of research
    \begin{itemize}
    \item dozens of languages and runtime environments of the last 50 years
    \end{itemize}
%
  \item the survivors:
    \begin{itemize}
      \item procedural programming, and its offspring structured programming
      \item functional programming
      \item object oriented programming
    \end{itemize}
%
  \item current areas of research:
    \begin{itemize}
    \item component oriented programming
    \item aspect programming
    \end{itemize}
%
  \item languages are important:
    \begin{itemize}
    \item they reflect an approach to computing
    \item they shape what is easily expressible
    \end{itemize}
%
  \item we'll take a quick tour of python
    \begin{itemize}
    \item resources: \url{www.python.org}
    \item overview of the language
    \item interactive sessions with the interpreter
    \item building extensions in \cc/\cpp
    \end{itemize}
%
  \end{itemize}
%
\end{frame}

% --------------------------------------
% preview: a python script
\begin{frame}[fragile]
%
  \frametitle{A python script}
%
  \begin{itemize}
  \item python reads like pseudocode
  \item here is the code for the $\pi$ estimator using Monte Carlo integration over the quarter
    disk
  \end{itemize}
%
\python{
  firstline=9,lastline=25,
  label={lst:python:pi},
  caption={\srcfile{pi.py}: Estimating $\pi$ in python},
}{listings/pi.py}

%
\end{frame}

% --------------------------------------
% overview
\begin{frame}[fragile]
%
  \frametitle{Overview}
%
  \begin{itemize}
%
  \item built-in objects and their operators
    \begin{itemize}
    \item numbers, strings, containers
    \item files
    \end{itemize}
%
  \item statements
    \begin{itemize}
      \item evaluating expressions, explicit and implicit assignments, logic, iteration
    \end{itemize}
%
  \item functions
    \begin{itemize}
    \item scope rules, argument passing, callable objects
    \end{itemize}
%
  \item modules and packages
    \begin{itemize}
    \item name qualification, importing symbols
    \end{itemize}
%
  \item user defined objects
    \begin{itemize}
    \item declarations and definitions, inheritance, overloading operators
    \end{itemize}
%
  \item exceptions
    \begin{itemize}
    \item raising and catching, exception hierarchies
    \end{itemize}
%
  \end{itemize}
%
\end{frame}

% --------------------------------------
% syntax
\begin{frame}[fragile]
%
  \frametitle{Syntax}
%
  \begin{itemize}
%
  \item comments: from a \literal{\#} to the end of the line
%
  \item indentation denotes scope
    \begin{itemize}
    \item avoid using tab characters; set your editor to insert a fixed number of spaces when
      the tab key is pressed
    \end{itemize}
%
  \item statements end at the end of the line, or at \literal{;}
    \begin{itemize}
    \item open delimiters imply continuation
      \item explicit continuation with \literal{$\backslash$}, but considered obsolete
    \end{itemize}
%
  \item identifiers
    \begin{itemize}
    \item start with an underscore or letter, followed by underscores, letters or digits
    \item unicode is supported in identifier names; details at
      {\scriptsize \url{http://docs.python.org/py3k/reference/lexical_analysis.html#literals}}
    \item identifiers are case sensitive
    \end{itemize}
%
  \item certain classes of identifiers have special meaning
    \begin{itemize}
    \item the pattern \literal{\_\_*\_\_} is reserved by python for its own use
    \item identifiers of the form \literal{\_\_*} in class definition are mangled and become
      private
    \item identifiers of the form \literal{\_*} are not bulk imported from modules; more on
      this later
    \end{itemize}
%
  \end{itemize}
%
\end{frame}

% --------------------------------------
% reserved words
\begin{frame}[fragile]
%
  \frametitle{Reserved words}
%
  \begin{itemize}
%
  \item the following words are reserved
    \begin{table}
      \begin{tabular}{lllll}
%
 \keyword{False} & \keyword{None} & \keyword{True} & \operator{and} & \keyword{as} \\
 \keyword{assert} & \keyword{break} & \keyword{class} & \keyword{continue} & \keyword{def} \\ 
 \keyword{del} & \keyword{elif} & \keyword{else} & \keyword{except} & \keyword{finally} \\ 
 \keyword{for} & \keyword{from} & \keyword{global} & \keyword{if} & \keyword{import} \\
 \operator{in} & \operator{is} & \keyword{lambda} & \keyword{nonlocal} & \operator{not} \\
 \operator{or} & \keyword{pass} & \keyword{raise} & \keyword{return} & \keyword{try} \\ 
 \keyword{while} & \keyword{with} & \keyword{yield} & &
%
      \end{tabular}
    \end{table}
%
  \end{itemize}
%
\end{frame}

% --------------------------------------
% built-in objects
\begin{frame}[fragile]
%
  \frametitle{Built-in objects}
%
  \begin{itemize}
  \item the more commonly used types
  \end{itemize}
%
  \begin{table}\footnotesize
    \begin{tabular}{ll}
      \emph{Type}    & \emph{Sample} \\ \midrule
      booleans       & \keyword{True}, \keyword{False} \\
      numbers        & \literal{1234}, \literal{3.14159}, \literal{3+4j} \\
      strings        & \literal{'help'}, \literal{"hello"}, \literal{"it's mine"},
                       \literal{"""multi-line strings"""} \\
      tuples         & \literal{(1, 'this', "other")} \\
      lists          & \literal{['this', ['and', 0], 2]} \\
      sets           & \literal{\{1,2,3\}} \\
      dictionaries   & \literal{\{'first': 'Jim', 'last': 'Brown'\}}
    \end{tabular}
  \end{table}
%
  \begin{itemize}
  \item there are others; details to follow, as necessary
  \end{itemize}
%
\end{frame}

% --------------------------------------
% operators and precedence
\begin{frame}[fragile]
%
  \frametitle{Operators and precedence}
%
  \begin{itemize}
  \item from lower to higher precendece
  \end{itemize}
%
  \begin{table}\footnotesize
    \begin{tabular}{ll}
      \emph{Operator} & \emph{Description} \\ \midrule
      \operator{lambda} & used to build anonymous functions \\
      \keyword{if} -- \keyword{else} & conditional expression (similar to \literal{?:} from
                                       \cc) \\
      \operator{or} & boolean or \\
      \operator{and} & boolean and \\
      \operator{not} & boolean not \\
      \operator{in}, \operator{not in}, \operator{is}, \operator{is not} &
      \multirow{2}{*}{membership tests, identity tests, comparisons} \\
      \operator{<}, \operator{<=}, \operator{>}, \operator{>=}, \operator{!=}, \operator{==} & \\
      \operator{|} & bitwise or \\
      \operator{\^{}} & bitwise xor \\
      \operator{\&} & bitwise and \\
      \operator{<<}, \operator{>>} & left and right bit shifts \\
      \operator{+}, \operator{-} & binary addition, binary subtraction \\
      \operator{*}, \operator{/}, \operator{//}, \operator{\%} &
      multiplication, division, integer division, modulo \\
      \operator{+}, \operator{-}, \operator{\~{}} & positive, negative, bitwise not \\
      \operator{**} & exponentiation \\
      \literal{[]}, \literal{[:]}, \literal{()}, \literal{.} &
      indexing, slicing, function call, attribute reference 
    \end{tabular}
  \end{table}
%
\end{frame}

% --------------------------------------
% number
\begin{frame}[fragile]
%
  \frametitle{Numbers}
%
  \begin{itemize}
%
  \item numeric literals
    \begin{table}\footnotesize
      \begin{tabular}{ll}
        \emph{Literal} & \emph{Description} \\ \midrule
        \literal{1234} & arbitrary precision integers \\
        \literal{3.1415}, \literal{6.023e23} & floats \\
        \literal{j}, \literal{1+j} & complex numbers \\
        \literal{0b1001} & binary integers \\
        \literal{0o777} & octal integers \\
        \literal{0xdeadbeef} & hexadecimal integers
      \end{tabular}
    \end{table}
%
  \item expressions:
    \begin{itemize}
    \item the usual arithmetic operators
    \item bitwise operators similar to \cc
    \item adjust precedence and association by using parenthesis; be aware of the ``tuple
      conflict''
    \item in expressions with mixed types, python converts towards the wider types
    \end{itemize}
%
  \end{itemize}
%
\end{frame}

% end of file 
