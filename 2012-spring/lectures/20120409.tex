% -*- LaTeX -*-
% -*- coding: utf-8 -*-
%
% michael a.g. aïvázis
% california institute of technology
% (c) 1998-2012 all rights reserved
%

lecture{Introduction to python}{20120409}

% --------------------------------------
% strings
\begin{frame}[fragile]
%
  \frametitle{Strings}
%
  \begin{itemize}
%
  \item string literals
    \begin{table}\footnotesize
      \begin{tabular}{ll}
        \emph{Literal} & \emph{Description} \\ \midrule
        \literal{'Hello'} & single quotes \\
        \literal{"Hello"}, & double quotes \\
        \literal{"It's embedded"} & mix and match \\
        \literal{'''Hello'''}, \literal{"""Hello"""} & triple quotes -- extend over multiple
        lines \\
      \end{tabular}
    \end{table}
%
  \item strings are immutable ordered sequences of unicode characters
    \begin{itemize}
    \item no \literal{char} type: \literal{'a'} is a string with one character
    \item native support for all code pages
  \end{itemize}
%
  \item common operations: given the strings $s_{i}$:
    \begin{table}\footnotesize
      \begin{tabular}{ll}
        \emph{Expression} & \emph{Description} \\ \midrule
        \literal{s1+s2}, \literal{s*4} & concatenation, repetition \\
        \literal{s[3]}, \literal{s[3:4]} & indexing, slicing \\
        \literal{len(s)} & length\\
        \literal{'H' in s} & membership\\
        \literal{for x in s} & iteration
      \end{tabular}
    \end{table}
%
  \end{itemize}
%
\end{frame}

% --------------------------------------
% string coercions
\begin{frame}[fragile]
%
  \frametitle{String coercions}
%
  \begin{itemize}
%
  \item all objects are representable as strings
  \item coercion can be triggered explicitly
    \begin{itemize}
    \item using the \keyword{str} constructor
    \item using the \keyword{repr} built-in function
    \end{itemize}
  \item strings have a powerful formatting method:
    \begin{ipython}{}
      'temperature={:+13.4f}, pressure={:13.4f}'.format(T, P)
    \end{ipython}
  \item more details after we go over the rules for function calls
%
  \end{itemize}
%
\end{frame}

% --------------------------------------
% tuples
\begin{frame}[fragile]
%
  \frametitle{Tuples}
%
  \begin{itemize}
%
  \item tuples are immutable ordered inhomogeneous sequences of objects
    \begin{table}\footnotesize
      \begin{tabular}{ll}
        \emph{Literal} & \emph{Description} \\ \midrule
        \literal{()} & the empty tuple \\
        \literal{(1,)}, & a tuple with one item \\
        \literal{(1,2,3,4)} & a longer tuple \\
        \literal{(1,'Hello', 'world')} & tuple elements don't have to be the same type\\
        \literal{(1,2,('Hello', 'world'),4)} & tuples nest arbitrarily deeply
      \end{tabular}
    \end{table}
%
  \item common operations:
    \begin{table}\footnotesize
      \begin{tabular}{ll}
        \emph{Expression} & \emph{Description} \\ \midrule
        \literal{t1+t2}, \literal{t*4} & concatenation, repetition \\
        \literal{t[3]}, \literal{t[3:4]} & indexing, slicing \\
        \literal{len(t)} & length\\
        \literal{x in t} & membership\\
        \literal{for x in t} & iteration
      \end{tabular}
    \end{table}
%
  \end{itemize}
%
\end{frame}

% --------------------------------------
% lists
\begin{frame}[fragile]
%
  \frametitle{Lists}
%
  \begin{itemize}
%
  \item lists are mutable ordered inhomogeneous sequences of objects
    \begin{table}\footnotesize
      \begin{tabular}{ll}
        \emph{Literal} & \emph{Description} \\ \midrule
        \literal{[]} & the empty list \\
        \literal{[1])}, & a list with one item \\
        \literal{[1,2,3,4]} & a longer list \\
        \literal{[1,'Hello', 'world']} & list elements don't have to be the same type\\
        \literal{[1,2,['Hello', 'world'],4]} & lists nest arbitrarily deeply
      \end{tabular}
    \end{table}
%
  \item common operations:
    \begin{table}\footnotesize
      \begin{tabular}{ll}
        \emph{Expression} & \emph{Description} \\ \midrule
        \literal{l1+l2}, \literal{l*4} & concatenation, repetition \\
        \literal{len(l)}, \literal{x in l}, \literal{for x in l} & length, membership, iteration \\
        \literal{l[3]}, \literal{l[3:4]} & indexing, slicing \\
        \literal{del l[3]}, \literal{l[3:7]=[]} & shrink \\
        \literal{l[3:7]=[1,2]} & slice assignment \\
        \literal{l.append(1)} & add to the end of the list
      \end{tabular}
    \end{table}
%
  \end{itemize}
%
\end{frame}

% --------------------------------------
% sets
\begin{frame}[fragile]
%
  \frametitle{Sets}
%
  \begin{itemize}
%
  \item sets are mutable unordered inhomogeneous containers of objects
    \begin{table}\footnotesize
      \begin{tabular}{ll}
        \emph{Literal} & \emph{Description} \\ \midrule
        \literal{set()} & the empty set \\
        \literal{\{1\})}, & a set with one item \\
        \literal{\{1,2,3,4\}} & a longer set \\
        \literal{\{1,'Hello', 'world'\}} & set elements don't have to be the same type
      \end{tabular}
    \end{table}
%
  \item common operations:
    \begin{table}\footnotesize
      \begin{tabular}{ll}
        \emph{Expression} & \emph{Description} \\ \midrule
        \literal{s1|s2} & union \\
        \literal{s1\&s2} & intersection \\
        \literal{s1-s2}, \literal{s1\^{}s2} & difference, symmetric difference \\
        \literal{len(s)}, \literal{x in s}, \literal{for x in s} & length, membership, iteration \\
        \literal{s.add(x)} & add \\
        \literal{s.discard(x)} & remove if present\\
        \literal{s.remove(x)} & remove; raise \literal{KeyError} if not present
      \end{tabular}
    \end{table}
%
  \end{itemize}
%
\end{frame}

% --------------------------------------
% dictionaries
\begin{frame}[fragile]
%
  \frametitle{Dictionaries}
%
  \begin{itemize}
%
  \item dictionaries are mutable inhomogeneous associative maps
    \begin{table}\footnotesize
      \begin{tabular}{ll}
        \emph{Literal} & \emph{Description} \\ \midrule
        \literal{\{\}} & the empty dictionary \\
        \literal{\{'first':'Guido', 'last':'van Rossum'\}}, & a dictionary with two items \\
      \end{tabular}
    \end{table}
%
  \item dictionary keys must be \emph{hashable}; dictionary values may be of any type
%
  \item common operations
    \begin{table}\footnotesize
      \begin{tabular}{ll}
        \emph{Expression} & \emph{Description} \\ \midrule
        \literal{d['first']='Guido'} & make an association \\
        \multirow{2}{*}{\literal{d['first']}} & value retrieval \\
        & raise \literal{KeyError} if key not present \\
        \literal{d.get('first')}, \literal{d.get('first', default='')} & value retrieval \\
        \literal{len(d)} & the number of keys \\
        \literal{x in d} & key presence \\
        \literal{d.keys()}, \literal{d.values()}, \literal{d.items()} &
        views of the dictionary contents
      \end{tabular}
    \end{table}
%
  \end{itemize}
%
\end{frame}

% --------------------------------------
% comparisons
\begin{frame}[fragile]
%
  \frametitle{Comparisons}
%
  \begin{itemize}
%
  \item the following are considered false in logical expressions
    \begin{itemize}
    \item the boolean constant \keyword{False}
    \item the special object \keyword{None}
    \item the number \literal{0}
    \item any empty container
    \end{itemize}
%
  \item other values are true, including the boolean constant \keyword{True}
%
  \item operators:
    \begin{itemize}
    \item the object identity operator \operator{is}
    \item the container membership operator \operator{is}
    \item the usual relational operators -- borrowed from \cc
    \end{itemize}
%
  \item object comparisons
    \begin{itemize}
    \item strings are compared lexicographically
    \item nested data structured are checked recursively
    \item lists and tuples are compared depth first, left to right
    \item dictionaries are compared as sorted (key, value) tuples
    \item user defined types can supply custom comparison functions by overloading the special
      methods
    \end{itemize}
%
  \end{itemize}
%
\end{frame}

% --------------------------------------
% assignments
\begin{frame}[fragile]
%
  \frametitle{Assignments}
%
  \begin{itemize}
%
  \item explicitly, using the \operator{=} operator
    \begin{ipython}{}
      greeting = 'Hello world!'
    \end{ipython}
    which makes the symbol \literal{greeting} become a name for the literal string in the right
    hand side
%
  \item implicitly, when defining a function
    \begin{ipython}{}
      def greeting(name): pass
    \end{ipython}
    which makes \literal{greeting} become a name for the function object built out of the
    statements that follow the \literal{:}
%
  \item implicitly, when defining a class
    \begin{ipython}{}
      class greeting: pass
    \end{ipython}
    which makes \literal{greeting} become a name for the \emph{class object} built out of the
    statements that follow the \literal{:}
%
  \item implicitly, when importing symbols from a module
    \begin{ipython}{}
      import sys
      from math import pi
      from math import pi as #@$\pi$@ 
    \end{ipython}
%
  \end{itemize}
%
\end{frame}

% --------------------------------------
% selections
\begin{frame}[fragile]
%
  \frametitle{Selections}
%
  \begin{itemize}
%
  \item using \keyword{if}
    \begin{ipython}{}
      if <expression>:
          <statements>
      elif <expression>:
          <statements>
      else:
          <statements>
    \end{ipython}
%
  \item no \keyword{switch} statement
    \begin{itemize}
    \item use an \keyword{if} cascade
    \item better yet, think about achieving the same result using containers; it's typically
      more efficient and robust
    \end{itemize}
%
  \end{itemize}
%
\end{frame}

% --------------------------------------
% iteration
\begin{frame}[fragile]
%
  \frametitle{Iteration}
%
  \begin{itemize}
%
  \item the \keyword{while} loop
    \begin{ipython}{}
      while <expression>:
          <statements>
      else:
          <statements>
    \end{ipython}
    executes the statements in its body until its expression evaluates to \keyword{False}, at
    which point it executes the optional \keyword{else} clause
%
  \item the \keyword{for} loop 
    \begin{ipython}{}
      for <name> in <expression>:
          <statements>
      else:
          <statements>
    \end{ipython}
    evaluates its expression once to get an iterator, binds \literal{name} to each object
    provided by the iterator and executes its body; iteration stops when the iterator is
    exhausted, at which point the \keyword{else} clause is executed; if execution encounters a
    \keyword{break} statement in the body of the loop, iteration is terminated, and the
    \keyword{else} clause is skipped; if execution encounters a \keyword{continue} statement,
    it skips the remainder of the loop body and proceeds with the next item from the iterator
%
  \end{itemize}
%
\end{frame}

% --------------------------------------
% functions
\begin{frame}[fragile]
%
  \frametitle{Functions}
%
  \begin{itemize}
%
  \item general form
    \begin{ipython}{}
      def <name>(<parameter_list>):
          <statements>
          return <expression>
    \end{ipython}
%
  \item creates a function object and assigns it to the given name
    \begin{itemize}
    \item the optional \keyword{return} statement sends an object back to the caller
    \item arguments are passed by \emph{assignment}
    \item no declarations of arguments, return types and local variables
    \end{itemize}
%
  \item a simple example
    \begin{ipython}{}
      def greet(friend):
          print('Hello {}!'.format(friend))
          return
    \end{ipython}
%
  \item it is invoked using a \emph{call expression}
    \begin{ipython}{}
      greet(friend='world')
    \end{ipython}
%
  \end{itemize}
%
\end{frame}

% --------------------------------------
% scoping rules
\begin{frame}[fragile]
%
  \frametitle{Scope and visibility}
%
  \begin{itemize}
%
  \item the enclosing module acts as the global scope
  \item each call to the function creates a new local scope
  \item all assignments in the function body are local by default
    \begin{itemize}
    \item you can override using the \keyword{global} statement
    \end{itemize}
  \item all other names should either be global or built-in
%
    \begin{ipython}{}
      root = 12

      def isServer(pid):
          if pid == root: return True
          return False

      def setServer(pid):
          global root
          root = pid
          return
    \end{ipython}
  \end{itemize}
%
\end{frame}

% --------------------------------------
% function arguments
\begin{frame}[fragile]
%
  \frametitle{Function arguments}
%
  \begin{itemize}
%
  \item argument passing rules:
    \begin{itemize}
    \item arguments are passed by creating a local reference to an existing object
    \item re-assigning the local variable does not affect the caller
    \item but modifying a mutable object through the reference impacts the caller
    \end{itemize}
%
  \item argument matching modes:
    \begin{itemize}
    \item by position
    \item by keyword
    \item using varargs:
      \begin{itemize}
      \item \keyword{*}: places non-keyword arguments in a tuple
      \item \keyword{**}: places keyword arguments in a dictionary
      \end{itemize}
    \item using default values supplied in the function declaration
    \end{itemize}
%
  \item ordering rules:
    \begin{itemize}
    \item declaration: normal, *arguments, **arguments
    \item caller: non-keyword arguments first, then keyword arguments
    \end{itemize}
%
    \item matching algorithm
      \begin{itemize}
      \item assign non-keyword arguments by position
      \item assign keyword arguments by matching names
      \item assign left over non-keyword arguments to \literal{*args}
      \item assign extra keyword arguments to \literal{**kwds}
      \item unassigned arguments in declaration get their default values
      \end{itemize}
      
  \end{itemize}
%
\end{frame}

% --------------------------------------
% function examples
\begin{frame}[fragile]
%
  \frametitle{Simple examples}
%
  \begin{itemize}
%
  \item here are some examples of function declarations and the possible ways to invoke them --
    don't forget that python uses ``pass by assignment''
%
  \item the simplest, but not the best, is positional invocation:
    \begin{ipython}{}
      # declare a function
      def greet(friend): print('Hello {}!'.format(friend))

      # invoke it
      greet('world')
    \end{ipython}
%
  \item a better way is to use the name of the dummy variable, as was done in our first example
    \begin{ipython}{}
      # declare a function
      def greet(friend): print('Hello {}!'.format(friend))

      # invoke it by explicitly binding the dummy variable to a value
      greet(friend='world')
    \end{ipython}
%
  \item it may look silly with one argument, but this technique eliminates more bugs than
    the strong type checking in languages like \cpp!
    \begin{itemize}
    \item (long rant removed by the editor...)
    \end{itemize}
%
  \end{itemize}
%
\end{frame}

% --------------------------------------
% function examples
\begin{frame}[fragile]
%
  \frametitle{Default arguments}
%
  \begin{itemize}
%
  \item a function declaration can provide default values for arguments that were not provided by
    the caller; consider the following declaration:
    \begin{ipython}{}
      # declare a function
      def say(what='Hello', whom='world'):
          print('{} {}!'.format(what, whom))
          return
    \end{ipython}
%
    \item this function can be invoked as before
    \begin{ipython}{}
      # invoke with a full argument set -- note the twist
      say(whom='cruel world', what='Goodbye')
    \end{ipython}
%
  \item either one of the arguments can be absent; it will be bound to the default value
    \begin{ipython}{}
      # change the target
      say(whom='class')
    \end{ipython}
    or
    \begin{ipython}{}
      # change the message
      say(what='Greetings')
    \end{ipython}
%
  \end{itemize}
%
\end{frame}

% --------------------------------------
% trickiness with default arguments
\begin{frame}[fragile]
%
  \frametitle{A potential pitfall}
%
  \begin{itemize}
%
  \item there is some trickiness to default arguments, having to do with how this feature is
    currently implemented by the interpreter
    \begin{itemize}
    \item the default values are treated as expressions that are evaluated during the
        function declaration, and stored along with the function itself
    \end{itemize}
%
  \item consider the following function:
    \begin{ipython}{}
      # the mail routine
      def mail(item, recipients=[]):
          # for each recipient
          for recipient in recipients:
              # mail the item
              print('mailing {} to {}'.format(item, recipient))
          # all done; return the recipient list back to the caller
          return recipients
    \end{ipython}
%
  \item what happens when the return value is modified?
    \begin{ipython}{}
      # send a letter to the default set of recipients
      friends = mail(item='a letter')
      # add a couple of people to the list
      friends += [ 'Alec', 'MacKenzie' ]
      # send a postcard to the default set of recipients
      mail(item='a postcard')
    \end{ipython}
    unlike what you might expect, the second invocation sends postcards to Alec and MacKenzie
%
  \end{itemize}
%
\end{frame}

% --------------------------------------
% varags
\begin{frame}[fragile]
%
  \frametitle{Variable number of arguments}
%
  \begin{itemize}
%
  \item occasionally -- but rarely -- there are legitimate reasons for a function to accept an
    unknown number of arguments
    \begin{itemize}
    \item but please consider alternatives before resorting to this
    \item we'll see a case where it is necessary when we discuss multiple inheritance
    \end{itemize}
%
  \item consider:
    \begin{ipython}{}
      # the mail routine
      def mail(item, *recipients):
          # for each recipient
          for recipient in recipients:
              # mail the item
              print('mailing {} to {}'.format(item, recipient))
          # all done
          return
    \end{ipython}
    which can be invoked as
    \begin{ipython}{}
      mail('letter', 'Alec', 'MacKenzie')
    \end{ipython}
    the variable \literal{recipients} gets bound to a tuple of all the arguments that follow
    \literal{'letter'}
%
  \end{itemize}
%
\end{frame}

% --------------------------------------
% varags
\begin{frame}[fragile]
%
  \frametitle{Variable number of arguments}
%
  \begin{itemize}
%
  \item it is now illegal to use the form \literal{item='letter'} in the function
    call
%
  \item instead you gain the ability to do this:
    \begin{ipython}{}
      friends = ['Alec', 'MacKenzie']
      mail('letter', *friends)
    \end{ipython}
%
  \item similar considerations apply to
    \begin{ipython}{}
      # the mail routine
      def configure(item, **options):
          print('configuring {!r}'.format(item))
          # for each option
          for option, value in options.items():
              # print the option
              print('  {} <- {}'.format(option, value))
          # all done
          return
    \end{ipython}
    which can be invoked using
    \begin{ipython}{}
      # set some options
      options = {
          'paper': 'A4',
          'orientation': 'landscape'
          }
      # configure the printer
      configure('printer', **options)
    \end{ipython}
%
  \end{itemize}
%
\end{frame}

% --------------------------------------
% function objects
\begin{frame}[fragile]
%
  \frametitle{Functions as objects}
%
  \begin{itemize}
%
  \item like everything else, functions are objects
    \begin{ipython}{}
      def hello():
          """say hello"""
          return "Hello"
      
      def goodbye():
          """say goodbye"""
          return "Goodbye"

      def greet(how=nice, whom='world'):
          """call {how} to compute what to say to {whom}"""
          print('{} {}!',format(how(), whom))
          return
          
      greet(how=goodbye, whom='class')
    \end{ipython}
%
  \item the name of a function is a reference to the callable object that \keyword{def} left
    behind
    \begin{itemize}
    \item you can use it anywhere a variable would go
    \item to invoke the function,  you must involve the reference in a call expression
    \end{itemize}
    \begin{ipython}{}
      greeter = nice
      message = greeter()
    \end{ipython}
%
  \end{itemize}
%
\end{frame}

% --------------------------------------
% printing
\begin{frame}[fragile]
%
  \frametitle{Printing}
%
  \begin{itemize}
%
  \item 
    \begin{itemize}
    \item
    \end{itemize}
%
  \end{itemize}
%
\end{frame}

% --------------------------------------
% namespaces
\begin{frame}[fragile]
%
  \frametitle{Namespaces}
%
  \begin{itemize}
%
  \item 
    \begin{itemize}
    \item
    \end{itemize}
%
  \end{itemize}
%
\end{frame}

% --------------------------------------
% namespace access
\begin{frame}[fragile]
%
  \frametitle{Namespace access}
%
  \begin{itemize}
%
  \item 
    \begin{itemize}
    \item
    \end{itemize}
%
  \end{itemize}
%
\end{frame}

% --------------------------------------
% namespaces as objects
\begin{frame}[fragile]
%
  \frametitle{Namespaces as objects}
%
  \begin{itemize}
%
  \item 
    \begin{itemize}
    \item
    \end{itemize}
%
  \end{itemize}
%
\end{frame}

% --------------------------------------
% other built-in objects
\begin{frame}[fragile]
%
  \frametitle{Other built-in objects}
%
  \begin{itemize}
%
  \item 
    \begin{itemize}
    \item
    \end{itemize}
%
  \end{itemize}
%
\end{frame}

% --------------------------------------
% type hierarchy
\begin{frame}[fragile]
%
  \frametitle{Type hierarchy}
%
  \begin{itemize}
%
  \item 
    \begin{itemize}
    \item
    \end{itemize}
%
  \end{itemize}
%
\end{frame}

% --------------------------------------
% classes
\begin{frame}[fragile]
%
  \frametitle{Classes}
%
  \begin{itemize}
%
  \item 
    \begin{itemize}
    \item
    \end{itemize}
%
  \end{itemize}
%
\end{frame}

% --------------------------------------
% inheritance
\begin{frame}[fragile]
%
  \frametitle{Inheritance}
%
  \begin{itemize}
%
  \item 
    \begin{itemize}
    \item
    \end{itemize}
%
  \end{itemize}
%
\end{frame}

% --------------------------------------
% methods
\begin{frame}[fragile]
%
  \frametitle{Methods}
%
  \begin{itemize}
%
  \item 
    \begin{itemize}
    \item
    \end{itemize}
%
  \end{itemize}
%
\end{frame}

% --------------------------------------
% class glossary
\begin{frame}[fragile]
%
  \frametitle{Class glossary}
%
  \begin{itemize}
%
  \item 
    \begin{itemize}
    \item
    \end{itemize}
%
  \end{itemize}
%
\end{frame}

% --------------------------------------
% special methods
\begin{frame}[fragile]
%
  \frametitle{Special methods}
%
  \begin{itemize}
%
  \item 
    \begin{itemize}
    \item
    \end{itemize}
%
  \end{itemize}
%
\end{frame}

% --------------------------------------
% namespace rules
\begin{frame}[fragile]
%
  \frametitle{Blank}
%
  \begin{itemize}
%
  \item 
    \begin{itemize}
    \item
    \end{itemize}
%
  \end{itemize}
%
\end{frame}

% --------------------------------------
% classes as objects
\begin{frame}[fragile]
%
  \frametitle{Classes as objects}
%
  \begin{itemize}
%
  \item 
    \begin{itemize}
    \item
    \end{itemize}
%
  \end{itemize}
%
\end{frame}

% --------------------------------------
% methods as objects
\begin{frame}[fragile]
%
  \frametitle{Methods as objects}
%
  \begin{itemize}
%
  \item 
    \begin{itemize}
    \item
    \end{itemize}
%
  \end{itemize}
%
\end{frame}

% --------------------------------------
% exceptions
\begin{frame}[fragile]
%
  \frametitle{Exceptions}
%
  \begin{itemize}
%
  \item 
    \begin{itemize}
    \item
    \end{itemize}
%
  \end{itemize}
%
\end{frame}

% --------------------------------------
% raising exceptions
\begin{frame}[fragile]
%
  \frametitle{Raising exceptions}
%
  \begin{itemize}
%
  \item 
    \begin{itemize}
    \item
    \end{itemize}
%
  \end{itemize}
%
\end{frame}

% --------------------------------------
% catching exceptions
\begin{frame}[fragile]
%
  \frametitle{Catching exceptions}
%
  \begin{itemize}
%
  \item 
    \begin{itemize}
    \item
    \end{itemize}
%
  \end{itemize}
%
\end{frame}

% --------------------------------------
% object oriented programming
\begin{frame}[fragile]
%
  \frametitle{Object oriented programming}
%
  \begin{itemize}
%
  \item 
    \begin{itemize}
    \item
    \end{itemize}
%
  \end{itemize}
%
\end{frame}

% end of file 
