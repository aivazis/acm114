% -*- LaTeX -*-
% -*- coding: utf-8 -*-
%
% michael a.g. aïvázis
% california institute of technology
% (c) 1998-2012 all rights reserved
%

\lecture{Introduction to python}{20120409}

% --------------------------------------
% strings
\begin{frame}[fragile]
%
  \frametitle{Strings}
%
  \begin{itemize}
%
  \item string literals
    \begin{table}\footnotesize
      \begin{tabular}{ll}
        \emph{Literal} & \emph{Description} \\ \midrule
        \literal{'Hello'} & single quotes \\
        \literal{"Hello"}, & double quotes \\
        \literal{"It's embedded"} & mix and match \\
        \literal{'''Hello'''}, \literal{"""Hello"""} & triple quotes -- extend over multiple
        lines \\
      \end{tabular}
    \end{table}
%
  \item strings are immutable ordered sequences of unicode characters
    \begin{itemize}
    \item no \literal{char} type: \literal{'a'} is a string with one character
    \item native support for all code pages
  \end{itemize}
%
  \item common operations: given the strings $s_{i}$:
    \begin{table}\footnotesize
      \begin{tabular}{ll}
        \emph{Expression} & \emph{Description} \\ \midrule
        \literal{s1+s2}, \literal{s*4} & concatenation, repetition \\
        \literal{s[3]}, \literal{s[3:4]} & indexing, slicing \\
        \literal{len(s)} & length\\
        \literal{'H' in s} & membership\\
        \literal{for x in s} & iteration
      \end{tabular}
    \end{table}
%
  \end{itemize}
%
\end{frame}

% --------------------------------------
% string coercions
\begin{frame}[fragile]
%
  \frametitle{String coercions}
%
  \begin{itemize}
%
  \item all objects are representable as strings
  \item coercion can be triggered explicitly
    \begin{itemize}
    \item using the \keyword{str} constructor
    \item using the \keyword{repr} built-in function
    \end{itemize}
  \item strings have a powerful formatting method:
    \begin{ipython}{}
      'temperature={:+13.4f}, pressure={:13.4f}'.format(T, P)
    \end{ipython}
  \item more details after we go over the rules for function calls
%
  \end{itemize}
%
\end{frame}

% --------------------------------------
% tuples
\begin{frame}[fragile]
%
  \frametitle{Tuples}
%
  \begin{itemize}
%
  \item tuples are immutable ordered inhomogeneous sequences of objects
    \begin{table}\footnotesize
      \begin{tabular}{ll}
        \emph{Literal} & \emph{Description} \\ \midrule
        \literal{()} & the empty tuple \\
        \literal{(1,)} & a tuple with one item \\
        \literal{(1,2,3,4)} & a longer tuple \\
        \literal{(1,'Hello', 'world')} & tuple elements don't have to be the same type\\
        \literal{(1,2,('Hello', 'world'),4)} & tuples nest arbitrarily deeply
      \end{tabular}
    \end{table}
%
  \item common operations:
    \begin{table}\footnotesize
      \begin{tabular}{ll}
        \emph{Expression} & \emph{Description} \\ \midrule
        \literal{t1+t2}, \literal{t*4} & concatenation, repetition \\
        \literal{t[3]}, \literal{t[3:4]} & indexing, slicing \\
        \literal{len(t)} & length\\
        \literal{x in t} & membership\\
        \literal{for x in t} & iteration
      \end{tabular}
    \end{table}
%
  \end{itemize}
%
\end{frame}

% --------------------------------------
% lists
\begin{frame}[fragile]
%
  \frametitle{Lists}
%
  \begin{itemize}
%
  \item lists are mutable ordered inhomogeneous sequences of objects
    \begin{table}\footnotesize
      \begin{tabular}{ll}
        \emph{Literal} & \emph{Description} \\ \midrule
        \literal{[]} & the empty list \\
        \literal{[1]} & a list with one item \\
        \literal{[1,2,3,4]} & a longer list \\
        \literal{[1,'Hello', 'world']} & list elements don't have to be the same type\\
        \literal{[1,2,['Hello', 'world'],4]} & lists nest arbitrarily deeply
      \end{tabular}
    \end{table}
%
  \item common operations:
    \begin{table}\footnotesize
      \begin{tabular}{ll}
        \emph{Expression} & \emph{Description} \\ \midrule
        \literal{l1+l2}, \literal{l*4} & concatenation, repetition \\
        \literal{len(l)}, \literal{x in l}, \literal{for x in l} & length, membership, iteration \\
        \literal{l[3]}, \literal{l[3:4]} & indexing, slicing \\
        \literal{del l[3]}, \literal{l[3:7]=[]} & shrink \\
        \literal{l[3:7]=[1,2]} & slice assignment \\
        \literal{l.append(1)} & add to the end of the list
      \end{tabular}
    \end{table}
%
  \end{itemize}
%
\end{frame}

% --------------------------------------
% sets
\begin{frame}[fragile]
%
  \frametitle{Sets}
%
  \begin{itemize}
%
  \item sets are mutable unordered inhomogeneous containers of objects
    \begin{table}\footnotesize
      \begin{tabular}{ll}
        \emph{Literal} & \emph{Description} \\ \midrule
        \literal{set()} & the empty set \\
        \literal{\{1\})} & a set with one item \\
        \literal{\{1,2,3,4\}} & a longer set \\
        \literal{\{1,'Hello', 'world'\}} & set elements don't have to be the same type
      \end{tabular}
    \end{table}
%
  \item common operations:
    \begin{table}\footnotesize
      \begin{tabular}{ll}
        \emph{Expression} & \emph{Description} \\ \midrule
        \literal{s1|s2} & union \\
        \literal{s1\&s2} & intersection \\
        \literal{s1-s2}, \literal{s1\^{}s2} & difference, symmetric difference \\
        \literal{len(s)}, \literal{x in s}, \literal{for x in s} & length, membership, iteration \\
        \literal{s.add(x)} & add \\
        \literal{s.discard(x)} & remove if present\\
        \literal{s.remove(x)} & remove; raise \literal{KeyError} if not present
      \end{tabular}
    \end{table}
%
  \end{itemize}
%
\end{frame}

% --------------------------------------
% dictionaries
\begin{frame}[fragile]
%
  \frametitle{Dictionaries}
%
  \begin{itemize}
%
  \item dictionaries are mutable inhomogeneous associative maps
    \begin{table}\footnotesize
      \begin{tabular}{ll}
        \emph{Literal} & \emph{Description} \\ \midrule
        \literal{\{\}} & the empty dictionary \\
        \literal{\{'first':'Guido', 'last':'van Rossum'\}}, & a dictionary with two items \\
      \end{tabular}
    \end{table}
%
  \item dictionary keys must be \emph{hashable}; dictionary values may be of any type
%
  \item common operations
    \begin{table}\footnotesize
      \begin{tabular}{ll}
        \emph{Expression} & \emph{Description} \\ \midrule
        \literal{d['first']='Guido'} & make an association \\
        \multirow{2}{*}{\literal{d['first']}} & value retrieval \\
        & raise \literal{KeyError} if key not present \\
        \literal{d.get('first')}, \literal{d.get('first', default='')} & value retrieval \\
        \literal{len(d)} & the number of keys \\
        \literal{x in d} & key presence \\
        \literal{d.keys()}, \literal{d.values()}, \literal{d.items()} &
        views of the dictionary contents
      \end{tabular}
    \end{table}
%
  \end{itemize}
%
\end{frame}

% end of file 
