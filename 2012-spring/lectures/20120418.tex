% -*- LaTeX -*-
% -*- coding: utf-8 -*-
%
% michael a.g. aïvázis
% california institute of technology
% (c) 1998-2012 all rights reserved
%

\lecture{Introduction to python}{20120418}

% --------------------------------------
% special methods
\begin{frame}[fragile]
%
  \frametitle{Overloading operators in classes}
%
  \begin{itemize}
%
  \item \emph{Don't!}
  \item most python operations involving instances can be intercepted and customized
  \item through methods that have special names
%
  \end{itemize}
%
  \vspace{.5em}
  \begin{minipage}{.40\linewidth}
    \begin{table}\footnotesize
      \begin{tabular}{ll}
        \emph{method} & \emph{purpose} \\
        \method{\_\_init\_\_} & construction: \literal{x = X()} \\
        \method{\_\_del\_\_} & destruction \\
        \method{\_\_str\_\_} & string coercion: \literal{str(x)} \\
        \method{\_\_repr\_\_} & representation: \literal{repr(x)} \\
        \method{\_\_len\_\_} & size, truth tests: \literal{len(x)} \\
        \method{\_\_cmp\_\_} & comparisons: \literal{cmp(x)}, \literal{x < other} \\
        \method{\_\_call\_\_} & function class: \literal{x()}
      \end{tabular}
    \end{table}
  \end{minipage}
%
  \hspace{.1\linewidth}
%
  \begin{minipage}{.40\linewidth}
    \begin{table}\footnotesize
      \begin{tabular}{ll}
        \emph{method} & \emph{purpose} \\
        \method{\_\_getattr\_\_} & member access: \literal{x.name} \\
        \method{\_\_getitem\_\_} & indexing: \literal{x[5]} \\
        \method{\_\_setitem\_\_} & indexing: \literal{x[5] = 0} \\
        \method{\_\_add\_\_} & addition: \literal{x + other} \\
        \method{\_\_radd\_\_} & addition: \literal{other + x} \\
        \method{\_\_and\_\_} & logic: \literal{x and other} \\
        \method{\_\_or\_\_} & logic: \literal{x or other}
      \end{tabular}
    \end{table}
  \end{minipage}
%
\end{frame}

% --------------------------------------
% namespace rules
\begin{frame}[fragile]
%
  \frametitle{Namespace rules}
%
  \begin{itemize}
%
  \item a more complete story
    \begin{itemize}
    \item unqualified names are looked up in a chain of lexical namespaces
    \item qualified names conduct a search in the indicated namespace
    \item scopes initialize object namespaces: packages/modules, classes, instances
    \end{itemize}
%
  \item unqualified names
    \begin{itemize}
    \item are global on read
    \item are local on write, unless explicitly marked \keyword{global}
    \end{itemize}
%
  \item qualified names, e.g. \literal{instance.name}, are looked up in the indicated namespace
    \begin{itemize}
    \item module and package
    \item instance, then class record, then ancestors as specified in the \identifier{\_\_mro\_\_}
    \end{itemize}
%
  \item namespace dictionaries
    \begin{itemize}
    \item \identifier{\_\_dict\_\_}
    \item name qualification is a dictionary lookup
    \end{itemize}
%
  \end{itemize}
%
\end{frame}

% --------------------------------------
% exceptions
\begin{frame}[fragile]
%
  \frametitle{Exceptions}
%
  \begin{itemize}
%
  \item a non-local, high level control flow device
%
  \item exceptions are used to signal
    \begin{itemize}
    \item critical errors
    \item but also recoverable runtime failures
    \end{itemize}
%
  \item raised by the interpreter whenever an error is detected
    \begin{itemize}
    \item there is an extensive exception class hierarchy that captures and documents a wide
      variety of errors
    \end{itemize}
%
  \item user code can trigger an exception using the \keyword{raise} statement
%
  \item the \literal{try ... except} statement sets up a net for catching exceptions
%
  \item proper exception hierarchies are an extremely important part of application design
%
  \end{itemize}
%
\end{frame}

% --------------------------------------
% raising exceptions
\begin{frame}[fragile]
%
  \frametitle{Raising and catching exceptions}
%
  \begin{itemize}
%
  \item exceptions are triggered by \keyword{raise}
    \begin{ipython}{}
      raise <expression>
    \end{ipython}
    where \literal{<expression>} must evaluate to an instance of a class that derives from
    \identifier{Exception}, the base class of all python exceptions
%
  \item exceptions may \emph{chain} to other exceptions
    \begin{ipython}{}
      raise <expression> from <expression>
    \end{ipython}
    this form is most useful after catching some other exception and you want to preserve the
    original error
%
  \item to catch an exception, you have to set up a \keyword{try} block; the full form is
    \begin{ipython}{}
      try:
          <statements>
      except <expression> as <name>:
          <statements>
      else:
          <statements>
      finally:
          <statements>
    \end{ipython}
    there may be as many \keyword{except} sections as you need; most parts of the statement are
    optional
%
  \end{itemize}
%
\end{frame}

% end of file 
