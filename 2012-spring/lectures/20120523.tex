% -*- LaTeX -*-
% -*- coding: utf-8 -*-
%
% michael a.g. aïvázis
% california institute of technology
% (c) 1998-2012 all rights reserved
%

\lecture{Introduction to \pyre}{20120523}

% --------------------------------------
% extending
\begin{frame}[fragile]
%
  \frametitle{Extensions}
%
  \begin{itemize}
 % 
  \item one of the major strengths of python is the ease with which you can get access to code
    written in low level languages
    \begin{itemize}
    \item for leverage, reuse, performance
    \end{itemize}
    the interpreter itself is written in \cc, so you can get access to code written in any
    language that is link compatible
%
  \item python provides
    \begin{itemize}
    \item a \cc\ library that grants you access to nearly every aspect of the interpreter
    \item a simple set of rules for constructing shared objects that can be imported, just like
      any other python module
    \end{itemize}
%
  \item typically, the solution to this problem has three layers
    \begin{itemize}
    \item the archive and headers of the library you want to expose to python
    \item the interface layer that translates data from python to \cc, which becomes the
      importable shared module
    \item a layer in pure python that provides an object oriented veneer to the library; in
      our case this layer will also double as the component specification
    \end{itemize}
%
  \item if tempted to collapse some of these, \emph{resist}
%
  \end{itemize}
%
\end{frame}

% --------------------------------------
% bottom
\begin{frame}[fragile]
%
  \frametitle{The bottom layer}
%
  we would like use the random number generators in \GSL; to get a feeling about what is
  involved, recall our earlier \cpp\ solution
%
  \C++{firstnumber=9, linerange={9-36}, basicstyle=\tt\tiny}{listings/simple/pi.cc}
%
 we must allocate, use and free a \identifier{gsl\_rng}
%
\end{frame}

% --------------------------------------
% top
\begin{frame}[fragile]
%
  \frametitle{The top layer}
%
  for plug-and-play, we must build a component; perhaps something similar to \identifier{Mersenne}:
%
  \begin{ipython}[basicstyle=\tt\tiny]{}
import pyre, itertools, gsl
from .PointCloud import PointCloud

class GSL(pyre.component, family="gauss.meshes.gsl", implements=PointCloud):
    """
    A point generator implemented using the large set of random number
    generators available as part of the gnu scientific library (GSL)
    """

    # interface
    @pyre.export
    def points(self, n, box):
        """
        Generate {n} random points in the interior of {box}
        """
        # unfold the bounding box
        intervals = tuple(box.intervals()) # realize, so we can reuse in the loop
        # loop over the sample size
        while n > 0:
            # make a point and yield it
            yield tuple(itertools.starmap(self.rng.uniform, intervals))
            # update the counter
            n -= 1
        # all done
        return
  \end{ipython}
%
  where \identifier{self.rng} is a \emph{handle} for the \identifier{gsl\_rng} pointer
%
\end{frame}

% --------------------------------------
% updates to the meshes package
\begin{frame}[fragile]
%
  \frametitle{Adding the new component to our package}
%
  in \srcfile{gauss/meshes/\_\_init\_\_.py}
%
  \begin{ipython}[firstnumber=9]{}
"""
Package that contains the implementations of point clouds
"""

# the interfaces
from .PointCloud import PointCloud as cloud

# the components
from .GSL import GSL as gsl
from .Mersenne import Mersenne as mersenne
  \end{ipython}
%
  the only reason to tuck this in \identifier{gauss.meshes} is the convenience of using the
  shorter name when configuring
%
  \begin{icfg}{}
[ mc ] ; configure our Monte Carlo integrator instance
samples = 10**6
mesh = gsl
region = ball
integrand = constant
  \end{icfg}{}
%
  access to \GSL\ could have been provided by a user of the \package{gauss} package just as
  easily, without any access to our source code
%
\end{frame}

% --------------------------------------
% module access
\begin{frame}[fragile]
%
  \frametitle{Module access}
%
  \begin{itemize}
  \item python extensions live in dynamically loaded libraries
  \begin{itemize}
  \item \identifier{DLL}s on windows, shared objects on unix
  \item building one is platform dependent; compilers have the right command line flags
  \end{itemize}
%
  \item making an extension requires a \emph{visible entry point} in a shared object, both of whose
    names python can infer correctly so that the statement
    \begin{ipython}{}
    import gsl
    \end{ipython}
    can be converted into a search for a particular file, followed by a lookup of a particular
    symbol
%
  \item e.g.~on unix, acceptable names for the \identifier{gsl} extension module are
    \srcfile{gsl.so} or \srcfile{gslmodule.so}
%
  \item looking for extensions is just another step in the sequence applied to the folders on
    \identifier{sys.path}, so the same rules apply as looking for regular modules
%
  \end{itemize}
%
\end{frame}

% --------------------------------------
% module initialization
\begin{frame}[fragile]
%
  \frametitle{The visible entry point}
%
  and here is the definition of the entry point
%
  \begin{iC++}{}
// initialization function for the gsl module
// *must* be called PyInit_gsl
PyMODINIT_FUNC PyInit_gsl()
{
    // create the module
    PyObject * module = PyModule_Create(&gsl::module_definition);
    // check whether module creation succeeded
    if (!module) { 
        // otherwise, raise an ImportError
        return 0; 
    }
    // we have an initialized module
    // set the error handler
    gsl_set_error_handler(&errorHandler);
    // initialize the table of known random number generators
    gsl::rng::initialize();

    // return the newly created module
    return module;
}
  \end{iC++}
%
 python will call this function to create the module object, if the dynamic loading of the
 extension is successful
%
\end{frame}

% --------------------------------------
% building a map of known generators
\begin{frame}[fragile]
%
  \frametitle{Building a table of known generators}
%
  \begin{iC++}
// the table of generators
namespace gsl {
    namespace rng {
        typedef std::map<std::string, const gsl_rng_type *> map_t;
        static map_t generators;
    }
}

// initialization of the known generators
void 
gsl::rng::initialize()
{
    // iterate over the registered names
    const gsl_rng_type **current = gsl_rng_types_setup();
    while (*current !=0) {
        // add each one to my map
        gsl::rng::generators[(*current)->name] = *current;
        // get the next one
        current++;
    }

    return;
}
  \end{iC++}
%
\end{frame}

% --------------------------------------
% module definition
\begin{frame}[fragile]
%
  \frametitle{Module definition}
%
  the identifier \identifier{module\_definition} refers to a \keyword{struct}
  \begin{iC++}{}
// the module documentation string
const char * const __doc__ = "sample module documentation string";
// the module definition structure
PyModuleDef gsl::module_definition = {
    // header
    PyModuleDef_HEAD_INIT,
    // the name of the module
    "gsl",
    // the module documentation string
    __doc__,
    // size of the per-interpreter state of the module
    -1, // don't touch
    // the methods defined in this module
    gsl::module_methods
};
  \end{iC++}
%
\end{frame}

% end of file 
