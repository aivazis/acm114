% -*- LaTeX -*-
% -*- coding: utf-8 -*-
%
% ~~~~~~~~~~~~~~~~~~~~~~~~~~~~~~~~~~~~~~~~~~~~~~~~~~~~~~~~~~~~~~~~~~~~~~~~~~~~~~
%
%                             michael a.g. aïvázis
%                      california institute of technology
%                      (c) 1998-2010  all rights reserved
%
% ~~~~~~~~~~~~~~~~~~~~~~~~~~~~~~~~~~~~~~~~~~~~~~~~~~~~~~~~~~~~~~~~~~~~~~~~~~~~~~
%

\lecture{Structured grids - II}{20100210}

% --------------------------------------
% sequential
\begin{frame}[fragile]
%
  \frametitle{Sequential implementation - user interface}
%
  \begin{lstlisting}[language=c++,name=seq:frame,firstnumber=77]
// main program
int main(int argc, char* argv[]) {
    // default values for our user configurable settings
    size_t N = 10;
    double tolerance = 1.0e-6;
    const char* filename = "laplace.csv";

    // read the command line
    int command;
    while ((command = getopt(argc, argv, "N:e:o:")) != -1) {
        switch (command) {
        // get the convergence tolerance
        case 'e':
            tolerance = atof(optarg);
            break;
        // get the grid size
        case 'N':
            N = (size_t) atof(optarg);
            break;
        // get the name of the output file
        case 'o':
            filename = optarg;
        }
    }
    
  \end{lstlisting}
% 
\end{frame}

% --------------------------------------
% sequential
\begin{frame}[fragile]
%
  \frametitle{Sequential implementation - driving the solver}
%
  \begin{lstlisting}[language=c++,name=seq:frame]
    // allocate space for the solution
    Grid potential(N);

    // initialize and apply our boundary conditions
    initialize(potential);

    // call the solver
    laplace(potential, tolerance);

    // open a stream to hold the answer
    std::fstream output(filename, std::ios_base::out);

    // build a visualizer and render the solution in our chosen format
    Visualizer visualizer;
    visualizer.csv(potential, output);

    // all done
    return 0;
}
  \end{lstlisting}
% 
\end{frame}

% --------------------------------------
% sequential
\begin{frame}[fragile]
%
  \frametitle{Sequential implementation - the preamble}
%
  \begin{itemize}
  \item back up to the beginning of the file
    \begin{lstlisting}[language=c++,name=seq:frame, firstnumber=1]
#include <getopt.h>
#include <cmath>
#include <cstdlib>
#include <fstream>
#include <iostream>

// forward declarations
class Grid;
class Visualizer;

// the solver; does nothing for the time being
void initialize(Grid & grid) {};
void laplace(Grid & grid, double tolerance){};

    \end{lstlisting}
%
  \item we have separated out {\em visualization} in a different object to support different
    formats without disturbing the data representation
%
  \item \identifier{initialize} and \identifier{laplace} have trivial implementations for now
    \begin{itemize}
    \item enables testing the scaffolding without worrying about the solver
      implementation just yet
    \end{itemize}
  \end{itemize}
% 
\end{frame}

% --------------------------------------
% sequential
\begin{frame}[fragile]
%
  \frametitle{Sequential implementation - the grid object stub}
%
  \begin{lstlisting}[language=c++,name=seq:frame]
// the solution representation
class Grid {
    // interface: TBD
public:

    // meta methods
public:
    Grid(size_t size);
    ~Grid();

    // private data members: TBD
private:

    // disabled interface
    // grid will own dynamic memory, so don't let the compiler screw up
private:
    Grid(const Grid &);
    const Grid & operator= (const Grid &);
};

// the grid implementation
Grid::Grid(size_t size) {
}

Grid::~Grid() {
}

  \end{lstlisting}
% 
\end{frame}

% --------------------------------------
% sequential
\begin{frame}[fragile]
%
  \frametitle{Sequential implementation - the visualizer stub}
%
  \begin{lstlisting}[language=c++,name=seq:frame, firstnumber=97]
// the visualizer class
class Visualizer {
    // local type aliases
public:
    typedef std::ostream stream_t;

    // interface
public:
    void csv(const Grid & grid, stream_t & stream);

    // meta methods
public:
    inline Visualizer() {}
};

// the Visualizer class implementation
void Visualizer::csv(const Grid & grid, Visualizer::stream_t & stream) {
    return;
}
  \end{lstlisting}
%

\begin{itemize}
\item the code now compiles and links
  \begin{itemize}
  \item consistency check that the object collaborations are ok, for now
  \item can be tested for command line option parsing
  \end{itemize}
\end{itemize}
% 
\end{frame}

% --------------------------------------
% the grid initializer
\begin{frame}[fragile]
%
  \frametitle{Fleshing out the initializer}
%
  \begin{lstlisting}[language=c++,name=seq:initializer]
// the grid initializer:
// clear the grid contents and apply our boundary conditions 
void initialize(Grid & grid) {
    // ask the grid to clear its memory
    grid.clear(1.0);
    // apply the dirichlet conditions
    for (size_t cell=0; cell < grid.size(); cell++) {
        // evaluate sin(pi x)
        double sin = std::sin(cell * grid.delta() * pi);
        // along the x axis, at top and  bottom
        grid(cell, 0) = sin;
        grid(cell, grid.size()-1) = sin * std::exp(-pi);
        // along the y axis, left and right
        grid(0, cell) = 0.0;
        grid(grid.size()-1, cell) = 0.0;
    }

    return;
}
  \end{lstlisting}
%
  \begin{itemize}
  \item the grid knows its size, its spacing $\delta$, and can initialize out its  memory
  \item access to grid elements happens through an overloaded \identifier{operator()} so we can
    {\em encapsulate} the indexing function
  \end{itemize}
%
\end{frame}

% --------------------------------------
% the grid declaration
\begin{frame}[fragile]
%
  \frametitle{The grid class declaration}
%
  \begin{lstlisting}[language=c++,name=seq:grid,firstnumber=29]
// the solution representation
class Grid {
    // interface
public:
    // set all cells to the specified value
    void clear(double value=0.0);
    // the grid dimensions
    size_t size() const {return _size;}
    // the grid spacing
    double delta() const {return _delta;}
    // access to the cells
    double & operator()(size_t i, size_t j) {return _block[j*_size+i];}
    double operator()(size_t i, size_t j) const {return _block[j*_size+i];}
    // meta methods
public:
    Grid(size_t size);
    ~Grid();
    // data members
private:
    const size_t _size;
    const double _delta;
    double* _block;
    // disable these
private:
    Grid(const Grid &);
    const Grid & operator= (const Grid &);
};

  \end{lstlisting}
%
\end{frame}

% --------------------------------------
% the grid implementation
\begin{frame}[fragile]
%
  \frametitle{The grid class implementation}
%
  \begin{lstlisting}[language=c++,name=seq:grid]
// the grid implementation
// interface
void Grid::clear(double value) {
    for (size_t i=0; i < _size*_size; i++) {
        _block[i] = value;
    }

    return;
}

// constructor
Grid::Grid(size_t size) :
    _size(size), 
    _delta((1.0 - 0.0)/(size-1)),
    _block(new double[size*size]) {
}

// destructor
Grid::~Grid() {
    delete [] _block;
}

  \end{lstlisting}
%
\end{frame}

% --------------------------------------
% the grid visualizer
\begin{frame}[fragile]
%
  \frametitle{Grid visualization}
%
  \begin{lstlisting}[language=c++,name=seq:visualizer,firstnumer=97]
// the visualizer class
class Visualizer {
    // local type aliases
public:
    typedef std::ostream stream_t;
    // interface
public:
    void csv(const Grid & grid, stream_t & stream);
    // meta methods
public:
    inline Visualizer() {}
};

// the Visualizer class implementation
void Visualizer::csv(const Grid & grid, Visualizer::stream_t & stream) {
    for (size_t j=0; j < grid.size(); j++) {
        stream << j;
        for (size_t i=0; i < grid.size(); i++) {
            stream << "," << grid(i,j);
        }
        stream << std::endl;
    }

    return;
}

  \end{lstlisting}
%
\end{frame}

% --------------------------------------
% the results of initialization
\begin{frame}[fragile]
%
  \frametitle{Printing out the initial grid}
%
  \begin{itemize}
  \item we should be able to print out the initialized grid
%
  \begin{shell}{}
#> mm laplace
#> laplace
#> cat laplace.csv
0,0,0.3827,0.7071,0.9239,1,0.9239,0.7071,0.3827,1.225e-16
1,0,1,1,1,1,1,1,1,0
2,0,1,1,1,1,1,1,1,0
3,0,1,1,1,1,1,1,1,0
4,0,1,1,1,1,1,1,1,0
5,0,1,1,1,1,1,1,1,0
6,0,1,1,1,1,1,1,1,0
7,0,1,1,1,1,1,1,1,0
8,0,0.01654,0.0306,0.03992,0.04321,0.0399,0.03056,0.01654,0
  \end{shell}
%
  \item notice that
    \begin{itemize}
    \item the top line contains some recognizable values
    \item the left and right borders are set to zero
    \item the interior of the grid is painted with our initial guess
    \end{itemize}
%
  \item still to do:
    \begin{itemize}
    \item write the update
    \item build a grid with the exact solution 
    \item build the error field (why?)
    \end{itemize}
  \end{itemize}
%
\end{frame}

% end of file 
