% -*- LaTeX -*-
% -*- coding: utf-8 -*-
%
% ~~~~~~~~~~~~~~~~~~~~~~~~~~~~~~~~~~~~~~~~~~~~~~~~~~~~~~~~~~~~~~~~~~~~~~~~~~~~~~
%
%                             michael a.g. aïvázis
%                      california institute of technology
%                      (c) 1998-2010  all rights reserved
%
% ~~~~~~~~~~~~~~~~~~~~~~~~~~~~~~~~~~~~~~~~~~~~~~~~~~~~~~~~~~~~~~~~~~~~~~~~~~~~~~
%

\lecture{A look at parallelism and implementations}{20100201}

% --------------------------------------
% problem statement
\begin{frame}[fragile]
%
  \frametitle{Approximating $\Li$ using a numerical quadrature}
%
  \begin{itemize}
%
  \item the second homework assignment involved $\Li(z)$, defined by
    \begin{equation*}
      \Li(z) \bydef - \int_{0}^{z} dz' \; \frac{\log(1-z')}{z'}
    \end{equation*}
%
    \item the assignment asked for approximating this integral using a simple
      quadrature based on the mid-point rule
      \begin{equation*}
      \Li(z) 
      \approx 
      \Li(z, N) 
      \bydef
      - \frac{z}{N} \sum_{n=0}^{N-1} \left.
        \frac{\log(1-z')}{z'}
      \right|_{z'=(n+\frac{1}{2})\frac{z}{N}}
    \end{equation*}
%
    \item three implementations
      \begin{itemize}
      \item sequential: to get a feeling for how to convert the algorithm into a functioning
        program
      \item parallel using threads: to walk through the parallelization steps and use
        \identifier{pthreads} to get better performance
      \item parallel using \mpi: to get a feel for how \mpi-based programs solve the task
        partitioning problem
      \end{itemize}
%
    \item let's walk through composing, building and running my solutions
      \begin{itemize}
      \item on my desktop, on \href{mind-meld.cacr.caltech.edu}, and on \href{shc.cacr.caltech.edu}
      \end{itemize}
%
  \end{itemize}
%
\end{frame}

% --------------------------------------
% the sequential driver - preamble
\begin{frame}[fragile]
%
  \frametitle{Sequential implementation - part 1}
%
  \begin{itemize}
  \item the preamble
  \begin{lstlisting}[language=c++,name=sequential]
#include <getopt.h> // for getopt and friends
#include <cstdlib>  // for atof
#include <cmath>    // for the correct abs, log

#include <map>
#include <iostream>
#include <iomanip>
  \end{lstlisting}
%
  \item quadrature using the midpoint rule to avoid the singularities
  \begin{lstlisting}[language=c++,name=sequential]
// dilog
double dilog(double z, long N) {
    // initialize
    double dx = z/N;
    double x = dx/2;
    double sum = 0;
    // loop
    for (long i=0; i < N; i++) {
        sum += std::log(1-x)/x;
        x += dx;
    }
    // return; don't forget the sign
    return -dx * sum;
}

  \end{lstlisting}
%
  \end{itemize}
%
\end{frame}

% --------------------------------------
% the sequential driver - the main program
\begin{frame}[fragile]
%
  \frametitle{Sequential implementation - part 2}
%
  \begin{itemize}
  \item using the command line to set $z$ and the number of subdivisions $N$
  \begin{lstlisting}[language=c++,name=sequential]
// main program
int main(int argc, char* argv[]) {
    //  default values for the command line options
    long N = 1000;
    double z = 1.0;

    // read the command line
    int command;
    while ((command = getopt(argc, argv, "z:N:")) != -1) {
        switch (command) {
        // get the argument of the dilogarithm 
        case 'z':
            z = atof(optarg);
            break;
        // get the number of subdivisions
        case 'N':
            N = (long) atof(optarg);
            break;
        }
    }
  \end{lstlisting}
%
  \end{itemize}
%
\end{frame}

% --------------------------------------
% the sequential driver - the main program
\begin{frame}[fragile]
%
  \frametitle{Sequential implementation - part 3}
%
  \begin{itemize}
  \item error checking and computation of the numerical integral
  \begin{lstlisting}[language=c++,name=sequential]
    // error checking
    // abort if N < 1
    if (N < 1) {
        std::cout 
            << "the number of subdivisions must be positive"
            << std::endl;
        return 0;
    }

    // abort for z > 1 to avoid dealing with the imaginary part
    if (z > 1.0) {
        std::cout << "math domain error: z > 1" << std::endl;
        return 0;
    } 

    // compute
    double value = dilog(z, N);
  \end{lstlisting}
%
  \end{itemize}
%
\end{frame}

% --------------------------------------
% the sequential driver - the main program
\begin{frame}[fragile]
%
  \frametitle{Sequential implementation - part 4}
%
  \begin{itemize}
  \item computing the error and printing out the results
  \begin{lstlisting}[language=c++,name=sequential]
    // build a database of the known dilogarithm values
    const double pi = M_PI;
    std::map<double, double> answers;
    answers[1.0] = pi*pi/6;
    answers[-1.0] = -pi*pi/12;

    // print out the value
    std::cout << "Li2(" << z << ")="
        << std::setprecision(17) << std::endl
        << " computed: " << value << std::endl;
    // check whether we know the right answer
    std::map<double,double>::const_iterator lookup = answers.find(z);
    if (lookup != answers.end()) {
        // and if we do, print it out
        double exact = lookup->second;
        std::cout << "    exact: " << exact << std::endl;
        // compute the approximation error and print it out
        double error = std::abs(exact-value)/exact;
        std::cout 
            << std::setiosflags(std::ios_base::scientific) 
            << "    error: " << error << std::endl;
    }

    return 0;
}
  \end{lstlisting}
%
  \end{itemize}
%
\end{frame}

% --------------------------------------
% running the sequential driver
\begin{frame}[fragile]
%
  \frametitle{Building and running the sequential driver}
%
  \begin{shell}{}
#> g++ dilog_sequential.cc -o dilog_sequential
#> dilog_sequential -N 1e7 -z 1.0
Li2(1)=
 computed: 1.6449340282186398
    exact: 1.6449340668482264
    error: 2.34839726522278546e-08
#> time dilog_sequential -N 1e9 -z 1.0
Li2(1)=
 computed: 1.6449339414016682
    exact: 1.6449340668482264
    error: 7.62623625958871898e-08

real    0m19.885s
user    0m19.877s
sys     0m0.003s
#>
  \end{shell}
%
\end{frame}

% --------------------------------------
% the threaded driver - preamble
\begin{frame}[fragile]
%
  \frametitle{Threaded implementation - part 1}
%
  \begin{itemize}
  \item the preamble
  \begin{lstlisting}[language=c++,name=threaded]
#include <getopt.h> // for getopt and friends
#include <pthread.h>

#include <cstdio>
#include <cstdlib>  // for atof
#include <cmath>

#include <map>
#include <iostream>
#include <iomanip>

  \end{lstlisting}
%
  \end{itemize}
%
\end{frame}

% --------------------------------------
% the threaded driver - thread data structures
\begin{frame}[fragile]
%
  \frametitle{Threaded implementation - part 2}
%
  \begin{itemize}
  \item private and shared data structures
  \begin{lstlisting}[language=c++,name=threaded]
// shared information
struct problem {
    int workers;           // total number of threads
    double dz;             // the width of each subdivision
    double sum;            // storage for the partial computations

    pthread_mutex_t lock;  // mutex to control access to the sum
};

// thread specific information
struct context {
    // thread info
    int id;
    pthread_t descriptor;
    // the workload for this thread
    long subdivisions;  // number of subdivisions
    double z_low;       // the lower limit of integration
    double partial;     // record the partial sum computed by this thread
    // the shared problem information
    problem* info;
};
  \end{lstlisting}
%
  \end{itemize}
%
\end{frame}

% --------------------------------------
% the threaded driver - worker
\begin{frame}[fragile]
%
  \frametitle{Threaded implementation - part 3}
%
  \begin{itemize}
  \item the coarse grain task
  \begin{lstlisting}[language=c++,name=threaded]
// worker
void* worker(void* arg) {
    context* ctxt = (context *) arg;
    // pull the problem information from the thread context
    double dz = ctxt->info->dz;
    double z = ctxt->z_low + dz/2;
    // loop over the subdivisions assigned to this thread
    double sum = 0.0;
    for (long i=0; i < ctxt->subdivisions; i++) {
        sum += std::log(1-z)/z;
        z += dz;
    }
    // multiply by the width of each subdivision and adjust the sign
    sum *= -dz;

    // grab the lock
    pthread_mutex_lock(&(ctxt->info->lock));
    // store the result
    ctxt->info->sum += sum;
    // and release the lock
    pthread_mutex_unlock(&(ctxt->info->lock));

    // all done
    return 0;
}
  \end{lstlisting}
%
  \end{itemize}
%
\end{frame}

% --------------------------------------
% the threaded driver - task master
\begin{frame}[fragile]
%
  \frametitle{Threaded implementation - part 4}
%
  \begin{itemize}
  \item the task master -- interface and allocation of storage
  \begin{lstlisting}[language=c++,name=threaded]
// driver
double dilog(double z, long N, int threads) {
    // the width of each interval subdivision
    const double dz = z/N;

    // setup the problem context
    problem info;
    info.workers = threads;
    info.dz = dz;
    info.sum = 0.0;
    pthread_mutex_init(&info.lock, 0);

    // and an array to hold the thread contexts
    context thr_info[threads];
    // partition the number of subdivisions
    long nominal_load = N/threads;

  \end{lstlisting}
%
  \end{itemize}
%
\end{frame}

% --------------------------------------
% the threaded driver - task master continued
\begin{frame}[fragile]
%
  \frametitle{Threaded implementation - part 5}
%
  \begin{itemize}
  \item the task master -- spawning the threads
  \begin{lstlisting}[language=c++,name=threaded]
    // spawn the workers
    for (int tid=0; tid<threads; tid++) {
        // store the thread id
        thr_info[tid].id = tid;
        // point to the shared problem info
        thr_info[tid].info = &info;

        // compute the starting point of the partial integral
        thr_info[tid].z_low = tid*nominal_load*dz;
        // compute the number of subdivisions for this thread
        if (tid == threads - 1) {
            // the last thread gets the leftovers
            thr_info[tid].subdivisions = N - tid*nominal_load;
        } else {
            thr_info[tid].subdivisions = nominal_load;
        }

        // create the thread
        int status = pthread_create(
            &(thr_info[tid].descriptor), 0, worker, &thr_info[tid]);
        if (status) {
            printf("error %d in pthread_create\n", status);
        }
    }
  \end{lstlisting}
%
  \end{itemize}
%
\end{frame}

% --------------------------------------
% the threaded driver - task master continued
\begin{frame}[fragile]
%
  \frametitle{Threaded implementation - part 6}
%
  \begin{itemize}
  \item the task master -- harvesting the threads and returning the result
  \begin{lstlisting}[language=c++,name=threaded]
    // harvest the threads
    for (int tid=0; tid<threads; tid++) {
        pthread_join(thr_info[tid].descriptor, 0);
    }

    // all done
    return info.sum;
}
  \end{lstlisting}
%
  \end{itemize}
%
\end{frame}

% --------------------------------------
% the threaded driver - main program
\begin{frame}[fragile]
%
  \frametitle{Threaded implementation - part 7}
%
  \begin{itemize}
  \item the main program -- reading the command line
  \begin{lstlisting}[language=c++,name=threaded]
// main program
int main(int argc, char* argv[]) {
    //  default values for the command line options
    long N = 1000;
    double z = 1.0;
    int threads = 8;

    // read the command line
    int command;
    while ((command = getopt(argc, argv, "z:N:t:")) != -1) {
        switch (command) {
        case 'z':
            // get the argument of the dilogarithm 
            z = atof(optarg);
            break;
        case 'N':
            // get the number of subdivisions
            N = (long) atof(optarg);
            break;
        case 't':
            // get the numberof threads
            threads = atoi(optarg);
            break;
        }
    }

  \end{lstlisting}
%
  \end{itemize}
%
\end{frame}

% --------------------------------------
% the threaded driver - main program
\begin{frame}[fragile]
%
  \frametitle{Threaded implementation - part 8}
%
  \begin{itemize}
  \item error checking and the invocation of the task master
  \begin{lstlisting}[language=c++,name=threaded]
    // error checking
    // abort if N < 1
    if (N < 1) {
        std::cout 
            << "the number of subdivisions must be positive"
            << std::endl;
        return 0;
    }

    // abort for z > 1 to avoid dealing with the imaginary part
    if (z > 1.0) {
        std::cout << "math domain error: z > 1" << std::endl;
        return 0;
    } 

    // compute
    double value = dilog(z, N, threads);

  \end{lstlisting}
%
  \end{itemize}
%
\end{frame}

% --------------------------------------
% the threaded driver - main program
\begin{frame}[fragile]
%
  \frametitle{Threaded implementation - part 9}
%
  \begin{itemize}
  \item the task master -- printing out the answers
  \begin{lstlisting}[language=c++,name=threaded]
    // build a database of the known dilogarithm values
    const double pi = M_PI;
    std::map<double, double> answers;
    answers[1.0] = pi*pi/6;
    answers[-1.0] = -pi*pi/12;

    // print out the value
    std::cout << "Li2(" << z << ")="
         << std::setprecision(17) << std::endl
         << " computed: " << value << std::endl;
    // check whether we know the right answer
    std::map<double,double>::const_iterator lookup = answers.find(z);
    if (lookup != answers.end()) {
        // and if we do, print it out
        double exact = lookup->second;
        std::cout << "    exact: " << exact << std::endl;
        // compute the approximation error and print it out
        double error = std::abs(exact-value)/exact;
        std::cout 
            << std::setiosflags(std::ios_base::scientific) 
            << "    error: " << error << std::endl;
    }

    return 0;
}
  \end{lstlisting}
%
  \end{itemize}
%
\end{frame}

% --------------------------------------
% running the threaded driver
\begin{frame}[fragile]
%
  \frametitle{Building and running the threaded driver}
%
  \begin{shell}{}
#> g++ dilog_threads.cc -o dilog_threads -pthread
#> dilog_threads -N 1e7 -z 1.0 -t 4
Li2(1)=
 computed: 1.6449340301295035
    exact: 1.6449340668482264
    error: 2.23223068274304058e-08
#> time dilog_threads -N 1e9 -z 1.0 -t 8
Li2(1)=
 computed: 1.6449340044883614
    exact: 1.6449340668482264
    error: 3.79102520315773892e-08

real    0m2.803s
user    0m20.693s
sys     0m0.006s
#>
  \end{shell}
%
\end{frame}

% --------------------------------------
% the mpi driver - the preamble
\begin{frame}[fragile]
%
  \frametitle{MPI implementation - part 1}
%
  \begin{itemize}
  \item the preamble
  \begin{lstlisting}[language=c++,name=mpi]
#include <getopt.h> // for getopt and friends
#include <mpi.h>

#include <cstdio>
#include <cstdlib>  // for atof
#include <cmath>

#include <map>
#include <iostream>
#include <iomanip>

  \end{lstlisting}
%
  \end{itemize}
%
\end{frame}

% --------------------------------------
% the mpi driver -  the coarse grain task
\begin{frame}[fragile]
%
  \frametitle{MPI implementation - part 2}
%
  \begin{itemize}
  \item coarse grain task
  \begin{lstlisting}[language=c++,name=mpi]
double dilog(double zprime, long N, int id, int processes) {
    // the width of each interval subdivision
    const double dz = zprime/N;
    // compute the starting point of the partial integral
    const double z_low = id*zprime/processes;
    // partition the number of subdivisions
    long nominal_load = N/processes;
    // the last process gets the leftovers
    if (id == processes - 1) {
        nominal_load = N - id*nominal_load;
    }
    // initialize the partial sum
    double sum = 0.0;
    double z = z_low + dz/2;
    // loop over the subdivisions assigned to this thread
    for (long i=0; i < nominal_load; i++)  {
        sum += std::log(1-z)/z;
        z += dz;
    }
    // collect the partial answers from all the processes
    double value;
    MPI_Allreduce(
        &sum, &value, 1, MPI_DOUBLE, MPI_SUM, MPI_COMM_WORLD);
    // multiply by the width of each subdivision and adjust the sign
    return -dz*value;
}
  \end{lstlisting}
%
  \end{itemize}
%
\end{frame}

% --------------------------------------
% the mpi driver - the main program
\begin{frame}[fragile]
%
  \frametitle{MPI implementation - part 3}
%
  \begin{itemize}
  \item the main program -- setting up \mpi
  \begin{lstlisting}[language=c++,name=mpi]
// main program
int main(int argc, char* argv[]) {
    // initialize MPI
    int status = MPI_Init(&argc, &argv);
    if (status != MPI_SUCCESS) {
        std::cout << "error in MPI_Init; aborting..." << std::endl;
        return status;
    }
    // get process information from the world communicator
    int id, processes;
    MPI_Comm_rank(MPI_COMM_WORLD, &id);
    MPI_Comm_size(MPI_COMM_WORLD, &processes);

  \end{lstlisting}
%
  \end{itemize}
%
\end{frame}

% --------------------------------------
% the mpi driver - the main program
\begin{frame}[fragile]
%
  \frametitle{MPI implementation - part 4}
%
  \begin{itemize}
  \item reading the command line
  \begin{lstlisting}[language=c++,name=mpi]
    //  default values for the command line options
    long N = 1000;
    double z = 1.0;
    // read the command line
    int command;
    while ((command = getopt(argc, argv, "z:N:")) != -1) {
        switch (command) {
        case 'z':
            // get the argument of the dilogarithm 
            z = atof(optarg);
            break;
        case 'N':
            // get the number of subdivisions
            N = (long) atof(optarg);
            break;
        }
    }
  \end{lstlisting}
%
  \end{itemize}
%
\end{frame}

% --------------------------------------
% the mpi driver -  error checking and computation
\begin{frame}[fragile]
%
  \frametitle{MPI implementation - part 5}
%
  \begin{itemize}
  \item error checking and computation
  \begin{lstlisting}[language=c++,name=mpi]
    // error checking
    // abort if N < 1
    if (N < 1) {
        if (id == 0) {
            std::cout 
                << "the number of subdivisions must be positive"
                << std::endl;
        }
        MPI_Finalize();
        return 0;
    }
    // abort for z > 1 to avoid dealing with the imaginary part
    if (z > 1.0) {
        if (id == 0) {
            std::cout << "math domain error: z > 1" << std::endl;
        }
        MPI_Finalize();
        return 0;
    } 
    // compute
    double value = dilog(z, N, id, processes);
    if (id != 0) { // let all but processor 0 die
        // shut down MPI
        MPI_Finalize();
        return 0;
    }
  \end{lstlisting}
%
  \end{itemize}
%
\end{frame}

% --------------------------------------
% the mpi driver - printing out the results
\begin{frame}[fragile]
%
  \frametitle{MPI implementation - part 6}
%
  \begin{itemize}
  \item printing out the results
  \begin{lstlisting}[language=c++,name=mpi]
    // build a database of the known dilogarithm values
    const double pi = M_PI;
    std::map<double, double> answers;
    answers[1.0] = pi*pi/6;
    answers[-1.0] = -pi*pi/12;

    // print out the value
    std::cout << "Li2(" << z << ")=" << std::setprecision(17) << std::endl;
    std::cout << " computed: " << value << std::endl;
    // check whether we know the right answer
    std::map<double,double>::const_iterator lookup = answers.find(z);
    if (lookup != answers.end()) {
        // and if we do, print it out
        double exact = lookup->second;
        std::cout << "    exact: " << exact << std::endl;
        // compute the approximation error and print it out
        double error = std::abs(exact-value)/exact;
        std::cout 
            << std::setiosflags(std::ios_base::scientific) 
            << "    error: " << error << std::endl;
    }

    // shut down MPI
    MPI_Finalize();
    return 0;
}
  \end{lstlisting}
%
  \end{itemize}
%
\end{frame}

% --------------------------------------
% running the mpi driver
\begin{frame}[fragile]
%
  \frametitle{Building and running the \mpi\ driver}
%
  \begin{itemize}
  \item on my desktop, or \href{mind-meld.cacr.caltech.edu}
    \begin{itemize}
    \item where there is no queue manager
    \end{itemize}
  \end{itemize}
%
  \begin{shell}{}
#> mpic++ dilog_mpi.cc -o dilog_mpi -lmpi_cxx -lmpi
#> mpirun -np 4 dilog_mpi -N 1e7 -z 1.0
Li2(1)=
 computed: 1.6449340301295035
    exact: 1.6449340668482264
    error: 2.23223068274304058e-08
#> time mpirun -np 8 dilog_mpi -N 1e9 -z 1.0
Li2(1)=
 computed: 1.6449340044883614
    exact: 1.6449340668482264
    error: 3.79102520315773892e-08

real    0m3.697s
user    0m0.018s
sys     0m0.015s
#>
  \end{shell}
%
\end{frame}

% --------------------------------------
% running the mpi driver
\begin{frame}[fragile]
%
  \frametitle{Running the \mpi\ driver on a shared resource}
%
  \begin{itemize}
  \item on \href{shc.cacr.caltech.edu} there is a queue manager
    \begin{itemize}
    \item don't use \identifier{mpirun}: you are running on the head node
    \item instead, request a dedicated node
    \end{itemize}
  \end{itemize}
%
  \begin{shell}{}
# shc-a> mpic++ dilog_mpi.cc -o dilog_mpi
# shc-a> qsub -I -l nodes=1:core8 -l walltime=0:15:00
qsub: waiting for job 105059.mistress to start
qsub: job 105059.mistress ready
Logging in as aivazis on shc168, a Linux-2.x_x86_64 system
  setting up: (environment) (aliases) (machines) (tools: Linux-2.x_x86_64)
# shc168> time mpirun -np 8 dilog_mpi -N 1e9 -z 1.0
Li2(1)=
 computed: 1.6449340044883614
    exact: 1.6449340668482264
    error: 3.79102520315773892e-08

real    0m10.501s
user    1m14.642s
sys     0m0.273s
# shc168> exit
logout
qsub: job 105059.mistress completed
# shc-a>
  \end{shell}
%
\end{frame}

% end of file 
