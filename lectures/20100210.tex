% -*- LaTeX -*-
% -*- coding: utf-8 -*-
%
% ~~~~~~~~~~~~~~~~~~~~~~~~~~~~~~~~~~~~~~~~~~~~~~~~~~~~~~~~~~~~~~~~~~~~~~~~~~~~~~
%
%                             michael a.g. aïvázis
%                      california institute of technology
%                      (c) 1998-2010  all rights reserved
%
% ~~~~~~~~~~~~~~~~~~~~~~~~~~~~~~~~~~~~~~~~~~~~~~~~~~~~~~~~~~~~~~~~~~~~~~~~~~~~~~
%

\lecture{Structured grids - II}{20100210}

% --------------------------------------
% sequential
\begin{frame}[fragile]
%
  \frametitle{Sequential implementation - driving the solver}
%
  \begin{lstlisting}[language=c++,name=seq:frame]
    // allocate space for the solution
    Grid potential(N);

    // initialize and apply our boundary conditions
    initialize(potential);

    // call the solver
    laplace(potential, tolerance);

    // open a stream to hold the answer
    std::fstream output(filename, std::ios_base::out);

    // build a visualizer and render the solution in our chosen format
    Visualizer visualizer;
    visualizer.csv(potential, output);

    // all done
    return 0;
}
  \end{lstlisting}
% 
\end{frame}

% --------------------------------------
% sequential
\begin{frame}[fragile]
%
  \frametitle{Sequential implementation - the preamble}
%
  \begin{itemize}
  \item back up to the beginning of the file
    \begin{lstlisting}[language=c++,name=seq:frame, firstnumber=1]
#include <getopt.h>
#include <cmath>
#include <cstdlib>
#include <fstream>
#include <iostream>

// forward declarations
class Grid;
class Visualizer;

// the solver; does nothing for the time being
void initialize(Grid & grid) {};
void laplace(Grid & grid, double tolerance){};

    \end{lstlisting}
%
  \item we have separated out {\em visualization} in a different object to support different
    formats without disturbing the data representation
%
  \item \identifier{initialize} and \identifier{laplace} have trivial implementations for now
    \begin{itemize}
    \item enables testing the scaffolding without worrying about the solver
      implementation just yet
    \end{itemize}
  \end{itemize}
% 
\end{frame}

% --------------------------------------
% sequential
\begin{frame}[fragile]
%
  \frametitle{Sequential implementation - the grid object stub}
%
  \begin{lstlisting}[language=c++,name=seq:frame]
// the solution representation
class Grid {
    // interface: TBD
public:

    // meta methods
public:
    Grid(size_t size);
    ~Grid();

    // private data members: TBD
private:

    // disabled interface
    // grid will own dynamic memory, so don't let the compiler screw up
private:
    Grid(const Grid &);
    const Grid & operator= (const Grid &);
};

// the grid implementation
Grid::Grid(size_t size) {
}

Grid::~Grid() {
}

  \end{lstlisting}
% 
\end{frame}

% --------------------------------------
% sequential
\begin{frame}[fragile]
%
  \frametitle{Sequential implementation - the visualizer stub}
%
  \begin{lstlisting}[language=c++,name=seq:frame, firstnumber=97]
// the visualizer class
class Visualizer {
    // local type aliases
public:
    typedef std::ostream stream_t;

    // interface
public:
    void csv(const Grid & grid, stream_t & stream);

    // meta methods
public:
    inline Visualizer() {}
};

// the Visualizer class implementation
void Visualizer::csv(const Grid & grid, Visualizer::stream_t & stream) {
    return;
}
  \end{lstlisting}
%

\begin{itemize}
\item the code now compiles and links
  \begin{itemize}
  \item consistency check that the object collaborations are ok, for now
  \item can be tested for command line option parsing
  \end{itemize}
\end{itemize}
% 
\end{frame}

% --------------------------------------
% the grid initializer
\begin{frame}[fragile]
%
  \frametitle{Fleshing out the initializer}
%
  \begin{lstlisting}[language=c++,name=seq:initializer]
// the grid initializer:
// clear the grid contents and apply our boundary conditions 
void initialize(Grid & grid) {
    // ask the grid to clear its memory
    grid.clear(1.0);
    // apply the dirichlet conditions
    for (size_t cell=0; cell < grid.size(); cell++) {
        // evaluate sin(pi x)
        double sin = std::sin(cell * grid.delta() * pi);
        // along the x axis, at top and  bottom
        grid(cell, 0) = sin;
        grid(cell, grid.size()-1) = sin * std::exp(-pi);
        // along the y axis, left and right
        grid(0, cell) = 0.0;
        grid(grid.size()-1, cell) = 0.0;
    }

    return;
}
  \end{lstlisting}
%
  \begin{itemize}
  \item the grid knows its size, its spacing $\delta$, and can initialize out its  memory
  \item access to grid elements happens through an overloaded \identifier{operator()} so we can
    {\em encapsulate} the indexing function
  \end{itemize}
%
\end{frame}

% --------------------------------------
% the grid declaration
\begin{frame}[fragile]
%
  \frametitle{The grid class declaration}
%
  \begin{lstlisting}[language=c++,name=seq:grid,firstnumber=29]
// the solution representation
class Grid {
    // interface
public:
    // set all cells to the specified value
    void clear(double value=0.0);
    // the grid dimensions
    size_t size() const {return _size;}
    // the grid spacing
    double delta() const {return _delta;}
    // access to the cells
    double & operator()(size_t i, size_t j) {return _block[j*_size+i];}
    double operator()(size_t i, size_t j) const {return _block[j*_size+i];}
    // meta methods
public:
    Grid(size_t size);
    ~Grid();
    // data members
private:
    const size_t _size;
    const double _delta;
    double* _block;
    // disable these
private:
    Grid(const Grid &);
    const Grid & operator= (const Grid &);
};

  \end{lstlisting}
%
\end{frame}

% --------------------------------------
% the grid implementation
\begin{frame}[fragile]
%
  \frametitle{The grid class implementation}
%
  \begin{lstlisting}[language=c++,name=seq:grid]
// the grid implementation
// interface
void Grid::clear(double value) {
    for (size_t i=0; i < _size*_size; i++) {
        _block[i] = value;
    }

    return;
}

// constructor
Grid::Grid(size_t size) :
    _size(size), 
    _delta((1.0 - 0.0)/(size-1)),
    _block(new double[size*size]) {
}

// destructor
Grid::~Grid() {
    delete [] _block;
}

  \end{lstlisting}
%
\end{frame}

% --------------------------------------
% the grid visualizer
\begin{frame}[fragile]
%
  \frametitle{Grid visualization}
%
  \begin{lstlisting}[language=c++,name=seq:visualizer,firstnumer=97]
// the visualizer class
class Visualizer {
    // local type aliases
public:
    typedef std::ostream stream_t;
    // interface
public:
    void csv(const Grid & grid, stream_t & stream);
    // meta methods
public:
    inline Visualizer() {}
};

// the Visualizer class implementation
void Visualizer::csv(const Grid & grid, Visualizer::stream_t & stream) {
    for (size_t j=0; j < grid.size(); j++) {
        stream << j;
        for (size_t i=0; i < grid.size(); i++) {
            stream << "," << grid(i,j);
        }
        stream << std::endl;
    }

    return;
}

  \end{lstlisting}
%
\end{frame}

% --------------------------------------
% the results of initialization
\begin{frame}[fragile]
%
  \frametitle{Printing out the initial grid}
%
  \begin{itemize}
  \item we should be able to print out the initialized grid
%
  \begin{shell}{}
#> mm laplace
#> laplace
#> cat laplace.csv
0,0,0.3827,0.7071,0.9239,1,0.9239,0.7071,0.3827,1.225e-16
1,0,1,1,1,1,1,1,1,0
2,0,1,1,1,1,1,1,1,0
3,0,1,1,1,1,1,1,1,0
4,0,1,1,1,1,1,1,1,0
5,0,1,1,1,1,1,1,1,0
6,0,1,1,1,1,1,1,1,0
7,0,1,1,1,1,1,1,1,0
8,0,0.01654,0.0306,0.03992,0.04321,0.0399,0.03056,0.01654,0
  \end{shell}
%
  \item notice that
    \begin{itemize}
    \item the top line contains some recognizable value
    \item the left and right borders are set to zero
    \item the interior of the grid is painted with our initial guess
    \end{itemize}
%
  \item still to do:
    \begin{itemize}
    \item write the update
    \item build a grid with the exact solution 
    \item build the error field (why?)
    \end{itemize}
  \end{itemize}
%
\end{frame}

% --------------------------------------
% the solver
\begin{frame}[fragile]
%
  \frametitle{Fleshing out the solver}
%
  \begin{lstlisting}[language=c++,basicstyle=\tt\bfseries\tiny,name=seq:solver,firstnumber=169]
// the solver driver
void laplace(Grid & current, double tolerance) {
    // create and initialize temporary storage
    Grid next(current.size());
    initialize(next);
    // put an upper bound on the number of iterations
    long max_iterations = (long) 1e4;;
    for (long iterations = 0; iterations<max_iterations; iterations++) {
        double max_dev = 0.0;
        // do an iteration step
        // leave the boundary alone
        // iterate over the interior of the grid
        for (size_t j=1; j < current.size()-1; j++) {
            for (size_t i=1; i < current.size()-1; i++) {
                // update
                next(i,j) = 0.25*(
                    current(i+1,j)+current(i-1,j)+current(i,j+1)+current(i,j-1));
                // compute the deviation from the last generation
                double dev = std::abs(next(i,j) - current(i,j));
                // and update the maximum deviation
                if (dev > max_dev) {
                    max_dev = dev;
                }
            }
        }
        // swap the blocks between the two grids
        Grid::swapBlocks(current, next);
        // check covergence
        if (max_dev < tolerance) {
            break;
        }
    }
    return;
}
  \end{lstlisting}
%
\end{frame}

% --------------------------------------
% the updated grid class
\begin{frame}[fragile]
%
  \frametitle{Adding the new grid interface}
%
  \begin{itemize}
  \item here is the declaration of \function{Grid::swapBlocks}
%
    \begin{lstlisting}[language=c++,firstnumber=30]
class Grid {
    // interface
    public:
    ...
    // exchange the data blocks of two compatible grids
    static void swapBlocks(Grid &, Grid &);
    ...
};
    \end{lstlisting}
%
  \item and its definition
%
    \begin{lstlisting}[language=c++,firstnumber=69]
void Grid::swapBlocks(Grid & g1, Grid & g2) {
    // bail out if the two operands are not compatible
    if (g1.size() != g2.size()) {
        throw "Grid::swapblocks: size mismatch";
    }
    if (g1.delta() != g2.delta()) {
        throw "Grid::swapblocks: spacing mismatch";
    }
    // but if they are, just exhange their data buffers
    double * temp = g1._block;
    g1._block = g2._block;
    g2._block = temp;
    // all done
    return;
}
    \end{lstlisting}
%
  \end{itemize}
%
\end{frame}

% --------------------------------------
% rework the driver to print out the exact solution and the error field
\begin{frame}[fragile]
%
  \frametitle{Reworking the driver}
%
  \begin{lstlisting}[language=c++,basicstyle=\tt\bfseries\tiny,name=seq:driver,firstnumber=239]
    // build a visualizer
    Visualizer vis;

    // compute the exact solution
    Grid solution(N);
    exact(solution);
    std::fstream exact_stream("exact.csv", std::ios_base::out);
    vis.csv(solution, exact_stream);
    
    // allocate space for the solution
    Grid potential(N);
    // initialize and apply our boundary conditions
    initialize(potential);
    // call the solver
    laplace(potential, tolerance);
    // open a stream to hold the answer
    std::fstream output_stream(filename, std::ios_base::out);
    // build a visualizer and render the solution in our chosen format
    vis.csv(potential, output_stream);

    // compute the error field
    Grid error(N);
    relative_error(potential, solution, error);
    std::fstream error_stream("error.csv", std::ios_base::out);
    vis.csv(error, error_stream);

    // all done
    return 0;
}
  \end{lstlisting}
%
\end{frame}

% --------------------------------------
% computing the exact and error fields
\begin{frame}[fragile]
% 
  \frametitle{Computing the exact solution and the error field}
%
  \begin{lstlisting}[language=c++,firstnumber=143]
void exact(Grid & grid) {
    //  paint the exact solution
    for (size_t j=0; j < grid.size(); j++) {
        for (size_t i=0; i < grid.size(); i++) {
            double x = i*grid.delta();
            double y = j*grid.delta();
            grid(i,j) = std::exp(-pi*y)*std::sin(pi*x);
        }
    }
    return;
}

void relative_error(
    const Grid & computed, const Grid & exact, Grid & error) {
    //  compute the relative error
    for (size_t j=0; j < exact.size(); j++) {
        for (size_t i=0; i < exact.size(); i++) {
            if (exact(i,j) == 0.0) { // hm... sloppy!
                error(i,j) = std::abs(computed(i,j));
            } else {
                error(i,j) = std::abs(computed(i,j) - exact(i,j))/exact(i,j);
            }
        }
    }
    return;
}

  \end{lstlisting}
%
\end{frame}

% --------------------------------------
% summary
\begin{frame}[fragile]
%
  \frametitle{Shortcomings}
%
  \begin{itemize}
%
  \item numerics:
    \begin{itemize}
    \item it converges very slowly; other update {\em schemes} improve on this
    \item our approximation is very low order, so it takes very large grids to produce a few
      digits of accuracy
    \item the convergence criterion has some unwanted properties; it triggers
      \begin{itemize}
      \item prematurely: large swaths of constant values may never get updated
      \item it would trigger if we were updating the wrong grid!
      \end{itemize}
    \end{itemize}
%
  \item design:
    \begin{itemize}
    \item separate the problem specification from its solution
    \item there are other objects lurking, waiting to be uncovered
    \item someone should make the graphic visualizer
    \item restarts anybody?
    \item how would you try out different convergence criteria? update schemes? memory layouts?
    \end{itemize}
%
    \item usability:
      \begin{itemize}
      \item supporting interchangeable parts requires damage to the top level driver
        \begin{itemize}
        \item to enable the user to make the selection
        \item to expose new command line arguments that configure the new parts
        \end{itemize}
      \end{itemize}
%
  \end{itemize}
%
\end{frame}

% end of file 
