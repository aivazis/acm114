% -*- LaTeX -*-
% -*- coding: utf-8 -*-
%
% ~~~~~~~~~~~~~~~~~~~~~~~~~~~~~~~~~~~~~~~~~~~~~~~~~~~~~~~~~~~~~~~~~~~~~~~~~~~~~~
%
%                             michael a.g. aïvázis
%                      california institute of technology
%                      (c) 1998-2010  all rights reserved
%
% ~~~~~~~~~~~~~~~~~~~~~~~~~~~~~~~~~~~~~~~~~~~~~~~~~~~~~~~~~~~~~~~~~~~~~~~~~~~~~~
%

\lecture{Overview of algorithms}{20100106}

% the INSERTION-SORT algorithm
\begin{frame}[fragile]
%
  \frametitle{A sorting algorithm}
%
    \begin{center}
      \parbox[]{.75\linewidth}{
        \begin{algorithm}[H]
%
          \dontprintsemicolon
          \nocaptionofalgo
          \setalcaphskip{0ex}
%
          \caption{\sc Insertion-Sort($S$)}
          \vspace{.5em}
%
          \For{$j \leftarrow 2$  \KwTo length[S]}{
            $key \leftarrow S[j]$ \;
            $i \leftarrow j-1$ \;
            \While{$i > 0$ \KwAnd $S[i] > key$}{
              $S[i+1] \leftarrow S[i]$ \;
              $i \leftarrow i-1$ \;
            }
            $S[i+1] \leftarrow key$
          }
          \vspace{.5em}
%
        \end{algorithm}
      }
    \end{center}
%
  \begin{itemize}
%
  \item valid inputs:
    \begin{itemize}
    \item empty sequence, singlet, other sequences of finite length
    \item what kinds of objects in $S$?
    \end{itemize}
%
  \item walk through it by hand with $S = (5, 2, 4, 6, 1, 3)$
%
  \end{itemize}
%
\end{frame}


\begin{frame}[fragile]
%
  \frametitle{Pseudocode conventions}
%
  \begin{itemize}
  \item the character $\triangleright$ indicates a comment through to the end of the line
  \item block structure is indicated by the indentation level
  \item all variables are local; no global variables, unless explicitly marked
  \item $i \leftarrow j \leftarrow k$ assigns the rightmost expression to all the other
    variables
  \item indexing: $S[i]$; slicing: $S[i .. j]$
  \item conditionals, looping constructs, function calls should be familiar
  \item compound objects have attributes or fields that are referenced using indexing,
    e.g. $length[S]$
  \item variables assigned to objects or containers are references
  \item parameters passed to procedures {\em by assignment}
  \end{itemize}
%
\end{frame}
        

% ------------------

% end of file 
