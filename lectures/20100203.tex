% -*- LaTeX -*-
% -*- coding: utf-8 -*-
%
% ~~~~~~~~~~~~~~~~~~~~~~~~~~~~~~~~~~~~~~~~~~~~~~~~~~~~~~~~~~~~~~~~~~~~~~~~~~~~~~
%
%                             michael a.g. aïvázis
%                      california institute of technology
%                      (c) 1998-2010  all rights reserved
%
% ~~~~~~~~~~~~~~~~~~~~~~~~~~~~~~~~~~~~~~~~~~~~~~~~~~~~~~~~~~~~~~~~~~~~~~~~~~~~~~
%

\lecture{Functional decomposition}{20100203}

% --------------------------------------
% functional decomposition
\begin{frame}[fragile]
%
  \frametitle{Functional decomposition}
%
  \begin{itemize}
%
  \item {\em functional decomposition} determines the fine grain parallel tasks by partitioning
    the problem into semi-independent tasks that can be executed in parallel
%
  \item our numerical integration examples fall in this category
    \begin{itemize}
    \item partitioning identified the finest grain work unit as the evaluation of the
      integrand; no need for fancy domain decomposition
    \item little or no communication/synchronization is required among the tasks, i.e. the are
      embarrassingly parallel
    \item coarsening consists of grouping fine grain tasks into larger work units in a straight
      forward manner
    \item the mapping of the coarse grain tasks onto processing units is trivial
    \end{itemize}
%
  \item in general, the computations involved in carrying out the coarse tasks are
    computationally equivalent
    \begin{itemize}
    \item the computation is {\em self balancing}
    \item or there is no need for sophisticated load balancing
    \end{itemize}
%
  \item scalability and parallel efficiency are determined by the particulars of the problem,
    such as inherent limitations on the largest problem size of interest
%
  \end{itemize}
%
\end{frame}

% --------------------------------------
% relevant problems
\begin{frame}[fragile]
%
  \frametitle{Blank}
%
  \begin{itemize}
%
  \item a variety of interesting problems can be cast in this form
%
  \end{itemize}
%
\end{frame}


% end of file 
