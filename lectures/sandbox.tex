% -*- LaTeX -*-
% -*- coding: utf-8 -*-
%
% ~~~~~~~~~~~~~~~~~~~~~~~~~~~~~~~~~~~~~~~~~~~~~~~~~~~~~~~~~~~~~~~~~~~~~~~~~~~~~~
%
%                             michael a.g. aïvázis
%                      california institute of technology
%                      (c) 1998-2010  all rights reserved
%
% ~~~~~~~~~~~~~~~~~~~~~~~~~~~~~~~~~~~~~~~~~~~~~~~~~~~~~~~~~~~~~~~~~~~~~~~~~~~~~~
%

\lecture{Introduction to scientific computing}{20100106}

% ------------------
\section{Scientific computing}

\subsection{Here we go}
\begin{frame}{The title of the first slide}

  \begin{itemize}
    \item Here is an important point
      \begin{algorithm}[H]
        %
        \dontprintsemicolon
        %
        $i \leftarrow 0$ \;
        $t \leftarrow 0$ \;
        \While{$t < N$}{
          $x \leftarrow \mbox{random()}$ \;
          $y \leftarrow \mbox{random()}$ \;
          \If{$\sqrt{x^{2}+y^{2}} \leq 1$}{ \nllabel{line:pi:check}
            $i \leftarrow i + 1$
          }
          $t \leftarrow t + 1$
        }
        $\pi_{N} = 4 i/t \;$ %\frac{i}{t}$ \;
      \end{algorithm}
%
      \begin{itemize}
        \item which is explained here
      \end{itemize}
  \end{itemize}
\end{frame}

\subsection{And once again}
\begin{frame}[fragile]
  \frametitle{The title of the second slide}
  \begin{itemize}
    \item Here is another important point
      {\tiny
      \begin{python}[h]
        # get access to the random number generator functions
        import random

        # sample size
        N = 10**6

        # initialize the counters
        total = 0
        interior = 0

        # integrate by sampling some number of times
        while total < N:
        # generate a random point
        x = random.random()
        y = random.random()
        # check whether it is inside the unit quarter circle
        if (x*x + y*y) <= 1.0: # no need to waste time computing the square root!
            # update the interior point counter
             interior += 1
        # update the total number of points
        total += 1

        # print the result
        # note that as of python 3.0, division of integers yields floats 
        print("pi: {0:.8f}".format(4 * interior / total))
      \end{python}
      }

      \begin{itemize}
        \item which is explained here
      \end{itemize}
  \end{itemize}
\end{frame}

% ------------------
\section{Algorithms}

\subsection{Here we go}
\begin{frame}{The title of the first slide}

  \begin{itemize}
    \item Here is an important point
      \begin{itemize}
        \item which is explained here
      \end{itemize}
  \end{itemize}
\end{frame}

\subsection{And once again}
\begin{frame}{The title of the second slide}
  \begin{itemize}
    \item Here is another important point
      \begin{itemize}
        \item which is explained here
      \end{itemize}
  \end{itemize}
\end{frame}

% end of file 
