% -*- LaTeX -*-
% -*- coding: utf-8 -*-
%
% ~~~~~~~~~~~~~~~~~~~~~~~~~~~~~~~~~~~~~~~~~~~~~~~~~~~~~~~~~~~~~~~~~~~~~~~~~~~~~~
%
%                             michael a.g. aïvázis
%                      california institute of technology
%                      (c) 1998-2010  all rights reserved
%
% ~~~~~~~~~~~~~~~~~~~~~~~~~~~~~~~~~~~~~~~~~~~~~~~~~~~~~~~~~~~~~~~~~~~~~~~~~~~~~~
%

\lecture{Structured grids - III}{20100212}

% --------------------------------------
% the results of initialization
\begin{frame}[fragile]
%
  \frametitle{Printing out the initial grid}
%
  \begin{itemize}
  \item we should be able to print out the initialized grid
%
  \begin{shell}{}
#> mm laplace
#> laplace
#> cat laplace.csv
0,0,0.3827,0.7071,0.9239,1,0.9239,0.7071,0.3827,1.225e-16
1,0,1,1,1,1,1,1,1,0
2,0,1,1,1,1,1,1,1,0
3,0,1,1,1,1,1,1,1,0
4,0,1,1,1,1,1,1,1,0
5,0,1,1,1,1,1,1,1,0
6,0,1,1,1,1,1,1,1,0
7,0,1,1,1,1,1,1,1,0
8,0,0.01654,0.0306,0.03992,0.04321,0.0399,0.03056,0.01654,0
  \end{shell}
%
  \item notice that
    \begin{itemize}
    \item the top line contains some recognizable values
    \item the left and right borders are set to zero
    \item the interior of the grid is painted with our initial guess
    \end{itemize}
%
  \item still to do:
    \begin{itemize}
    \item write the update
    \item build a grid with the exact solution 
    \item build the error field (why?)
    \end{itemize}
  \end{itemize}
%
\end{frame}

% --------------------------------------
% the solver
\begin{frame}[fragile]
%
  \frametitle{Fleshing out the solver}
%
  \begin{lstlisting}[language=c++,basicstyle=\tt\bfseries\tiny,name=seq:solver,firstnumber=169]
// the solver driver
void laplace(Grid & current, double tolerance) {
    // create and initialize temporary storage
    Grid next(current.size());
    initialize(next);
    // put an upper bound on the number of iterations
    long max_iterations = (long) 1e4;;
    for (long iterations = 0; iterations<max_iterations; iterations++) {
        double max_dev = 0.0;
        // do an iteration step
        // leave the boundary alone
        // iterate over the interior of the grid
        for (size_t j=1; j < current.size()-1; j++) {
            for (size_t i=1; i < current.size()-1; i++) {
                // update
                next(i,j) = 0.25*(
                    current(i+1,j)+current(i-1,j)+current(i,j+1)+current(i,j-1));
                // compute the deviation from the last generation
                double dev = std::abs(next(i,j) - current(i,j));
                // and update the maximum deviation
                if (dev > max_dev) {
                    max_dev = dev;
                }
            }
        }
        // swap the blocks between the two grids
        Grid::swapBlocks(current, next);
        // check covergence
        if (max_dev < tolerance) {
            break;
        }
    }
    return;
}
  \end{lstlisting}
%
\end{frame}

% --------------------------------------
% the updated grid class
\begin{frame}[fragile]
%
  \frametitle{Adding the new grid interface}
%
  \begin{itemize}
  \item here is the declaration of \function{Grid::swapBlocks}
%
    \begin{lstlisting}[language=c++,firstnumber=30]
class Grid {
    // interface
    public:
    ...
    // exchange the data blocks of two compatible grids
    static void swapBlocks(Grid &, Grid &);
    ...
};
    \end{lstlisting}
%
  \item and its definition
%
    \begin{lstlisting}[language=c++,firstnumber=69]
void Grid::swapBlocks(Grid & g1, Grid & g2) {
    // bail out if the two operands are not compatible
    if (g1.size() != g2.size()) {
        throw "Grid::swapblocks: size mismatch";
    }
    if (g1.delta() != g2.delta()) {
        throw "Grid::swapblocks: spacing mismatch";
    }
    // but if they are, just exhange their data buffers
    double * temp = g1._block;
    g1._block = g2._block;
    g2._block = temp;
    // all done
    return;
}
    \end{lstlisting}
%
  \end{itemize}
%
\end{frame}

% --------------------------------------
% rework the driver to print out the exact solution and the error field
\begin{frame}[fragile]
%
  \frametitle{Reworking the driver}
%
  \begin{lstlisting}[language=c++,basicstyle=\tt\bfseries\tiny,name=seq:driver,firstnumber=239]
    // build a visualizer
    Visualizer vis;

    // compute the exact solution
    Grid solution(N);
    exact(solution);
    std::fstream exact_stream("exact.csv", std::ios_base::out);
    vis.csv(solution, exact_stream);
    
    // allocate space for the solution
    Grid potential(N);
    // initialize and apply our boundary conditions
    initialize(potential);
    // call the solver
    laplace(potential, tolerance);
    // open a stream to hold the answer
    std::fstream output_stream(filename, std::ios_base::out);
    // build a visualizer and render the solution in our chosen format
    vis.csv(potential, output_stream);

    // compute the error field
    Grid error(N);
    relative_error(potential, solution, error);
    std::fstream error_stream("error.csv", std::ios_base::out);
    vis.csv(error, error_stream);

    // all done
    return 0;
}
  \end{lstlisting}
%
\end{frame}

% --------------------------------------
% computing the exact and error fields
\begin{frame}[fragile]
% 
  \frametitle{Computing the exact solution and the error field}
%
  \begin{lstlisting}[language=c++,firstnumber=143]
void exact(Grid & grid) {
    //  paint the exact solution
    for (size_t j=0; j < grid.size(); j++) {
        for (size_t i=0; i < grid.size(); i++) {
            double x = i*grid.delta();
            double y = j*grid.delta();
            grid(i,j) = std::exp(-pi*y)*std::sin(pi*x);
        }
    }
    return;
}

void relative_error(
    const Grid & computed, const Grid & exact, Grid & error) {
    //  compute the relative error
    for (size_t j=0; j < exact.size(); j++) {
        for (size_t i=0; i < exact.size(); i++) {
            if (exact(i,j) == 0.0) { // hm... sloppy!
                error(i,j) = std::abs(computed(i,j));
            } else {
                error(i,j) = std::abs(computed(i,j) - exact(i,j))/exact(i,j);
            }
        }
    }
    return;
}

  \end{lstlisting}
%
\end{frame}

% --------------------------------------
% assessment
\begin{frame}[fragile]
%
  \frametitle{Shortcomings}
%
  \begin{itemize}
%
  \item numerics:
    \begin{itemize}
    \item it converges very slowly; other update {\em schemes} improve on this
    \item our approximation is very low order, so it takes very large grids to produce a few
      digits of accuracy
    \item the convergence criterion has some unwanted properties; it triggers
      \begin{itemize}
      \item prematurely: large swaths of constant values may never get updated
      \item it would trigger even if we were updating the wrong grid!
      \end{itemize}
    \end{itemize}
%
  \item design:
    \begin{itemize}
    \item separate the problem specification from its solution
    \item there are other objects lurking, waiting to be uncovered
    \item someone should make the graphic visualizer
    \item restarts anybody?
    \item how would you try out different convergence criteria? update schemes? memory layouts?
    \end{itemize}
%
    \item usability:
      \begin{itemize}
      \item supporting interchangeable parts requires damage to the top level driver
        \begin{itemize}
        \item to enable the user to make the selection
        \item to expose new command line arguments that configure the new parts
        \end{itemize}
      \end{itemize}
%
  \end{itemize}
%
\end{frame}

% --------------------------------------
% assessment
\begin{frame}[fragile]
%
  \frametitle{Assessing our fundamental}
%
  \begin{itemize}
%
  \item \class{Grid} is a good starting point for abstracting structured grids
    \begin{itemize}
    \item assumes ownership of the memory associated with a structured grid
    \item encapsulates the indexing function
    \item extend it to
      \begin{itemize}
      \item support different memory layout strategies
      \item support non-square grids (?)
      \item support non-uniform grids (?)
      \item higher dimensions
      \item if you need any of these, consider using one of the many excellent class libraries
        written by experts
      \end{itemize}
    \end{itemize}
%
  \item \class{Visualizer}, under another name, can form the basis for a more general
    persistence library
    \begin{itemize}
    \item to support HDF5, NetCDF, bitmaps, voxels, etc.
    \end{itemize}
%
  \end{itemize}
%
\end{frame}

% --------------------------------------
% the Problem class
\begin{frame}[fragile]
% 
  \frametitle{The \class{Problem} class: the interface}
%
  \begin{lstlisting}[language=c++,name=Problem]
// the solution representation
class acm114::laplace::Problem {
    //typedefs
public:
    typedef std::string string_t;
    // interface
public:
    inline string_t name() const;
    inline const Grid & exact() const;
    inline const Grid & deviation() const;
    inline Grid & solution();
    inline const Grid & solution() const;
    inline Grid & error();
    inline const Grid & error() const;
    // abstract
    virtual void initialize() = 0;
    virtual void initialize(Grid &) const = 0;
    // meta methods
public:
    inline Problem(string_t name, double width, size_t points);
    virtual ~Problem();
    // data members
  \end{lstlisting}
%
\end{frame}

% --------------------------------------
% the Problem class
\begin{frame}[fragile]
% 
  \frametitle{The \class{Problem} class: the data}
%
  \begin{lstlisting}[language=c++,name=Problem]
protected:
    string_t _name;
    double _delta;
    Grid _solution;
    Grid _exact;
    Grid _error;
    Grid _deviation;
    // disable these
private:
    Problem(const Problem &);
    const Problem & operator= (const Problem &);
};
  \end{lstlisting}
%
\end{frame}

% --------------------------------------
% our specific example
\begin{frame}[fragile]
% 
  \frametitle{The \class{Example} class}
%
  \begin{lstlisting}[language=c++]
class acm114::laplace::Example : public acm114::laplace::Problem {
    // interface
public:
    virtual void initialize();
    virtual void initialize(Grid &) const;

    // meta methods
public:
    inline Example(string_t name, double width, size_t points);
    virtual ~Example();

    // disable these
private:
    Example(const Example &);
    const Example & operator= (const Example &);
};

  \end{lstlisting}
%
\end{frame}

% --------------------------------------
% the solver basis
\begin{frame}[fragile]
% 
  \frametitle{The \class{Solver} class}
%
  \begin{lstlisting}[language=c++]
class acm114::laplace::Solver {
    // interface
public:
    virtual void solve(Problem &) = 0;

    // meta methods
public:
    inline Solver();
    virtual ~Solver();

    // data members
private:

    // disable these
private:
    Solver(const Solver &);
    const Solver & operator= (const Solver &);
};

  \end{lstlisting}
%
\end{frame}

% --------------------------------------
% the jacobi solver 
\begin{frame}[fragile]
% 
  \frametitle{The \class{Jacobi} class}
%
  \begin{lstlisting}[language=c++]
class acm114::laplace::Jacobi : public acm114::laplace::Solver {
    // interface
public:
    virtual void solve(Problem &);

    // meta methods
public:
    inline Jacobi(double tolerance, size_t workers);
    virtual ~Jacobi();

    // implementation details
protected:
    virtual void _solve(Problem &);
    static void * _update(void *);

    // data members
private:
    double _tolerance;
    size_t _workers;

    // disable these
private:
    Jacobi(const Jacobi &);
    const Jacobi & operator= (const Jacobi &);
};

  \end{lstlisting}
%
\end{frame}

% end of file 
