% -*- LaTeX -*-
% -*- coding: utf-8 -*-
%
% ~~~~~~~~~~~~~~~~~~~~~~~~~~~~~~~~~~~~~~~~~~~~~~~~~~~~~~~~~~~~~~~~~~~~~~~~~~~~~~
%
%                             michael a.g. aïvázis
%                      california institute of technology
%                      (c) 1998-2010  all rights reserved
%
% ~~~~~~~~~~~~~~~~~~~~~~~~~~~~~~~~~~~~~~~~~~~~~~~~~~~~~~~~~~~~~~~~~~~~~~~~~~~~~~
%

\pagestyle{headandfoot}
\runningfootrule
\firstpageheader{ACM/CS 114}{Assignment 2}{Due: 27 Jan 2010}
\runningheader{}{}{}
\firstpagefooter{}{}{}
\runningfooter{ACM/CS 114}{Assignment 2}{\thepage}



% --------------------------------------
% reduction using threads
\def\Li{\mbox{\rm Li}_{2}}
\def\len{\mbox{\rm length}}
\def\dilog{\mbox{\tt dilog}}

A better definition for the dilog $\Li$ is given by
%
\begin{equation}
\Li(z) \bydef
- \int_{0}^{z} dz' \; \frac{\log(1-z')}{z'} \label{eq:li-def}
\end{equation}
%
This is well defined for arbitrary complex values $z$, with a branch cut along the real axis
for $z > 1$. It can be shown that
%
\begin{eqnarray}
  \Li(1)  & = & \frac{\pi^{2}}{6} \label{eq:li1} \\
  \Li(-1) & = & - \frac{\pi^{2}}{12} \label{eq:li-1}
\end{eqnarray}
%
We will try to establish that \eqref{li1} and \eqref{li-1} are true by numerically evaluating
the integral in \eqref{li-def} both sequentially and {\em in parallel} using threads and \mpi.
%
We will employ the following simple numerical integration scheme:
%
\begin{itemize}
 \item partition the interval $(-1,0)$ into a large number $n$ of subintervals of equal length
 \item evaluate the integrand of \eqref{goal} at the center of each subinterval (why?)
 \item multiply this value by the width of the subinterval to form its contribution to the
   integral
 \item sum up all the contributions
\end{itemize}

\begin{questions}

\question Write a function \function{integrand} that evaluates the integrand of \eqref{goal}
for a given $z$. Assume $z$ is real.

\question Write a function \function{integrator} that implements our integration scheme, given
an interval $(a, b)$ and the desired number of subdivisions $n$.

\question Explain how you would use these two functions to compute an approximation of the
integral in \eqref{goal} {\em in parallel}. In particular, 
\begin{parts}
  \part How would you partition the work into fine grain tasks?
  \part What are the communication requirements among these tasks?
  \part What is the correct coarsening strategy?
  \part How you would map the coarse tasks onto physical processing units?
\end{parts}

\question 

\begin{parts}
  \part Implement this strategy using pthreads.
  \part Implement this strategy using \mpi.
  \part Fill out the following  table
\end{parts}

\end{questions}

% end of file 
