% -*- LaTeX -*-
% -*- coding: utf-8 -*-
%
% ~~~~~~~~~~~~~~~~~~~~~~~~~~~~~~~~~~~~~~~~~~~~~~~~~~~~~~~~~~~~~~~~~~~~~~~~~~~~~~
%
%                             michael a.g. aïvázis
%                      california institute of technology
%                      (c) 1998-2010  all rights reserved
%
% ~~~~~~~~~~~~~~~~~~~~~~~~~~~~~~~~~~~~~~~~~~~~~~~~~~~~~~~~~~~~~~~~~~~~~~~~~~~~~~
%

\pagestyle{headandfoot}
\runningfootrule
\firstpageheader{ACM/CS 114}{Assignment 1}{Due: 20 Jan 2010}
\runningheader{}{}{}
\firstpagefooter{}{}{}
\runningfooter{ACM/CS 114}{Assignment 1}{\thepage}

\begin{questions}

% --------------------------------------
% merge sort
\question
\mergesort\ is an example of a {\em divide-and conquer} algorithm. The algorithm
sorts an input sequence $S$ of numbers along the following steps:
%
\begin{itemize}
\item {\em divide}: split $S$ into two parts of roughly equal length
\item {\em conquer}: sort the subsequences recursively
\item {\em combine}: merge the two sorted subsequences to produce the sorted output
\end{itemize}
%
In pseudocode:
%
\begin{center}
  \begin{minipage}{.5\linewidth}
    \begin{algorithm}[H]
      \label{alg:merge-sort}
%
      \dontprintsemicolon
      %\nocaptionofalgo
      \setalcaphskip{0ex}
%
      \caption{\mergesort($S$, $p$, $r$)}
      \vspace{.5em}
%
      \If{$p < r$}{
        $q \leftarrow \lfloor (p+r)/2 \rfloor$ \;
        \mergesort($S$, $p$, $q$) \;
        \mergesort($S$, $q+1$, $r$) \;
        \merge($S$, $p$, $q$, $r$) \;
      }
%
      \vspace{.5em}
%
    \end{algorithm}
  \end{minipage}
\end{center}
%
\begin{parts}

\part Explain the role of $p$ and $r$ in the algorithm specification. What values should they
have upon initial invocation of the algorithm?

\part Write \merge. It was claimed in class that \merge\ can be implemented to run in
$\Theta(r-p+1)$ time. How does your implementation compare?

\part Implement \mergesort\ in python. 
Write a driver that invokes it with $S = (5, 2, 4, 6, 1, 3)$.

\end{parts}

% --------------------------------------
% reduction tree
\question implement the reduction in the lecture notes using threads

% --------------------------------------
% reduction tree
\question {\em reduction tree}

\end{questions}

% end of file 
