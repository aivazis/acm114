% -*- LaTeX -*-
% -*- coding: utf-8 -*-
%
% ~~~~~~~~~~~~~~~~~~~~~~~~~~~~~~~~~~~~~~~~~~~~~~~~~~~~~~~~~~~~~~~~~~~~~~~~~~~~~~
%
%                             michael a.g. aïvázis
%                      california institute of technology
%                      (c) 1998-2010  all rights reserved
%
% ~~~~~~~~~~~~~~~~~~~~~~~~~~~~~~~~~~~~~~~~~~~~~~~~~~~~~~~~~~~~~~~~~~~~~~~~~~~~~~
%

\pagestyle{headandfoot}
\runningfootrule
\firstpageheader{ACM/CS 114}{Assignment 1}{Due: 20 Jan 2010}
\runningheader{}{}{}
\firstpagefooter{}{}{}
\runningfooter{ACM/CS 114}{Assignment 1}{\thepage}

\begin{questions}

% --------------------------------------
% merge sort
\question
\mergesort\ is an example of a {\em divide-and conquer} algorithm. The algorithm
sorts an input sequence $S$ of numbers along the following steps:
%
\begin{itemize}
\item {\em divide}: split $S$ into two parts of roughly equal length
\item {\em conquer}: sort the subsequences recursively
\item {\em combine}: merge the two sorted subsequences to produce the sorted output
\end{itemize}
%
In pseudocode:
%
\begin{center}
  \begin{minipage}{.5\linewidth}
    \begin{algorithm}[H]
      \label{alg:merge-sort}
%
      \dontprintsemicolon
      %\nocaptionofalgo
      \setalcaphskip{0ex}
%
      \caption{\mergesort($S$, $p$, $r$)}
      \vspace{.5em}
%
      \If{$p < r$}{
        $q \leftarrow \lfloor (p+r)/2 \rfloor$ \;
        \mergesort($S$, $p$, $q$) \;
        \mergesort($S$, $q+1$, $r$) \;
        \merge($S$, $p$, $q$, $r$) \;
      }
%
      \vspace{.5em}
%
    \end{algorithm}
  \end{minipage}
\end{center}
%
\begin{parts}

\part Explain the role of $p$ and $r$ in the algorithm specification. What values should they
have upon initial invocation of the algorithm?

\part Write \merge. It was claimed in class that \merge\ can be implemented to run in
$\Theta(r-p+1)$ time. How does your implementation compare?

\part Implement \mergesort\ in a language of your choice. 
\begin{subparts}
  \subpart Write a driver that invokes it with $S = (5, 2, 4, 6, 1, 3)$.
  \subpart Build a container with $10^6$ random numbers. Sort it using your implementation.
\end{subparts}

\end{parts}

% --------------------------------------
% reduction 
\def\Li{\mbox{\rm Li}_{2}}
\def\len{\mbox{\rm length}}
\def\dilog{\mbox{\tt dilog}}
\def\sdlog{\mbox{\tt sdilog}}

\question Consider the function $\Li$ defined for $|z| \leq 1$ by
\[
\Li(z) \bydef
     - \sum_{n=1}^{\infty} \frac{z^{n}}{n}
     =
     - z - \frac{z^{2}}{2} - \frac{z^{3}}{3} - \frac{z^{4}}{4} - \cdots
\]

\begin{parts}

\part In the language of your choice, implement a procedure \dilog($z$, $n$) that computes the
sum of the first $n$ terms of the series for the given floating point number $z$.

\part Implement a procedure \sdlog($Z$, $n$) that accepts a sequence $Z$ of floating point
numbers and computes the sum
\[
\sdlog(Z, n) \bydef \sum_{z \in Z} \dilog(z, n)
\]

\part Build a cost model for \sdlog\ as a function of $n$ and $m \bydef \len(Z)$. Assume that
additions and subtractions cost $c_{+}$ each, multiplications and divisions cost $c_{\times}$
each, and that the raising a number to the $n$th power costs $c_{\star}$.

\end{parts}
        
\end{questions}

% end of file 
